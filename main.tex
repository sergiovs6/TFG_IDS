% !TeX spellcheck = es_ES
\documentclass[12pt, a4paper, twoside]{report}   % Tamaño de papel, de fuente y márgenes

% Codificación e idioma
\usepackage[spanish,es-tabla]{babel}
% \usepackage[spanish,es-tabla,shorthands=off]{babel}
\usepackage[utf8]{inputenc}
\usepackage[T1]{fontenc}

% Matemáticas y símbolos
\usepackage{amsmath}
\usepackage{amssymb} % para \varnothing, etc.

% Tablas e imágenes
\usepackage{longtable}
\usepackage{caption}
\usepackage{float}
\usepackage{graphicx}
\usepackage{adjustbox}
\usepackage{pdflscape}
\usepackage{calc}

% Diagramas
\usepackage{tikz}
\usetikzlibrary{calc, positioning}

% SVG (si NO compilas con --shell-escape o no tienes Inkscape, comenta estas 2 líneas)
% \usepackage{svg}
% \svgsetup{inkscapearea=page,inkscapeversion=1}

% Algoritmos (sin columna de comentarios; líneas numeradas)
\usepackage[ruled,linesnumbered]{algorithm2e}
\SetAlgoLined
\DontPrintSemicolon
\renewcommand{\algorithmcfname}{Algoritmo}
\SetKw{KwRet}{retornar}
\SetKwInOut{KwIn}{Entrada}
\SetKwInOut{KwOut}{Salida}
\SetKwFor{ForCada}{Para cada}{:}{fin} % alias en español

% Código (listings)
\usepackage{xcolor}
\usepackage{listings}

\lstdefinestyle{csvline}{
  basicstyle=\ttfamily\small,
  columns=fullflexible,
  breaklines=true,
  breakatwhitespace=false,
  literate={,}{,\allowbreak}1,
  postbreak=\mbox{\tiny$\hookrightarrow$ },
  showstringspaces=false,
  frame=single,
  xleftmargin=0pt
}

\lstdefinestyle{tfgpython}{
  language=Python,
  basicstyle=\ttfamily\small,
  keywordstyle=\color{blue!70!black}\bfseries,
  stringstyle=\color{orange!60!black},
  commentstyle=\color{teal!70!black},
  numberstyle=\tiny\color{gray},
  numbers=left,
  stepnumber=1,
  numbersep=8pt,
  showstringspaces=false,
  breaklines=true,
  breakatwhitespace=true,
  columns=fullflexible,
  tabsize=2,
  keepspaces=true,
  literate=
   {á}{{\'a}}1 {é}{{\'e}}1 {í}{{\'\i}}1 {ó}{{\'o}}1 {ú}{{\'u}}1
   {Á}{{\'A}}1 {É}{{\'E}}1 {Í}{{\'I}}1 {Ó}{{\'O}}1 {Ú}{{\'U}}1
   {ñ}{{\~n}}1 {Ñ}{{\~N}}1
   {¿}{{\textquestiondown}}1 {¡}{{\textexclamdown}}1
   {→}{{$\to$}}1
}

\lstdefinestyle{tfgbash}{
  language=bash,
  basicstyle=\ttfamily\small,
  commentstyle=\color{teal!70!black},
  stringstyle=\color{orange!70!black},
  showstringspaces=false,
  breaklines=true,
  columns=fullflexible,
  keepspaces=true,
  literate=
   {á}{{\'a}}1 {é}{{\'e}}1 {í}{{\'\i}}1 {ó}{{\'o}}1 {ú}{{\'u}}1
   {Á}{{\'A}}1 {É}{{\'E}}1 {Í}{{\'I}}1 {Ó}{{\'O}}1 {Ú}{{\'U}}1
   {ñ}{{\~n}}1 {Ñ}{{\~N}}1
   {¿}{{\textquestiondown}}1 {¡}{{\textexclamdown}}1
}

\newcommand{\codepy}[3]{\begin{lstlisting}[style=tfgpython,caption={#1},label={#2}]#3\end{lstlisting}}
\newcommand{\codebash}[3]{\begin{lstlisting}[style=tfgbash,caption={#1},label={#2}]#3\end{lstlisting}}

% Otros
\usepackage{eurosym}
\usepackage{changepage}
\usepackage{ifoddpage}

% SIEMPRE AL FINAL
\usepackage{hyperref}

% ============================================================|
%                                                             |
% Plantilla para TFG del Dpto. de Informática de la EPSJ v1.1 |
%                                                             |
% Ajustada según la normativa de estilo indicada por la EPSJ  |
% en el documento https://eps.ujaen.es/sites/centro_epsj/files|
% /uploads/documents/grados/TFG/criteriosYestilo_TFG.pdf      |
%                                                             |
%    - A4 con márgenes de 2.5cm                               |
%    - Intelineado de 1.5 líneas                              |
%    - Primera línea de cada párrafo con sangrado             |
%    - Fuente Arial de 12pt                                   |
%    - Cabecera con nombre de estudiante y título de TFG      |
%    - Pie con nombre del centro y nº de página               |
%    - Portada oficial EPSJ - Dpto. Informática con datos     |
%                                                             |
% 2020/09/25 - Francisco Charte Ojeda - Versión inicial       |
% 2024/07/19 - Francisco Charte Ojeda - Ajustes para generar  |
%              la nueva portada de la EPSJ                    |
% CC0 1.0 Universal (CC0 1.0)                                 |
% https://creativecommons.org/publicdomain/zero/1.0/deed.es   |
%                                                             |
%=============================================================|

% ==== Introducir aquí el nombre del estudiante
\def\Estudiante{Sergio Villar Serrano}

% ==== Introducir aquí el nombre de los tutores. Si solo hay uno dejar las llaves de \TutorB vacías
\def\TutorA{Prof. D. Manuel José Lucena López}
\def\TutorB{}

% ==== Introducir aquí el título de completo y abreviado (para las cabeceras) del TFG
\def\TituloTFG{Sistema de detección de intrusiones basado en aprendizaje automático}
\def\TituloAbreviado{Machine learning y ciberseguridad}

% ==== Introducir aquí la fecha de presentación del TFG en el formato día/mes/año
\def\Fecha{31/7/2024}

% ============================================================|
%                                                             |
% Plantilla para TFG del Dpto. de Informática de la EPSJ v1.1 |
%                                                             |
% Ajustada según la normativa de estilo indicada por la EPSJ  |
% en el documento https://eps.ujaen.es/sites/centro_epsj/files|
% /uploads/documents/grados/TFG/criteriosYestilo_TFG.pdf      |
%                                                             |
%    - A4 con márgenes de 2.5cm                               |
%    - Intelineado de 1.5 líneas                              |
%    - Primera línea de cada párrafo con sangrado             |
%    - Fuente Arial de 12pt                                   |
%    - Cabecera con nombre de estudiante y título de TFG      |
%    - Pie con nombre del centro y nº de página               |
%    - Portada oficial EPSJ - Dpto. Informática con datos     |
%                                                             |
% 2020/09/25 - Francisco Charte Ojeda - Versión inicial       |  
% 2024/07/19 - Francisco Charte Ojeda - Ajustes para generar  |
%              la nueva portada de la EPSJ                    |
% CC0 1.0 Universal (CC0 1.0)                                 |
% https://creativecommons.org/publicdomain/zero/1.0/deed.es   |
%                                                             |
%=============================================================|

\usepackage[margin=2.5cm]{geometry}

\usepackage{setspace} % Para fijar el interlineado

\usepackage{fancyhdr} % Para los encabezados y pies
\pagestyle{fancy}
\fancyhf{}

\usepackage{lipsum} % Para generar texto como contenido de prueba, se puede eliminar en la memoria final

\usepackage{helvet}

\usepackage{wallpaper}   % Para introducir el PDF con la portada EPSJ
\usepackage[absolute,overlay]{textpos}

\usepackage{etoolbox}  % Herramientas y bibliografía
\usepackage[numbers]{natbib}
\usepackage{csquotes}

\usepackage{hyperref}  % Para URL como enlaces
\hypersetup{
    colorlinks=true,
    linkcolor=blue,
    filecolor=magenta,      
    urlcolor=cyan,
    pdftitle={\TituloTFG},
    pdfauthor={\Estudiante}
    bookmarks=true,
    bookmarksopen=true,
    pdfpagemode=FullScreen,
    breaklinks=true,
    citecolor=cyan,
}

\usepackage{booktabs}  % Para tablas con mejor apariencia

\usepackage[ruled,vlined]{algorithm2e} % Para incluir algoritmos
\renewcommand{\algorithmcfname}{Algoritmo}
\renewcommand{\listalgorithmcfname}{Lista de algoritmos}

\usepackage{listings}  % Para incluir listados de código
\usepackage{xcolor}
\lstset{ %
  %language=delphi,                % the language of the code
  basicstyle=\linespread{0.7}\small\ttfamily,       % the size of the fonts that are used for the code
  numbers=left,                   % where to put the line-numbers
  numberstyle=\footnotesize\color{gray},  % the style that is used for the line-numbers
  stepnumber=1,                   % the step between two line-numbers. If it's 1, each line will be numbered
  numbersep=7pt,                  % how far the line-numbers are from the code
  backgroundcolor=\color{gray!5},  % choose the background color. You must add \usepackage{color}
  showspaces=false,               % show spaces adding particular underscores
  showstringspaces=true,         % underline spaces within strings
  showtabs=true,                 % show tabs within strings adding particular underscores
  frameround=fttt,
  frame=rtBL,                   % adds a frame around the code
  rulecolor=\color{black},        % if not set, the frame-color may be changed on line-breaks within not-black text (e.g. commens (green here))
  tabsize=4,                      % sets default tabsize to 2 spaces
  aboveskip=1em,
  captionpos=b,                   % sets the caption-position to bottom
  breaklines=true,                % sets automatic line breaking
  breakatwhitespace=false,        % sets if automatic breaks should only happen at whitespace
  title=\lstname,                 % show the filename of files included with \lstinputlisting;  also try caption instead of title
  keywordstyle=\bf\ttfamily,          % keyword style
  commentstyle=\color{black!60}\ttfamily,       % comment style
  stringstyle=\color{blue}\ttfamily,         % string literal style
  escapeinside={\%*}{*)}            % if you want to add a comment within your code
}
\renewcommand{\lstlistingname}{Listado}
\renewcommand{\lstlistlistingname}{Lista de listados de código}
\lstset
{
    language=[LaTeX]TeX,
    breaklines=true,
    basicstyle=\tt\scriptsize,
    keywordstyle=\color{blue},
    identifierstyle=\color{gray},
    texcl=true
}
\lstset{
    language=Python,
    literate={á}{{\'a}}1
        {ã}{{\~a}}1
        {é}{{\'e}}1
        {ó}{{\'o}}1
        {í}{{\'i}}1
        {ñ}{{\~n}}1
        {¡}{{!`}}1
        {¿}{{?`}}1
        {ú}{{\'u}}1
        {Í}{{\'I}}1
        {Ó}{{\'O}}1
}

\renewcommand{\familydefault}{\sfdefault}
\setlength{\parskip}{1em}

% Configuración de encabezado y pie
\fancyhead[RE,LO]{{\color{gray}\Estudiante}}
\fancyhead[LE,RO]{{\color{gray}\TituloAbreviado}}
\fancyfoot[RE,LO]{{\color{gray}Escuela Politécnica Superior de Jaén}}
\fancyfoot[LE,RO]{{\color{gray}\thepage}}
\renewcommand{\footrulewidth}{1pt}

\renewcommand{\spanishtablename}{Tabla.}  % Título para las tablas
\renewcommand{\spanishcontentsname}{Tabla de contenidos}  % y los índices
\renewcommand{\spanishlistfigurename}{Lista de figuras}
\renewcommand{\spanishlisttablename}{Lista de tablas}

% Comando que se encarga de componer la portada del TFG
% julio 2024 - Nueva portada la EPSJ para TFG
\definecolor{naranja}{RGB}{198,97,21}
\newcommand{\Portada}{ %
    \thispagestyle{empty}
    \ULCornerWallPaper{1}{imagenes/portada.pdf}
    \begin{textblock*}{18.45cm}(1.4cm,12cm) 
        \centering
        {\fontsize{26}{38}\selectfont \textbf{\color{naranja}\TituloTFG}}
      \vspace{.5cm}

      {\fontsize{16}{19}\selectfont Autor: \Estudiante}

      {\fontsize{14}{19}\selectfont Grado: Ingeniería en Informática}

      {\fontsize{12}{10}\selectfont \ifdefempty{\TutorB}{Director}{Directores}: \TutorA\ifdefempty{\TutorB}{}{ y \TutorB}}
      \vspace{-.4cm}
      
      {\fontsize{12}{10}\selectfont Departamento del director: Informática}
      
      \vspace{1cm}
      {\fontsize{14}{10}\selectfont Fecha: \Fecha}
    \end{textblock*}
    ~
    \clearpage
    \ClearWallPaper
    \thispagestyle{empty}
    \definecolor{flashwhite}{rgb}{0.95, 0.95, 0.96}
    \cleardoublepage

    \thispagestyle{empty}
    \begin{figure}
        \centering
        \includegraphics[width=.4\textwidth]{imagenes/uja.jpg}
    \end{figure}
    
    \vspace*{4em}
    
    D./Dª \TutorA \ifdefempty{\TutorB}{}{~y D./Dª \TutorB}, tutor(es) del Trabajo Fin de Grado titulado: \textbf{\TituloTFG}, que presenta \Estudiante, autoriza(n) su presentación para defensa y evaluación en la Escuela Politécnica Superior de Jaén.
    
    \vspace*{2em}
    \begin{center}
        Jaén, \Fecha
    \end{center}
    
    \clearpage\thispagestyle{empty}
    \onehalfspacing  % Fijamos el interlineado   
}


\begin{document}

    \Portada~			

    \pagenumbering{roman}  % Numeración romana para los agradecimientos, dedicatoria y tablas de contenidos

    \renewcommand{\abstractname}{Agradecimientos}
\begin{abstract}

En esta sección del trabajo, me gustaría expresar mi más sincero agradecimiento a cada una de las personas que me han apoyado y acompañado en la realización de este trabajo del cual sacó una gran experiencia y formación en este ámbito que tanto me apasiona.

En primer lugar, me gustaría mostrar mis muestras de cariño y agradecimiento a mis padres, José y Rosa, sin ellos hubiera sido incapaz de culminar este viaje. A pesar de todas las adversidades, siempre me han estado ayudando y apoyándome en todo lo posible, a veces, centrándose más en el desarrollo de mi trabajo que en sus propios problemas. Muchas gracias de corazón.

Destacar también, a muchos de mis amigos, por sus ánimos infinitos, su gran apoyo e inspiración, han hecho que siga adelante con mis sueños y objetivos haciendo de esta experiencia algo inimaginable.

En segundo lugar, me gustaría acordarme de cada uno de mis seres queridos que no están, sé que desde allí arriba me habías dado muchísimas fuerzas para nunca rendirme y dar todo lo mejor de mí. Allá donde estéis, muchas gracias por todo.

En tercer lugar, no me podía olvidar de mis abuelos, un pilar fundamental en mi vida, gracias a ellos, soy la persona que soy, y también, me gustaría acordarme de mi tito Antonio, un familiar muy especial para mí que desde muy pequeño me ha hecho ser muy feliz.

Por último y no menos importante, darle las gracias a mi Tutor, Manuel José Lucena, por brindarme tantos consejos y ánimos que me han servido para la culminación de un gran trabajo del cual extraigo un gran aprendizaje.

Muchas gracias a todos.


\end{abstract}

 % Editar este archivo para introducir los agradecimientos/dedicatoria

    \cleardoublepage
    \tableofcontents  % Tabla de contenidos

% ===== Comentar o eliminar las líneas de los índices que no deseen incluirse al inicio de la memoria

    \clearpage\thispagestyle{empty}\cleardoublepage
    \listoffigures		% Índice de figuras

    \clearpage\thispagestyle{empty}\cleardoublepage
    \listoftables 		% Índice de tablas

    \clearpage\thispagestyle{empty}\cleardoublepage
    \listofalgorithms
    
    \clearpage\thispagestyle{empty}\cleardoublepage
    \lstlistoflistings		% Índice de listados
          
    \clearpage\thispagestyle{empty}\cleardoublepage
    \pagenumbering{arabic} % Numeración arábiga para el resto del documento

% ===== Archivos LaTeX con los distintos capítulos que componen la memoria

% !TeX spellcheck = es_ES

% Cada capítulo de la memoria de TFG comienza con \chapter{TÍTULO DEL CAPÍTULO}, tal y como requiere la normativa de la EPSJ
\chapter{INTRODUCCIÓN}  

El presente Proyecto de Fin de Grado aborda una de las problemáticas más acuciantes en el ámbito de la ciberseguridad: la detección proactiva de intrusiones en redes informáticas. En un panorama digital en constante evolución, donde las amenazas son cada vez más sofisticadas y persistentes, la capacidad de identificar y neutralizar actividades maliciosas en tiempo real se ha convertido en un pilar fundamental para la protección de la información y la infraestructura crítica. Tradicionalmente, los Sistemas de Detección de Intrusiones (IDS) se basaban en firmas o reglas predefinidas, un enfoque que, si bien eficaz contra amenazas conocidas, presenta limitaciones intrínsecas ante nuevos ataques o variaciones de los existentes.

Este trabajo propone el desarrollo de un Sistema de Detección de Intrusiones (IDS) innovador, que fusiona la captura y el monitoreo de tráfico de red en tiempo real con la potencia de las técnicas de aprendizaje automático (\textit{Machine Learning}). El objetivo principal es la creación de un modelo de clasificación robusto y preciso, capaz de discernir de forma autónoma si un determinado flujo de tráfico de red es benigno (legítimo) o malicioso (indicativo de una intrusión o ataque). Esta aproximación busca superar las deficiencias de los sistemas basados en firmas, ofreciendo una mayor adaptabilidad y capacidad para detectar anomalías y comportamientos nunca antes vistos.

Para lograr este cometido, se hará uso de diversas tecnologías y metodologías de vanguardia. La fase de captura y preprocesamiento de datos de red se implementará mediante una aplicación diseñada específicamente para tal fin, garantizando la obtención de información relevante y en el formato adecuado para su análisis. Posteriormente, esta información será alimentada a un modelo de \textit{Machine Learning}~\cite{bishop2007prml}, concretamente un algoritmo de \textit{Random Forest} (Bosques Aleatorios), seleccionado por su probada eficacia, robustez y capacidad para manejar grandes volúmenes de datos con alta dimensionalidad, características esenciales en el análisis de tráfico de red.

La estructura de este primer capítulo sentará las bases conceptuales y técnicas de todo el proyecto. A través de una descripción detallada de los componentes clave y la visión general del sistema propuesto, se facilitará la comprensión de los antecedentes del trabajo, los cuales serán explorados en profundidad en el segundo capítulo de esta memoria. Este enfoque metodológico permite establecer un marco claro para la posterior justificación, diseño e implementación de la solución planteada, enfatizando la relevancia y el impacto potencial de un IDS basado en aprendizaje automático en el contexto actual de la ciberseguridad.

\section{Fundamentos}

En este capítulo se establecerán las bases teóricas y conceptuales fundamentales que sustentan el análisis, diseño e implementación del sistema de detección de intrusiones propuesto. Se abordará en profundidad el paradigma de los Sistemas de Detección de Intrusiones, las bases del aprendizaje automático como disciplina, y se hará un énfasis particular en la comprensión del algoritmo de \textit{Random Forest}, pieza clave de la solución desarrollada. Finalmente, se detallarán las métricas esenciales para la evaluación rigurosa del rendimiento de modelos de clasificación en el contexto de la ciberseguridad. Se comenzará explicando el concepto de \textbf{ciberseguridad y amenazas en redes}~\cite{iso27000}.

\subsection{Ciberseguridad y amenazas en redes}

La seguridad de la información constituye un pilar fundamental en la sociedad digital contemporánea, donde la interconexión global y la dependencia de los sistemas informáticos son crecientes. La protección de los activos digitales se articula en torno a los principios de confidencialidad, integridad y disponibilidad (CID)~\cite{iso27000,anderson2020security}. La confidencialidad garantiza que la información sea accesible únicamente por entidades autorizadas; la integridad asegura que la información no ha sido alterada de forma no autorizada; y la disponibilidad se refiere a la capacidad de los usuarios autorizados para acceder a la información y los sistemas cuando sea necesario. Cualquier evento que comprometa uno o más de estos principios se clasifica como una amenaza, la cual, al explotar una vulnerabilidad existente, puede materializarse en un riesgo para la organización.

El panorama de amenazas en redes informáticas evoluciona constantemente, presentando un desafío dinámico para la ciberseguridad. Si bien históricamente los ataques se centraban en la explotación de vulnerabilidades conocidas o errores de configuración, la sofisticación actual se manifiesta en técnicas de evasión más complejas y la emergencia de amenazas persistentes avanzadas (APT). Entre los tipos de ataques~\cite{enisaETL2023,anderson2020security} más prevalentes que un sistema de detección de intrusiones debe ser capaz de identificar se encuentran:
\begin{itemize}

    \item\textbf{Ataques de Denegación de Servicio (DoS/DDoS)}: Dirigidos a agotar los recursos de un sistema o red, impidiendo el acceso a servicios legítimos.
    
    \item\textbf{Escaneo de Puertos y Reconocimiento}: Fases preliminares donde un atacante explora una red en busca de puntos débiles o servicios abiertos.
    
    \item\textbf{Ataques de Fuerza Bruta}: Intentos sistemáticos y repetitivos para adivinar credenciales de acceso o claves de cifrado.
    
    \item\textbf{\textit{Malware} (\textit{Software} Malicioso)}: Incluye virus, troyanos, \textit{ransomware} y \textit{spyware}, diseñados con propósitos destructivos, de espionaje o de control remoto.
    
    \item\textbf{Explotación de Vulnerabilidades}: Aprovechamiento de fallos de diseño o implementación en \textit{software} y \textit{hardware} para obtener acceso no autorizado o ejecutar código malicioso.

\end{itemize}

La capacidad de identificar estos y otros comportamientos anómalos es crucial para mantener la postura de seguridad de una infraestructura de red.
    
\subsection{Sistema de detección de intrusiones}\label{Sec.Capitulos}

En respuesta a la creciente complejidad del panorama de amenazas, los Sistemas de Detección de Intrusiones (IDS) han emergido como componentes esenciales de una estrategia de defensa en profundidad. Un IDS puede definirse como una aplicación de seguridad que monitoriza el tráfico de red o la actividad de un sistema con el fin de identificar patrones o comportamientos indicativos de una intrusión o una violación de políticas de seguridad. Su función principal no es bloquear el ataque (labor de un IPS o \textit{firewall}), sino generar alertas que permitan a los administradores de seguridad tomar las medidas correctivas oportunas.

La clasificación de los IDS se realiza habitualmente atendiendo a su método de detección y a su ubicación en la infraestructura:
\begin{itemize}
    
    \item\textbf{IDS Basados en Firmas (\textit{Signature-based IDS - SIDS})}: Estos sistemas operan mediante la comparación del tráfico de red o eventos del sistema con una base de datos de firmas predefinidas, las cuales corresponden a patrones conocidos de ataques. Su principal ventaja reside en una alta precisión en la detección de ataques ya identificados y catalogados, con una baja tasa de falsos positivos en esos escenarios. No obstante, su limitación inherente radica en la incapacidad para detectar ataques novedosos o variantes polimórficas para las cuales no existe una firma en su base de datos, lo que exige una constante actualización de las mismas.
    
    \item\textbf{IDS Basados en Anomalías (\textit{Anomaly-based IDS - AVIDS})}: A diferencia de los SIDS, los AVIDS no dependen de firmas de ataques conocidos. En su lugar, construyen un perfil de comportamiento "normal" de la red, los usuarios o las aplicaciones. Cualquier desviación significativa de este perfil es considerada una anomalía y, por ende, una posible intrusión. La principal fortaleza de los AVIDS es su capacidad para detectar ataques \textit{"zero-day"}  (desconocidos previamente) y ataques sutiles que no encajan en patrones preestablecidos. Sin embargo, su principal desafío es la mayor propensión a generar falsos positivos, ya que cualquier cambio en el comportamiento normal (como la introducción de un nuevo servicio o una carga de tráfico inusual pero legítima) puede ser erróneamente clasificado como un ataque, requiriendo un ajuste continuo y afinado.
    
\end{itemize}

En cuanto a su ubicación, se distinguen los \textit{Network-based IDS} (NIDS), que analizan el tráfico de red en puntos estratégicos de la infraestructura sin depender de los \textit{hosts} individuales, y los \textit{Host-based IDS} (HIDS), que monitorizan la actividad interna de un sistema operativo o aplicación específica. El sistema desarrollado en este proyecto se enmarca dentro de la categoría de NIDS, enfocándose en el análisis del tráfico de red.

\subsection{Clasificación}

La clasificación es una de las principales tareas dentro de la minería de datos.
Como se expresó previamente, el objetivo de esta tarea es predecir una etiqueta categórica para cada ejemplo de un conjunto de datos a partir de sus atributos conocidos \cite{hastie2009elements}.

Existen diversos tipos de clasificación. Entre los más destacados, podemos encontrar:

\begin{itemize}
    \item\textbf{Clasificación binaria}: en este problema concreto de clasificación, es necesario determinar a qué clase pertenece cada ejemplo de los datos, pero únicamente hay dos clases a elegir. Se suelen usar para problemas de carácter binario, como si una persona está sana o enferma o, en el contexto de esta memoria, si una conexión determinada es benigna o un ataque.
    
    \item\textbf{Clasificación multiclase}: al contrario que en el caso previo, se debe determinar la clase de cada ejemplo de entre un conjunto de clases de tamaño superior a dos. En este TFG, también se hará uso de la clasificación multiclase para tratar de determinar el tipo de ataque de una conexión que se sabe maliciosa.
    
    \item\textbf{Clasificación multietiqueta}: un caso particular dentro de la clasificación multiclase es la clasificación multietiqueta, ya que, en lugar de asignar una sola clase a cada ejemplo del conjunto de datos, se le deben asignar un conjunto de clases en función a las características de los datos, a menudo siendo posible que cada elemento tenga un número de clases asociadas distinto.

\end{itemize}

\subsection{Descubrimiento de conocimiento en bases de datos}

El Descubrimiento de Conocimiento en Bases de Datos (KDD) es un proceso sistemático y iterativo, no trivial, diseñado para la extracción de patrones válidos, novedosos, potencialmente útiles y comprensibles a partir de grandes volúmenes de datos. Representa la filosofía subyacente a la transformación de datos crudos en inteligencia accionable, siendo crucial en dominios donde la toma de decisiones basada en evidencia es primordial, como la ciberseguridad. 

Al principio, el mayor problema a la hora de extraer información de los datos era, precisamente, conseguir una cantidad de datos considerable para obtener la información buscada. 

Actualmente, el problema ya no radica en obtener datos de los que extraer información, sino en el conjunto de procesos a los que hay que someter dichos datos para poder lograrlo. Por esto precisamente surge el concepto del descubrimiento de conocimiento en bases de datos, o KDD por sus siglas en inglés, propuesto originalmente por Brachman y Anand~\cite{brachman1994process} y expandido por Fayyad et al. ~\cite{fayyad1996kdd}.

En el contexto de un Sistema de Detección de Intrusiones (IDS) basado en aprendizaje automático, el KDD proporciona el marco metodológico para derivar el ``conocimiento" que permite al sistema identificar comportamientos anómalos o maliciosos en el tráfico de red.

El proceso de KDD se articula a través de una secuencia de etapas interdependientes:

\textbf{Fases del Proceso KDD}
\begin{enumerate}

    \item\textbf{Selección}: Esta etapa inicial define y adquiere los datos relevantes del dominio de aplicación. Para un IDS, esto implica la selección de \textit{datasets} de tráfico de red, como el CIC-IDS2018, que contengan una representación adecuada de comportamientos tanto benignos como maliciosos. La exhaustividad y representatividad de esta selección son fundamentales para la generalizabilidad del modelo resultante.
    
    \item\textbf{Preprocesamiento}: Considerada a menudo la fase más laboriosa y crítica, el preprocesamiento aborda la preparación de los datos brutos. Los datos de red, por su naturaleza, suelen ser ruidosos, incompletos, o inconsistentes. Esta etapa involucra:
    \begin{itemize}

        \item\textbf{Limpieza de Datos}: Eliminación o corrección de datos erróneos, duplicados, inconsistencias y manejo de valores ausentes.
        
        \item\textbf{Normalización/Estandarización}: Ajuste de las escalas de las características numéricas para evitar que aquellas con rangos de valores más amplios dominen en el análisis.
        
        \item\textbf{Transformación}: Conversión de datos a formatos adecuados para la minería. En el caso de tráfico de red, esto incluye la agregación de paquetes individuales en flujos de red y la extracción de características de esos flujos (ej., duración, número de paquetes, banderas TCP), elementos esenciales para el aprendizaje automático.
          
    \end{itemize}
    \item\textbf{Transformación}: Aunque intrínsecamente ligada al preprocesamiento, esta fase se centra en refinar la representación de los datos para la minería. La ingeniería de características es su componente clave, donde el conocimiento experto del dominio de red se aplica para derivar atributos más complejos y discriminatorios (ej., ratios de bytes por paquete, entropía de los puertos destino). El objetivo es crear un conjunto de características que optimicen la capacidad del algoritmo de \textit{Machine Learning} para identificar patrones de intrusión.
    
    \item\textbf{Minería de Datos (\textit{Data Mining})}: Este es el corazón del proceso KDD, donde se aplican algoritmos computacionales para descubrir patrones ocultos, asociaciones, cambios significativos, desviaciones o estructuras significativas en los datos. No es simplemente una técnica, sino un conjunto de enfoques, tareas y técnicas aplicadas a los datos ya preprocesados:
    \begin{itemize}

    \item\textbf{Enfoques de Minería de Datos}:
        \begin{itemize}
        
            \item\textbf{Aprendizaje Supervisado}: Como se detalla en el punto de Transformación, implica el uso de datos etiquetados para predecir un resultado. La detección de intrusiones es, en esencia, un problema de clasificación supervisada, donde se predice si un flujo es benigno o malicioso.
            
            \item\textbf{Aprendizaje No Supervisado}: Se utiliza para encontrar estructuras o patrones en datos sin etiquetas previas (ej., \textit{clustering} para agrupar tráficos similares, o detección de anomalías para identificar comportamientos que se desvían de lo normal sin una etiqueta de "ataque" explícita). Aunque tu proyecto se centra en lo supervisado, estos enfoques complementarios son relevantes en KDD.
            
            \item\textbf{Aprendizaje Semi-supervisado}: Combina datos etiquetados y no etiquetados, útil en escenarios donde el etiquetado es costoso.
            
        \end{itemize}

    \item\textbf{Tareas Típicas de Minería de Datos}:
        \begin{itemize}
        
            \item\textbf{Clasificación}: Asignar elementos a categorías predefinidas (ej., "ataque" o "normal"). Esta es la tarea principal de tu IDS.
            
            \item\textbf{Regresión}: Predecir un valor numérico continuo.
            
            \item\textbf{Agrupación (\textit{Clustering})}: Dividir un conjunto de datos en grupos (\textit{clusters}) de elementos similares.
            
            \item\textbf{Asociación}: Descubrir reglas que describen relaciones entre elementos.
            
            \item\textbf{Detección de Anomalías/\textit{Outliers}}: Identificar patrones que no se ajustan a un comportamiento esperado.
              
        \end{itemize}
    \item\textbf{Técnicas Aplicadas en este Proyecto}: En este trabajo, la fase de minería de datos se concreta en la aplicación del algoritmo \textit{Random Forest} para la tarea de clasificación. Su capacidad para manejar un gran número de características y su robustez ante el ruido lo hacen idóneo para los complejos \textit{datasets} de tráfico de red.
    \end{itemize}
    
    \item\textbf{Evaluación y Presentación}: La etapa final valida la significancia y utilidad de los patrones descubiertos. Los modelos de clasificación se evalúan utilizando métricas específicas (precisión, sensibilidad, \textit{F1-score}, curva ROC) en un conjunto de datos de prueba independiente, asegurando que el conocimiento extraído sea generalizable y no un artefacto del entrenamiento. La interpretabilidad del modelo, aunque un desafío, es un objetivo deseable que permite a los analistas de seguridad comprender la razón de una detección. Finalmente, la presentación del conocimiento se realiza a través de informes, visualizaciones (como el \textit{dashboard} de tu aplicación), que comunican de manera efectiva los \textit{insights} a los usuarios finales.

\end{enumerate}
\subsection{Machine Learning}

El aprendizaje automático (\textit{Machine Learning - ML})~\cite{bishop2007prml}, una rama de la inteligencia artificial, ha transformado diversas disciplinas al dotar a los sistemas de la capacidad de aprender de datos y mejorar su rendimiento con la experiencia, sin ser programados explícitamente para cada tarea. En el contexto de la ciberseguridad, y específicamente en la detección de intrusiones, el ML ofrece un enfoque adaptativo que puede identificar patrones complejos en grandes volúmenes de tráfico de red, superando las limitaciones de los métodos tradicionales basados en reglas fijas.

El ciclo de vida de un modelo de \textit{Machine Learning} generalmente comprende las siguientes fases interdependientes:

\begin{itemize}
    
    \item\textbf{Recopilación y Preprocesamiento de Datos}: Etapa crucial que implica la obtención, limpieza, transformación y estandarización de los datos. La calidad de los datos de entrada es determinante para el éxito del modelo.
    
    \item\textbf{Ingeniería de Características (\textit{Feature Engineering})}: Proceso de seleccionar, crear o transformar variables (\textit{features}) a partir de los datos brutos que mejor representen la información subyacente y sean más relevantes para el problema de predicción.
    
    \item\textbf{Selección y Entrenamiento del Modelo}: Consiste en elegir el algoritmo de ML más adecuado y ajustar sus parámetros utilizando un subconjunto de datos (conjunto de entrenamiento) para que aprenda los patrones deseados.
    
    \item\textbf{Evaluación y Optimización del Modelo}: Se mide el rendimiento del modelo con un conjunto de datos no visto (conjunto de validación o prueba) y se realizan ajustes de hiperparámetros para mejorar su eficacia.
    
    \item\textbf{Despliegue y Monitorización}: Una vez validado, el modelo se integra en un entorno de producción para realizar predicciones en tiempo real, siendo fundamental un monitoreo continuo de su desempeño.

\end{itemize}

Dentro del ML, el aprendizaje supervisado es el paradigma central para la detección de intrusiones basada en clasificación. En este enfoque, el modelo aprende de un conjunto de datos que incluye tanto las características de entrada (ej., métricas del tráfico de red) como sus correspondientes etiquetas de salida (ej., "tráfico normal" o "ataque"). El objetivo es que el modelo infiera una función que mapee las entradas a las salidas, permitiendo clasificar nuevas instancias de datos sin etiquetas. Esta investigación se inscribe en este paradigma, buscando clasificar el tráfico de red como benigno o malicioso.

\subsubsection{Preprocesamiento de datos para modelos de Machine Learning}

La efectividad de cualquier modelo de aprendizaje automático depende críticamente de la calidad y el formato de los datos de entrada. En el contexto de la detección de intrusiones, los datos brutos de tráfico de red, capturados en formato de paquetes, no son directamente utilizables por los algoritmos de \textit{Machine Learning}. El preprocesamiento de datos es, por tanto, una fase indispensable que transforma estos paquetes en un formato estructurado y significativo para el análisis.

Las etapas fundamentales del preprocesamiento en un IDS basado en ML incluyen:
\begin{itemize}
   
    \item\textbf{Extracción y Reconstrucción de Flujos de Red}: Los algoritmos de ML operan sobre ``instancias" o "muestras", que en el contexto de la red, son típicamente flujos. Un flujo se define como una secuencia de paquetes que comparten una tupla común de cinco elementos: dirección IP de origen, puerto de origen, dirección IP de destino, puerto de destino y protocolo de transporte. La reconstrucción de flujos a partir de paquetes individuales permite contextualizar la información y calcular características que abarcan la duración total de la conexión.
    
     \item\textbf{Generación de Características (\textit{Feature Engineering})}: Una vez reconstruidos los flujos, se extraen o derivan diversas características numéricas y categóricas que describen el comportamiento del flujo. Estas características deben ser consistentes con las utilizadas durante la fase de entrenamiento del modelo. Ejemplos de características incluyen:
    \begin{itemize}
         \item\textbf{Métricas de Volumen}: Número total de paquetes, bytes transmitidos (bidireccional), duración del flujo.
        
         \item\textbf{Métricas de Tasa}: Paquetes por segundo, bytes por segundo.
        
         \item\textbf{Características del Protocolo}: Banderas TCP (SYN, ACK, FIN, RST, PSH, URG), tipo de protocolo (TCP, UDP, ICMP).
        
         \item\textbf{Información de Puertos}: Puertos de origen y destino.
        
         \item\textbf{Características Temporales}: Variabilidad en el tiempo entre paquetes (\textit{jitter}).
        
         \item\textbf{Entropía de la Carga Útil}: Medida de la aleatoriedad en los datos de la carga útil, que puede indicar cifrado o ciertos tipos de ataques.
    \end{itemize}
    
     \item\textbf{Manejo de Variables Categóricas}: Las características no numéricas, como los nombres de protocolos (TCP, UDP, ICMP), deben transformarse a un formato numérico. La técnica común de \textit{One-Hot Encoding} crea nuevas columnas binarias para cada categoría, donde un '1' indica la presencia de esa categoría y un '0' su ausencia.
    
     \item\textbf{Escalado de Características Numéricas}: Las características numéricas a menudo presentan rangos de valores muy diferentes (ej., la duración de un flujo puede ser de milisegundos a horas, mientras que el número de paquetes es un entero pequeño). El escalado (normalización o estandarización) es crucial para evitar que las características con rangos más amplios dominen desproporcionadamente en el proceso de aprendizaje del modelo. Técnicas como \textit{StandardScaler} (resta la media y divide por la desviación estándar) o \textit{MinMaxScaler} (escala los valores a un rango específico, como [0, 1]) son comúnmente empleadas.
    
     \item\textbf{Manejo de Valores Ausentes o Ruidosos}: Los datos de red pueden contener valores faltantes o erróneos. Es fundamental aplicar estrategias de imputación (ej., rellenar con la media, mediana o moda) o eliminación de instancias para asegurar la integridad del \textit{dataset}.
    
     \item\textbf{Manejo del Desequilibrio de Clases (\textit{Class Imbalance})}: En la detección de intrusiones, el tráfico legítimo es abrumadoramente más frecuente que el tráfico malicioso (clases desequilibradas). Si no se aborda, el modelo puede sesgarse hacia la clase mayoritaria y tener un rendimiento deficiente en la detección de la clase minoritaria (ataques). Técnicas como el \textit{oversampling} (ej., SMOTE para crear instancias sintéticas de la clase minoritaria), \textit{undersampling} (reducir la cantidad de instancias de la clase mayoritaria), o el uso de pesos de clase durante el entrenamiento del modelo, son esenciales para mitigar este problema.

\end{itemize}
\subsubsection{Evaluación de modelos en Sistemas de Detección de Intrusiones}

La evaluación rigurosa del rendimiento de un modelo de clasificación es un paso crítico en el desarrollo de un IDS basado en aprendizaje automático. Dadas las implicaciones de seguridad, es imperativo no solo medir la precisión global, sino también comprender cómo el modelo maneja los errores, especialmente los falsos negativos, que representan ataques no detectados.

La matriz de confusión es la herramienta fundamental para una evaluación detallada. Esta tabla resume las predicciones del modelo frente a las etiquetas reales del \textit{dataset} y se compone de cuatro cuadrantes clave en el contexto de clasificación binaria (Normal/Ataque):

\begin{itemize}

    \item\textbf{Verdaderos Positivos (TP)}: Instancias de tráfico malicioso correctamente clasificadas como ataques.
    
    \item\textbf{Verdaderos Negativos (TN)}: Instancias de tráfico normal correctamente clasificadas como benignas.
    
    \item\textbf{Falsos Positivos (FP)}: Instancias de tráfico normal erróneamente clasificadas como ataques (falsa alarma).
    
    \item\textbf{Falsos Negativos (FN)}: Instancias de tráfico malicioso erróneamente clasificadas como normales (ataque no detectado).

\end{itemize}

A partir de la matriz de confusión, se derivan métricas esenciales para evaluar el desempeño del IDS:

\begin{itemize}

    \item\textbf{Exactitud (\textit{Accuracy})}: 
    
        \begin{equation}
        \mathrm{Accuracy} = \frac{TP + TN}{TP + TN + FP + FN}
        \label{eq:accuracy}
        \end{equation}
    
     Mide la proporción de predicciones correctas sobre el total de predicciones. Aunque intuitiva, puede ser engañosa en \textit{datasets} desequilibrados, donde una alta precisión podría deberse simplemente a la correcta clasificación de la clase mayoritaria.
    
    \item\textbf{Sensibilidad / Exhaustividad (\textit{Recall} / \textit{True Positive Rate - TPR})}: 

          \begin{equation}
            \mathrm{Recall} = \frac{TP}{TP + FN}
            \label{eq:recall}
          \end{equation}
    

     Cuantifica la capacidad del modelo para identificar correctamente todas las instancias positivas (ataques reales). En un IDS, maximizar el \textit{recall} es a menudo una prioridad, ya que un alto número de falsos negativos implica ataques que pasan desapercibidos.
    
    \item\textbf{Especificidad (\textit{True Negative Rate - TNR})}:
    
      \begin{equation}
        \mathrm{TNR} = \frac{TN}{TN + FP}
        \label{eq:tnr}
      \end{equation}
        
     Indica la proporción de instancias negativas (tráfico normal) que son correctamente identificadas.
    
    \item\textbf{Precisión (\textit{Precision} / \textit{Positive Predictive Value - PPV})}: 

      \begin{equation}
        \mathrm{Precision} = \frac{TP}{TP + FP}
        \label{eq:precision}
      \end{equation}

     Responde a la pregunta: de todas las instancias que el modelo clasificó como ataques, ¿cuántas fueron realmente ataques? Una alta precisión minimiza los falsos positivos, lo cual es crucial para evitar la "fatiga de alerta'' en los operadores de seguridad.
    
    \item\textbf{\textit{F1-Score}}:

      \begin{equation}
        F1 = 2 \cdot \frac{\mathrm{Precision} \cdot \mathrm{Recall}}{\mathrm{Precision} + \mathrm{Recall}}
        \label{eq:f1}
      \end{equation}
    
     Es la media armónica de la precisión y la sensibilidad. Proporciona una métrica equilibrada que es especialmente útil cuando existe un desequilibrio significativo entre las clases, ya que penaliza los modelos con altos falsos positivos y falsos negativos.
    
    \item\textbf{Tasa de Falsos Positivos (\textit{False Positive Rate - FPR})}: 

      \begin{equation}
        \mathrm{FPR} = \frac{FP}{FP + TN}
        \label{eq:fpr}
      \end{equation}
    

     Complementa la especificidad y es una métrica crítica en IDS, ya que una alta tasa de falsas alarmas puede llevar a que los analistas ignoren alertas legítimas.

\end{itemize}
La Curva ROC (\textit{Receiver Operating Characteristic})~\cite{datacampAUC} y el Área bajo la Curva (AUC) son herramientas gráficas y métricas complementarias. La curva ROC representa la relación entre la Tasa de Verdaderos Positivos (TPR) y la Tasa de Falsos Positivos (FPR) a distintos umbrales de clasificación, permitiendo visualizar el compromiso entre ambas. El AUC cuantifica el rendimiento general del clasificador en todos los umbrales posibles; un valor de AUC cercano a 1.0 indica un excelente poder discriminatorio.

Para garantizar la robustez y generalizabilidad del modelo, se emplean técnicas de validación. La Validación Cruzada \textit{K-Fold} es un método estándar en el que el conjunto de datos se divide en k subconjuntos (\textit{folds}). El modelo se entrena k veces; en cada iteración, se utiliza un \textit{fold} diferente como conjunto de prueba y los k−1 restantes como conjunto de entrenamiento. Los resultados se promedian para obtener una estimación más fiable del rendimiento del modelo, reduciendo el sesgo y la varianza que podrían surgir de una única división aleatoria. Además, es esencial distinguir entre el conjunto de entrenamiento, el conjunto de validación (utilizado para el ajuste de hiperparámetros) y el conjunto de prueba (utilizado únicamente para la evaluación final imparcial del modelo).

\subsection{Ciencia de datos}\label{Sec.Referencias}

La Ciencia de Datos~\cite{iadbCienciaDatos,ibmCienciaDatos} emerge como una disciplina transversal que integra metodologías y principios de campos como la estadística, la informática, las matemáticas, el conocimiento del dominio y la visualización, con el propósito primordial de extraer conocimiento, patrones e \textit{insights} accionables a partir de volúmenes de datos complejos y heterogéneos. Su objetivo no es meramente el procesamiento de información, sino la capacidad de transformar datos brutos en inteligencia estratégica, facilitando la toma de decisiones informadas y la predicción de eventos futuros. En la actualidad, su aplicación es ubicua, abarcando desde la optimización de procesos empresariales hasta la investigación científica y, pertinentemente para este proyecto, la ciberseguridad.

Dentro del vasto espectro de la Ciencia de Datos, el aprendizaje automático (\textit{Machine Learning - ML})~\cite{bishop2007prml} y el aprendizaje profundo (\textit{Deep Learning - DL}) constituyen dos de sus pilares más potentes y de mayor crecimiento. Mientras que el aprendizaje automático engloba un conjunto de algoritmos que permiten a los sistemas aprender de los datos para realizar predicciones o tomar decisiones sin ser programados explícitamente para cada tarea, el aprendizaje profundo representa una subcategoría del ML que se basa en redes neuronales artificiales con múltiples capas, capaces de aprender representaciones de datos jerárquicas y abstractas.

En el marco de este proyecto de fin de grado, la Ciencia de Datos se erige como el eje metodológico para abordar el problema de la detección de intrusiones en redes. La fase inicial implica la adquisición y preprocesamiento de grandes volúmenes de datos de tráfico de red, los cuales se presentan con una multitud de características y métricas de conexión. Este proceso es crucial para transformar el tráfico en bruto en un formato estructurado y apto para el análisis computacional.

Posteriormente, y de manera exclusiva, se recurrirá a las técnicas de \textit{Machine Learning} para desarrollar la capacidad predictiva del sistema. Específicamente, se ha optado por el entrenamiento de un modelo basado en el algoritmo \textit{Random Forest}. Este modelo, conocido por su robustez, eficiencia y capacidad para manejar conjuntos de datos de alta dimensionalidad y características complejas, será el encargado de resolver un problema de clasificación. Su función cardinal será la de discernir, con alta fiabilidad, entre patrones de tráfico de red que corresponden a un comportamiento normal (benigno) y aquellos que denotan una actividad maliciosa o una posible intrusión. La implementación de esta aproximación permitirá que el sistema de detección evolucione y se adapte a nuevas amenazas basándose en el aprendizaje continuo de los patrones inherentes a los datos de la red.

\section{Descripción del proyecto}
El propósito principal de este proyecto es la implementación de un capturador de paquetes en tiempo real que sea capaz de capturar dichos paquetes y agruparlos en distintos flujos para que posteriormente, se muestren en una aplicación web basada en un sistema de detección de intrusiones aplicándole mecanismos de \textit{machine learning}.

El proceso comienza con la captura del tráfico de red mediante una aplicación que tiene como fin la captura de paquetes en claro y a partir de estos, organiza los paquetes en distintos flujos y extrae las características necesarias para que pasen a un formato parecido al que se rige CICFlowMeter, que es una herramienta de generación y análisis de flujos de tráfico de red que produce flujos bidireccionales a partir de paquetes de red.

Por último, la aplicación mostrará en un \textit{dashboard}, los flujos de paquetes en un panel, y en otro, aparecerán cuáles de estos se consideran una amenaza. La lógica de control para discernir entre un flujo benigno y maligno, se encargará el modelo de IA que hemos elegido, el cual, estará integrado en el \textit{backend} al igual que el propio capturador en tiempo real que hemos implementado.

El presente Proyecto de Fin de Grado aborda el diseño e implementación de un Sistema de Detección de Intrusiones (IDS) innovador y multifuncional que integra la captura de tráfico de red en tiempo real con técnicas avanzadas de aprendizaje automático (\textit{Machine Learning}). El objetivo primordial es desarrollar una solución robusta capaz de discernir, con alta fiabilidad, entre tráfico de red legítimo (benigno) y actividad maliciosa o anómala.

El objetivo principal de este trabajo es la creación de una herramienta especializada para la captura y el procesamiento de paquetes de red. Esta aplicación ha sido diseñada con la capacidad de operar en tiempo real, monitorizando continuamente el flujo de datos a través de la red. Un aspecto distintivo de esta herramienta es su versatilidad para generar y persistir flujos de paquetes, transformando la información de bajo nivel de los paquetes individuales en un formato estructurado y significativo para el análisis posterior. La aplicación permite la conversión de estos flujos en formatos de salida estandarizados como .csv o .txt, lo que facilita la creación de nuevas bases de datos de tráfico de red. Esta funcionalidad es crucial, ya que emula y complementa la metodología empleada en \textit{datasets} de referencia en ciberseguridad, como los de la serie CIC-IDS (ej., CIC-IDS2017, CIC-IDS2018, CIC-IDS2019), proporcionando una base para la investigación y el desarrollo futuros en el campo. Además, la capacidad de invocación por consola dota a la herramienta de flexibilidad para ser integrada en diversos entornos y automatizaciones, permitiendo la generación de datos para el entrenamiento y la evaluación de modelos de \textit{Machine Learning}.

Para validar y demostrar la eficacia del capturador de paquetes y para completar la arquitectura del IDS, se ha implementado un módulo de detección basado en \textit{Machine Learning}. Este módulo utiliza un modelo de clasificación \textit{Random Forest}, seleccionado por su reconocida capacidad para manejar grandes volúmenes de datos con alta dimensionalidad y su robustez ante el ruido. El modelo ha sido entrenado para identificar patrones y características específicas que distinguen el tráfico normal de diversas categorías de ataques, lo que le permite clasificar de forma predictiva cada flujo de red entrante.

Complementando estas funcionalidades, se ha desarrollado una aplicación web, concebida como un \textit{dashboard} intuitivo, que permite visualizar en tiempo real la actividad del capturador de paquetes y el rendimiento del modelo de detección. Esta interfaz gráfica no solo ofrece una representación clara de los eventos de red, sino que también sirve como una herramienta práctica para monitorizar las detecciones de tráfico malicioso, facilitando la interacción del usuario con el IDS.

\section{Objetivos y motivación}\label{sec:objetivos}

La proliferación incesante de las tecnologías digitales y la interdependencia sistémica de las infraestructuras de red han configurado un panorama donde la ciberseguridad ya no es una mera consideración técnica, sino un pilar fundamental para la operatividad y la resiliencia de cualquier entidad, desde organizaciones gubernamentales hasta empresas privadas y usuarios individuales. En este entorno, la capacidad de detectar y responder eficazmente a las intrusiones cibernéticas se ha vuelto crítica. Las amenazas evolucionan con una celeridad asombrosa, adoptando formas cada vez más sofisticadas y evasivas que superan las capacidades de los sistemas de defensa convencionales, como los Sistemas de Detección de Intrusiones (IDS) basados exclusivamente en firmas. Estos últimos, si bien eficientes contra amenazas conocidas y previamente catalogadas, son inherentemente limitados frente a los ataques \textit{zero-day} o las variantes polimórficas que modifican sus patrones para eludir la detección.

Esta brecha en las capacidades de detección tradicional ha impulsado la necesidad de explorar paradigmas más adaptativos y predictivos. El aprendizaje automático (\textit{Machine Learning} — ML) emerge como una solución prometedora, dada su capacidad para identificar patrones complejos, anomalías y comportamientos desviados en grandes volúmenes de datos, sin requerir una programación explícita para cada posible amenaza \cite{Zhang2022AICybersecurity}. La aplicación de ML en la detección de intrusiones no solo promete una mayor tasa de detección de ataques novedosos, sino que también ofrece la posibilidad de reducir la dependencia de actualizaciones manuales de firmas y reglas.

La motivación principal de este Proyecto de Fin de Grado radica precisamente en responder a esta necesidad crítica. Se aspira a contribuir de forma tangible al campo de la ciberseguridad mediante el desarrollo de un IDS que fusione la monitorización en tiempo real del tráfico de red con la inteligencia predictiva del \textit{Machine Learning}. Más allá de la mera detección, se reconoce un desafío recurrente en la investigación de IDS: la escasez de \textit{datasets} de tráfico de red etiquetados que sean representativos y actuales para el entrenamiento y la validación de modelos de ML \cite{PolaniaArias2021EvaluacionMLIDS}. La creación de tales \textit{datasets} es laboriosa y costosa, lo que limita la evaluación comparativa y el desarrollo de nuevas arquitecturas de detección. Esta dificultad intrínseca proporciona una motivación adicional y un objetivo crucial: disponer de una herramienta que no solo detecte intrusiones, sino que también \textbf{genere \textit{datasets} de alta calidad} emulando metodologías de \textit{benchmarks} reconocidos como los de la serie CIC-IDS (p.\,ej., CIC-IDS2017, CIC-IDS2018, CIC-DDoS2019). Esta capacidad es fundamental para impulsar futuras investigaciones y para el entrenamiento continuo de modelos más robustos y adaptativos.

Con base en esta motivación y en las limitaciones identificadas, los objetivos del proyecto se estructuran como sigue.

\subsection*{Objetivo general}
Este TFG tiene como objetivo principal \textbf{Diseñar, desarrollar e implementar una herramienta de captura en tiempo real} que reconstruya flujos de conexión, extraiga características y permita la \textbf{generación de \textit{datasets} exportables} (\emph{CSV/TXT}) compatibles con esquemas tipo CIC; y, como demostrador, un \textbf{prototipo de IDS} holístico y funcional que consuma dichos datos para clasificar tráfico benigno/malicioso y tipificar ataques mediante un modelo de ML operando en tiempo real.

\subsection*{Objetivos específicos}
\begin{itemize}

    \item\textbf{Concebir y Construir una Herramienta Avanzada para la Captura y el Preprocesamiento de Tráfico de Red}: El desarrollo de una aplicación personalizada es fundamental. Esta herramienta no solo deberá ser eficiente en la captura de paquetes en tiempo real, sino que también integrará capacidades robustas para la reconstrucción de flujos de conexión y la extracción detallada de un amplio conjunto de características pertinentes para el análisis de seguridad. Un valor añadido de esta herramienta será su funcionalidad para generar salidas estructuradas en formatos estándar (\emph{CSV/TXT}), lo que permitirá la creación sistemática de nuevas bases de datos de tráfico de red. Esta capacidad de generación de \textit{datasets}, invocable desde la consola, es vital para la replicabilidad, la extensibilidad y la contribución a la comunidad de investigación en ciberseguridad.
    
    \item\textbf{Investigar, Implementar y Optimizar un Modelo de Clasificación de \textit{Machine Learning} de Alto Rendimiento}: La selección de un algoritmo de aprendizaje automático es crucial. Se ha optado por el algoritmo \textit{Random Forest} debido a su probada eficacia en problemas de clasificación multiclase y binaria, su resistencia al sobreajuste, y su capacidad para gestionar conjuntos de datos con un elevado número de características y una significativa dimensionalidad \cite{Breiman2001RandomForests}. El modelo será sometido a un riguroso proceso de entrenamiento con \textit{datasets} de tráfico etiquetado, con el fin de optimizar su rendimiento para la minimización simultánea de falsos positivos (que generan fatiga en los analistas) y, lo que es más crítico, de falsos negativos (que implican ataques no detectados) \cite{PolaniaArias2021EvaluacionMLIDS}.
    
    \item\textbf{Asegurar la Integración Fluida entre la Captura de Datos y el Motor de Detección de \textit{Machine Learning}}: Se establecerá un \textit{pipeline} de procesamiento de datos que permita la alimentación continua y eficiente de los flujos de red preprocesados desde la herramienta de captura hacia el modelo de \textit{Machine Learning}. Esta integración garantizará que el IDS pueda clasificar el tráfico de forma dinámica y emitir alertas inmediatas ante la detección de actividades sospechosas, operando en un modo casi en tiempo real.
    
    \item\textbf{Diseñar y Desarrollar una Interfaz Web Intuitiva para la Monitorización y Visualización}: Para facilitar la interacción del usuario y la supervisión del sistema, se implementará un prototipo de aplicación web que actúe como un \textit{dashboard}. Esta interfaz ofrecerá una representación clara y amigable de la actividad de la red, los eventos de detección generados por el modelo de \textit{Machine Learning}, y las métricas operativas del IDS. La visualización en tiempo real es clave para proporcionar a los administradores de seguridad una panorámica inmediata del estado de la red.
    
    \item\textbf{Realizar una Evaluación Exhaustiva y Rigurosa del Rendimiento del IDS Propuesto}: La validación empírica del sistema es indispensable. Se llevarán a cabo pruebas exhaustivas utilizando metodologías de evaluación estándar para problemas de clasificación en el ámbito de la ciberseguridad. Esto incluirá el cálculo y análisis de métricas como la precisión (\textit{accuracy}), sensibilidad (\textit{recall}), precisión (\textit{precision}), \textit{F1-score}, la construcción y análisis de la matriz de confusión, y la curva ROC con su respectivo AUC \cite{PolaniaArias2021EvaluacionMLIDS}. El objetivo es demostrar la eficacia del IDS en la identificación de diferentes tipos de ataques y validar su viabilidad como una herramienta de seguridad operativa y de valor añadido. Para más información, consulte este apartado ~\ref{ch:resultados}.
 
\end{itemize}

A través de la consecución de estos objetivos, este proyecto no solo aspira a generar una solución técnica funcional para la detección de intrusiones, sino también a contribuir al cuerpo de conocimiento en la intersección de la ciberseguridad y la inteligencia artificial, ofreciendo herramientas y metodologías que pueden ser de utilidad para la comunidad científica y profesional.

\section{Metodología de desarrollo de software}
Una metodología de desarrollo de \textit{software} es un conjunto de prácticas,
técnicas y procedimientos que se utilizan para organizar, planificar y ejecutar proyectos de desarrollo de \textit{software}. Su objetivo principal es mejorar la eficiencia y la calidad del proceso de desarrollo, asegurando que el \textit{software} se entregue a tiempo, dentro del presupuesto y cumpla con los requisitos del cliente. Estas metodologías proporcionan un marco estructurado que guía a los equipos de desarrollo a través de las diferentes fases del ciclo de vida del \textit{software}, desde la concepción y el diseño hasta la implementación, prueba y mantenimiento.

\subsection{SCRUM como modelo de desarrollo}
Dada la estructura del proyecto y sus dimensiones, se ha optado por utilizar la metodología \textit{Scrum}. \textit{Scrum} es una metodología ágil que se basa en la realización de incrementos pequeños y manejables, permitiendo así una mayor flexibilidad y adaptación a los cambios a lo largo del proceso de desarrollo.

Algunos de los componentes clave de \textit{Scrum} son:
\begin{itemize}

    \item\textbf{Roles: \textit{Scrum} define tres roles principales}:
    \begin{itemize}
 
        \item\textbf{\textit{Product Owner}}: Responsable de maximizar el valor del producto y gestionar el \textit{backlog} del producto.
        \item\textbf{\textit{Scrum Master}}: Facilita el proceso \textit{Scrum}, ayuda al equipo a seguir las prácticas de \textit{Scrum} y elimina impedimentos.
        \item\textbf{Equipo de Desarrollo}: Grupo multifuncional que trabaja en la entrega de incrementos del producto.
          
    \end{itemize}
    \item\textbf{Eventos}:
    \begin{itemize}

        \item\textbf{\textit{Sprint}}: Periodo de tiempo fijo (generalmente de 1 a 4 semanas) durante el cual se realiza un incremento del producto.
        \item\textbf{\textit{Sprint Planning}}: Reunión para planificar el trabajo que se realizará durante el \textit{sprint}.
        \item\textbf{\textit{Daily Scrum}}: Reunión diaria de 15 minutos para sincronizar el trabajo del equipo.
        \item\textbf{\textit{Sprint Review}}: Reunión para revisar el incremento y adaptar el \textit{backlog} del producto si es necesario.
        \item\textbf{\textit{Sprint Retrospective}}: Reunión para reflexionar sobre el \textit{sprint} y buscar mejoras continuas.
           
    \end{itemize}
    \item\textbf{Artefactos}:
    \begin{itemize}

        \item\textbf{\textit{Product Backlog}}: Lista priorizada de todo el trabajo que se necesita en el producto.
        \item\textbf{\textit{Sprint Backlog}}: Lista de tareas seleccionadas del \textit{product backlog} para completarse en el \textit{sprint} actual.
        \item\textbf{\textit{Increment}}: El resultado de un \textit{sprint}, que debe ser un producto utilizable y potencialmente desplegable.
          
    \end{itemize}
   
\end{itemize}

\textit{Scrum} se caracteriza por su enfoque en la transparencia, inspección y
adaptación, permitiendo a los equipos responder rápidamente a los cambios y mejorar continuamente.

Véase en la siguiente ilustración ~\ref{Fig.FasesSCRUM} el proceso de las fases de desarrollo de la metodología Scrum.

\begin{figure}[ht!]
  \centering
  \includegraphics[width=0.8\textwidth]{imagenes/proceso_scrum.png}
  \caption{Proceso de las fases de desarrollo Scrum.}
  \label{Fig.FasesSCRUM}
\end{figure}

\subsection{Adaptación de Scrum para proyectos individuales}
Aunque \textit{Scrum} está diseñado originalmente para equipos, sus principios y
estructura pueden adaptarse para el desarrollo por una sola persona. Basándonos en el artículo “\textit{Scrum for One}” de Lucidchart ~\cite{lucidchartScrumForOne} vamos a desarrollar el proceso de adaptación de esta metodología para una sola persona. 

En un entorno de un solo desarrollador, la misma persona asume los roles de \textit{Product Owner}, \textit{Scrum Master} y Equipo de Desarrollo. Esto requiere una clara organización y autogestión. El desarrollador único debe gestionar el \textit{backlog} del producto, facilitar su propio proceso de desarrollo y eliminar impedimentos de manera autónoma. Este enfoque puede ser beneficioso ya que el desarrollador tiene control total sobre las decisiones y priorizaciones, lo que puede agilizar el proceso. 

Se pueden mantener los eventos clave de \textit{Scrum}, pero de manera simplificada y flexible para una sola persona:
\begin{itemize}
    
    \item\textbf{\textit{Sprint Planning}}: Planificar los \textit{sprints} sigue siendo esencial. El desarrollador único debe dedicar tiempo a establecer objetivos claros para cada \textit{sprint}, definir las tareas necesarias y priorizarlas en el \textit{sprint backlog}.
    
    \item\textbf{\textit{Daily Scrum}}: Aunque no hay un equipo con el cual sincronizarse, el desarrollador puede realizar una auto-revisión diaria. Esto ayuda a mantener el enfoque y la disciplina, permitiendo reflexionar sobre el progreso y ajustar el plan si es necesario.
    
    \item\textbf{\textit{Sprint Review}}: Al final de cada \textit{sprint}, el desarrollador revisa el trabajo completado. Este evento puede incluir la evaluación de cómo se han alcanzado los objetivos del \textit{sprint} y la identificación de cualquier ajuste necesario en el \textit{backlog} del producto.

    \item\textbf{\textit{Sprint Retrospective}}: Es crucial para la mejora continua. El desarrollador reflexiona sobre lo que funcionó bien y lo que se puede mejorar para los próximos \textit{sprints}.
  
\end{itemize}

Los artefactos de \textit{Scrum} también siguen siendo importantes ya que una vez
vistas las modificaciones en los eventos se puede concluir que los artefactos se usarán de la misma forma que en un equipo con varias personas. 

Además, la naturaleza iterativa de \textit{Scrum} permite una gran flexibilidad y
capacidad de adaptación. Esto significa que el desarrollador puede responder rápidamente a los cambios y ajustar su enfoque según sea necesario, lo cual es vital en un entorno de desarrollo dinámico. Las retrospectivas regulares fomentan una cultura de mejora continua, permitiendo identificar áreas de mejora y aplicar cambios en ciclos cortos, lo que facilita la evolución y optimización del proceso de desarrollo.

\subsection{Sprints de desarrollo}\label{Sec.Sprints}
El desarrollo de un sistema tan complejo como un Sistema de Detección de Intrusiones (IDS) que integra captura de tráfico en tiempo real, procesamiento de datos, análisis mediante aprendizaje automático y una interfaz web, exige una metodología ágil que permita gestionar la complejidad, adaptarse a los desafíos emergentes y entregar valor de forma incremental. Para este proyecto, se adoptó un enfoque basado en \textit{Sprints} de Desarrollo, inspirados en la metodología SCRUM. Esta aproximación facilitó la división del proyecto en iteraciones cortas y manejables, cada una enfocada en la consecución de funcionalidades específicas, garantizando así un progreso continuo y la capacidad de integrar retroalimentación a lo largo del ciclo de vida del desarrollo. A continuación, se describen los \textit{sprints} planificados y ejecutados, delineando los objetivos y entregables clave de cada fase.


\textbf{\textit{Sprint} 1: Fundamentos Teóricos y Análisis de Requisitos}

Este \textit{sprint} inaugural se centró en la construcción de una base de conocimiento sólida indispensable para abordar un proyecto de la envergadura de un IDS basado en \textit{Machine Learning}. Las primeras semanas se dedicaron a una inmersión profunda en la literatura científica y técnica. Se revisó una colección de \textit{papers} y artículos especializados proporcionados por la dirección del proyecto, enfocados en la intersección de la inteligencia artificial y la ciberseguridad, con especial énfasis en los principios y arquitecturas de los Sistemas de Detección de Intrusiones \cite{Charmet2022XAI}, \cite{Zhang2022AICybersecurity}. Paralelamente, se llevó a cabo una investigación exhaustiva en bases de datos académicas y repositorios en línea para complementar el entendimiento de conceptos fundamentales de IDS y los distintos paradigmas de inteligencia artificial, inicialmente explorando tanto el aprendizaje supervisado como el no supervisado. Una vez afianzado el conocimiento sobre las técnicas de aprendizaje automático, la atención se dirigió hacia la identificación y análisis de conjuntos de datos (\textit{datasets}) adecuados para el entrenamiento y evaluación de modelos de IDS. Se familiarizó con la estructura, características y taxonomía de \textit{datasets} de referencia en la comunidad, tales como CIC-IDS (2017, 2018, 2019), KDD99, NSL-KDD, y UNSW \cite{PolaniaArias2021EvaluacionMLIDS}, comprendiendo sus particularidades y la información que estos proporcionan para la detección de anomalías. Este \textit{sprint} concluyó con una clara comprensión del marco teórico y una dirección definida hacia la aplicación de técnicas de aprendizaje supervisado, sentando las bases conceptuales para las fases de implementación.

\textbf{\textit{Sprint} 2: Diseño del Módulo de Captura y Selección de Tecnologías} 

Con la base teórica establecida, el segundo \textit{sprint} se focalizó en el diseño arquitectónico del módulo de captura de paquetes y la selección de las tecnologías más adecuadas. La fase inicial de diseño incluyó la evaluación de posibles integraciones con herramientas existentes como Wireshark; sin embargo, las complejidades asociadas a su código base y las potenciales dificultades de integración con componentes de IA llevaron a su descarte. Se llevó a cabo una investigación profunda sobre los lenguajes de programación y las librerías disponibles para el desarrollo de capturadores de paquetes eficientes. Esta fase concluyó con la elección de Python, un lenguaje que demostró ser idóneo por su robustez, su amplio ecosistema de librerías para la manipulación de redes \cite{Shrefler2017Networking} y su facilidad de integración con \textit{frameworks} de \textit{Machine Learning} \cite{Pedregosa2011ScikitLearn}. Crucialmente, se identificó una API especializada que proporcionaba un conjunto rico de funcionalidades para la extracción de características detalladas a partir del tráfico de red. Al finalizar este \textit{sprint}, se dispuso de un diseño conceptual claro del capturador y una pila tecnológica definida para su implementación.

\textbf{\textit{Sprint} 3: Implementación Básica del Capturador de Paquetes y CLI}

Este \textit{sprint} se dedicó a la implementación inicial del capturador de paquetes en Python. Se desarrolló la funcionalidad \textit{core} que permite la intercepción y procesamiento de paquetes de red. El foco principal fue la creación de una interfaz de línea de comandos (CLI) que posibilitara la invocación del programa con parámetros específicos, tales como la interfaz de red a monitorear. Se logró que el capturador pudiera iniciar y detener la recolección de flujos de paquetes desde la terminal, sentando las bases operativas de la herramienta. Las pruebas iniciales se centraron en verificar la correcta captura y el \textit{parseo} básico de los datos de los paquetes, asegurando que el flujo de información se manejara de manera estable.

\textbf{\textit{Sprint} 4: Desarrollo de la Funcionalidad de Exportación de Datos}

Continuando con el módulo de captura, el cuarto \textit{sprint} se concentró en una funcionalidad crítica para la utilidad del proyecto: la exportación de los flujos de paquetes procesados a formatos estructurados. Reconociendo la dificultad de obtener \textit{datasets} de IDS actualizados, se implementó la capacidad de generar archivos en formato .csv y .txt a partir de los flujos capturados. Esta funcionalidad es fundamental, ya que permite que el IDS contribuya a la creación de nuevas bases de datos de tráfico de red, emulando la estructura de \textit{datasets} de referencia como los de la serie CIC-IDS. Esta capacidad es clave para la investigación y el desarrollo de modelos de \textit{Machine Learning}, al proporcionar acceso a datos más representativos del tráfico moderno. Se realizaron pruebas exhaustivas para validar la integridad y el formato de los archivos generados, asegurando que pudieran ser utilizados directamente para el preprocesamiento y entrenamiento de modelos de IA.

\textbf{\textit{Sprint} 5: Definición de la Arquitectura Web y Prototipado de la Interfaz}

Con el módulo de captura en progreso, este \textit{sprint} abordó el diseño y la arquitectura de la aplicación web que albergaría el IDS completo. Se llevó a cabo una investigación de \textit{frameworks} de desarrollo web \textit{full-stack}, evaluando opciones como Angular, React (para \textit{frontend}) combinados con \textit{backends} como Laravel (PHP) o Flask/Django (Python). La decisión final se inclinó por Reflex, un \textit{framework} de Python que ofrecía una solución unificada para el desarrollo \textit{full-stack} en un solo lenguaje, lo que simplificaba la integración con los componentes de Python existentes y optimizaba el flujo de trabajo. Una vez seleccionado el \textit{framework}, se procedió al análisis detallado y el diseño de la interfaz de usuario (UI). Esto incluyó la elaboración de diagramas de flujo de usuario, la creación de \textit{mockups} visuales y \textit{wireframes} para definir la estructura y navegación del \textit{dashboard}. El objetivo fue crear un diseño intuitivo que facilitara la visualización del tráfico de red y las futuras detecciones.

\textbf{\textit{Sprint} 6: Implementación del \textit{Frontend} y Conexión al \textit{Backend} del Capturador}

Este \textit{sprint} se centró en la construcción del \textit{frontend} de la aplicación web según los diseños establecidos en el \textit{sprint} anterior. Se implementaron las principales pantallas y componentes de la interfaz de usuario, dando vida al \textit{dashboard}. Simultáneamente, se inició el desarrollo de la capa de \textit{backend} de la aplicación web, preparada para recibir y procesar los datos del capturador. Un hito crítico de este \textit{sprint} fue la integración inicial del módulo de captura de paquetes con el \textit{backend} de la aplicación web. Para lograr una comunicación en tiempo real y asíncrona entre ambos componentes, se implementó un sistema basado en colas síncronas. Las pruebas de este \textit{sprint} se dirigieron a verificar que los datos de tráfico capturados por el módulo de Python se transmitieran correctamente al \textit{backend} y que el panel de visualización del tráfico de red en el \textit{frontend} se actualizara en tiempo real, confirmando la operatividad del flujo de datos de extremo a extremo.

\textbf{\textit{Sprint} 7: Preprocesamiento y Unificación del \textit{Dataset} para ML}

Paralelo al avance de la aplicación web, este \textit{sprint} se dedicó intensivamente a la preparación del conjunto de datos para el entrenamiento del modelo de \textit{Machine Learning}. Se seleccionó el CIC-IDS2018 como \textit{dataset} principal debido a su representatividad y variedad de ataques \cite{PolaniaArias2021EvaluacionMLIDS}. La tarea crucial fue la unificación de los diversos ficheros que componían este \textit{dataset} (correspondientes a diferentes días de captura y tipos de ataque) en una única base de datos cohesiva. Esta consolidación fue fundamental para asegurar que el modelo de aprendizaje tuviera acceso a un volumen variado y robusto de flujos de tráfico, crucial para su capacidad de generalización \cite{PolaniaArias2021EvaluacionMLIDS}. Tras la unificación, se llevó a cabo un exhaustivo proceso de preprocesamiento, análisis y limpieza de los datos. Esto incluyó la gestión de valores ausentes, la corrección de inconsistencias, la normalización/estandarización de características numéricas y la codificación de variables categóricas, preparando el \textit{dataset} para la fase de entrenamiento \cite{James2013ISLR}.

\textbf{\textit{Sprint} 8: Entrenamiento y Evaluación del Modelo de \textit{Machine Learning} (\textit{Random Forest})}

Una vez que el \textit{dataset} estuvo preparado y limpio, este \textit{sprint} se centró por completo en el desarrollo y la evaluación del modelo de \textit{Machine Learning}. Se procedió al entrenamiento del clasificador \textit{Random Forest}, un algoritmo elegido por su eficacia en la detección de intrusiones en entornos de alta dimensionalidad y su robustez ante el sobreajuste \cite{Breiman2001RandomForests}. Durante esta fase, se ajustaron los hiperparámetros del modelo para optimizar su rendimiento. Tras el entrenamiento, se realizó una evaluación rigurosa del modelo utilizando métricas clave de clasificación como la precisión (\textit{accuracy}), la sensibilidad (\textit{recall}), la precisión (\textit{precision}), el \textit{F1-score}, y el análisis detallado de la matriz de confusión y la curva ROC con su Área bajo la Curva (AUC) \cite{PolaniaArias2021EvaluacionMLIDS}. Este análisis permitió determinar la capacidad predictiva del modelo y su fiabilidad para distinguir entre tráfico benigno y malicioso. El entregable de este \textit{sprint} fue un modelo de \textit{Machine Learning} entrenado y validado, listo para ser integrado en el sistema en tiempo real.

\textbf{\textit{Sprint} 9: Integración Final del Modelo ML y Detección en Tiempo Real}

Este \textit{sprint} fue el punto culminante de la integración de todos los componentes del IDS. El modelo de \textit{Machine Learning} entrenado fue integrado en el \textit{backend} de la aplicación web. Esto permitió que los flujos de paquetes, capturados por el módulo Python y transmitidos a través de las colas síncronas al \textit{backend}, fueran ahora analizados en tiempo real por el modelo \textit{Random Forest}. La funcionalidad clave de este \textit{sprint} fue la capacidad del sistema para determinar si un flujo de tráfico era benigno o malicioso de forma automática y mostrar esta clasificación instantáneamente en la interfaz web. Se realizaron pruebas de integración de extremo a extremo para asegurar que el ciclo completo, desde la captura hasta la predicción y la visualización, operara sin fallos y con la menor latencia posible.

\textbf{\textit{Sprint} 10: Pruebas Exhaustivas, Optimización y Validación Final}

El \textit{sprint} final se dedicó a la consolidación y refinamiento de todo el sistema. Se ejecutaron multitud de pruebas exhaustivas y multifacéticas para garantizar la robustez, la estabilidad y el rendimiento óptimo de la aplicación en su conjunto. Esto incluyó pruebas de estrés para verificar la capacidad del capturador y del modelo bajo cargas de tráfico elevadas, pruebas de usabilidad de la interfaz web, y pruebas de regresión para asegurar que las nuevas funcionalidades no introdujeran fallos en componentes existentes. Asimismo, se llevó a cabo una fase de optimización general del rendimiento de la aplicación, buscando mejorar los tiempos de respuesta del capturador, la latencia en la predicción del modelo y la fluidez de la interfaz de usuario. Finalmente, se realizó una validación completa del sistema \cite{NIST2020SP800-115} para confirmar que todos los objetivos del proyecto habían sido alcanzados, que el IDS operaba de manera confiable y que estaba listo para su defensa.

\section{Presupuesto del proyecto}

La planificación y ejecución de cualquier proyecto, incluyendo un Trabajo Fin de Grado, conlleva una serie de costes asociados que deben ser estimados y desglosados para comprender la inversión económica necesaria. Este apartado detalla el presupuesto aproximado del desarrollo del Sistema de Detección de Intrusiones (IDS) y la herramienta de generación de \textit{datasets} a lo largo de los ocho meses de duración estimados para el proyecto. Para la elaboración de este presupuesto, se han considerado los recursos humanos implicados, los recursos materiales utilizados y los costes indirectos asociados.

\textbf{Recursos Humanos}

Aunque este proyecto ha sido realizado por un único estudiante en el marco de su Trabajo Fin de Grado, para fines de presupuesto y para reflejar un escenario de desarrollo profesional, se estimará el coste equivalente a la contratación de un Programador \textit{Junior} / Investigador \textit{Junior} en Ciberseguridad. Este perfil asumiría las responsabilidades de diseño, programación, investigación y gestión del proyecto.

Para la estimación salarial, se puede tomar como referencia el Convenio colectivo estatal de empresas de consultoría y estudios de mercado y de la opinión pública, o bien, promedios de salarios para perfiles \textit{junior} en el sector tecnológico en España. Considerando un salario anual bruto para un programador/investigador \textit{junior} en un rango de 18.000€ a 22.000€ (más realista para el perfil y las habilidades requeridas en \textit{Machine Learning} y ciberseguridad que los 14.800,66€ mencionados en un convenio específico que puede no aplicar directamente al rol técnico avanzado), tomaremos un valor medio de 20.000€ anuales para este cálculo.
\begin{itemize}

    \item\textbf{Salario base anual estimado}: 20.000,00 \euro{}
    
    \item \textbf{Salario mensual:}$\displaystyle \frac{20.000,00}{12} = 1.666,67 $ \euro{}/mes
   
    \item\textbf{Coste total de recursos humanos (8 meses)}:$\displaystyle 1.666,67 \text{ \euro{}/mes} \times 8 \text{ meses} = 13.333,36 $ \euro{}

\end{itemize}

A este coste salarial se le deberían añadir los costes sociales (aportaciones a la Seguridad Social por parte de la empresa), que en España suelen rondar el 30-35\% del salario bruto. Para una estimación, consideraremos un 32\%.
\begin{itemize}
    
    \item\textbf{Costes Sociales mensuales}:$\displaystyle 1.666,67 \text{ \euro{}/mes} \times 0.32 = 533,33 $ \euro{}/mes
    
    \item\textbf{Costes Sociales totales (8 meses)}:$\displaystyle 533,33 \text{ \euro{}/mes} \times 8 \text{ meses} = 4.266,64 $ \euro{}
    
    \item\textbf{Coste total de Recursos Humanos (Salario + Costes Sociales)}:$\displaystyle 13.333,36 \text{ \euro{}} + 4.266,64 \text{ \euro{}} = 17.600,00 $ \euro{}
    
\end{itemize}

\textbf{Recursos Materiales}
 
El desarrollo del proyecto ha requerido el uso de equipamiento informático para la programación, la ejecución de modelos de \textit{Machine Learning} y la realización de pruebas de captura de tráfico en tiempo real. Aunque los equipos ya eran propiedad del desarrollador, para el presupuesto se considerará una amortización proporcional al tiempo de uso del proyecto.

Se han utilizado los siguientes equipos principales:

\begin{itemize}

    \item\textbf{Ordenador Portátil (HP Laptop 15s-fq1xxx)}:
        \begin{itemize}

            \item\textbf{Coste de adquisición}: 800 \euro{} 
            
            \item\textbf{Vida útil estimada}: 4 años (48 meses)
            
            \item\textbf{Amortización mensual}l:$\displaystyle \frac{800}{48} = 16,66 $ \euro{}/mes
            
            \item\textbf{Coste por 8 meses}:$\displaystyle 16,66 \text{ \euro{}/mes} \times 8 \text{ meses} = 133,28 $ \euro{}
                   
        \end{itemize}
    
    \item\textbf{\textit{Smartphone} (OnePlus 10 pro, para pruebas de red móvil si aplicase, o como dispositivo de \textit{test} para la \textit{app} web)}:
    
        \begin{itemize}
        
            \item\textbf{Coste de adquisición}: 500,00 \euro{} (basado en la referencia del compañero)
            
            \item\textbf{Vida útil estimada}: 3 años (36 meses)
            
            \item\textbf{Amortización mensual}:$\displaystyle \frac{500,00}{36} = 13.88 $ \euro{}/mes
            
            \item\textbf{Coste por 8 meses}:$\displaystyle 13.88 \text{ \euro{}/mes} \times 8 \text{ meses} = 111,04 $ \euro{}
                
        \end{itemize}
    
    \item\textbf{\textit{Software} y Herramientas (Licencias/Suscripciones)}:
    Aunque la mayoría de las herramientas utilizadas (Python, librerías como Scikit-learn, Reflex, Wireshark/Scapy, etc.) son de código abierto o gratuitas, en un entorno profesional se considerarían costes para herramientas de desarrollo (IDEs avanzados), sistemas de control de versiones \textit{premium} (GitHub Enterprise), o servicios de computación en la nube para el entrenamiento de modelos más grandes (ej. Google Colab Pro, AWS, Azure, etc.). Para este TFG, asumiremos costes mínimos o nulos por licencias de \textit{software}, pero se podría estimar un coste simbólico por el acceso a ciertos recursos o licencias de herramientas de diseño/gestión si se hubieran utilizado versiones de pago. Aquí asumiremos 0€ dado el contexto de TFG, pero se podría incluir un pequeño monto (ej. 50-100€) para una licencia de IDE o \textit{software} de diseño.
    
    \item\textbf{Conexión a Internet y Electricidad}:
    Estos costes se suelen incluir en los costes operativos indirectos de un proyecto. Considerando una parte proporcional del coste de una conexión doméstica y el consumo eléctrico del equipo durante las horas de desarrollo.
    
        \begin{itemize}
        
            \item\textbf{Conexión a Internet}:$\displaystyle \frac{50 \text{ \euro{}/mes}}{2} \text{ (uso profesional)} \times 8 \text{ meses} = 200,00 $ \euro{}
            
            \item\textbf{Electricidad}:$\displaystyle \frac{30 \text{ \euro{}/mes}}{2} \text{ (uso profesional)} \times 8 \text{ meses} = 120,00 $ \euro{}
                    
        \end{itemize}    
    
    \item\textbf{Total de Recursos Materiales}:$\displaystyle 133,28 \text{ \euro{}} + 111,04 \text{ \euro{}} + 0,00 \text{ \euro{}} + 200,00 \text{ \euro{}} + 120,00 \text{ \euro{}} = 564,32 $ \euro{}
   
\end{itemize}

\textbf{Costes Indirectos y Contingencias} 

Estos costes cubren gastos generales y posibles imprevistos no directamente asignables a las categorías anteriores. Incluyen, por ejemplo, material de oficina, formación específica no prevista, gastos de comunicación, etc. Se suele estimar un porcentaje sobre el total de los costes directos.
\begin{itemize}
    \item\textbf{Costes Directos Totales (Recursos Humanos + Recursos Materiales)}: $\displaystyle 17.600,00 \text{ \euro{}} + 564,32 \text{ \euro{}} = 18.164,32 \text{ \euro{}}$
    
    \item\textbf{Contingencias (10\% de los costes directos)}:$\displaystyle 18.164,32 \text{ \euro{}} \times 0.10 = 1.816,43 \text{ \euro{}}$
\end{itemize}

\textbf{Resumen del presupuesto final del proyecto}

En la siguiente tabla ~\ref{Tabla.Presupuesto} se puede observar el resumen del desglose de los costes estimados para el desarrollo de este Trabajo Fin de Grado durante un periodo de 8 meses:

\input{tablas/tablaPresupuesto}

Este presupuesto final de aproximadamente 19.980,75 euros proporciona una estimación realista del valor de mercado de un proyecto de esta envergadura y complejidad si se llevara a cabo en un entorno profesional, abarcando la investigación, el desarrollo de un módulo de captura especializado, la implementación de un modelo de \textit{Machine Learning} y la creación de una aplicación web interactiva.

\section{Planificación de tiempo y costes}
La gestión del tiempo es un pilar fundamental en la ejecución exitosa de cualquier proyecto, especialmente en un Trabajo Fin de Grado (TFG) que requiere una secuencia lógica de actividades y una dedicación constante. La planificación temporal de este proyecto se ha basado en la metodología ágil de \textbf{\textit{Sprints} de Desarrollo} citados en el apartado ~\ref{Sec.Sprints}, la cual ha permitido una organización estructurada de las tareas, la adaptación a los desafíos y la entrega incremental de funcionalidades, tal como se detalló en la sección de Metodología.

Para la estimación y seguimiento del tiempo, se ha considerado un período de ocho meses, que abarca desde la fase inicial de los fundamentos teóricos hasta la validación y documentación final. La duración de cada \textit{sprint} fue estimada en dos semanas aproximadamente, permitiendo flexibilidad para la investigación y el desarrollo intensivo de cada módulo. Este enfoque se alinea con prácticas comunes en la planificación de proyectos de ingeniería de \textit{software}, buscando optimizar el esfuerzo y los recursos disponibles \cite{NIST2020SP800-115}.

Para este proyecto, el cronograma se presenta como una secuencia de los \textit{sprints} ya definidos, cada uno con sus objetivos específicos y una duración estimada.

\subsection{Cronograma del Proyecto}

A continuación, se presenta el cronograma detallado del proyecto, desglosado por los \textit{sprints} de desarrollo en la figura ~\ref{Fig.Cronograma}. Este cronograma visualiza la secuencia de las actividades principales y la estimación de su duración en meses.

\begin{figure}[ht!]
  \centering
  \includegraphics[width=1\textwidth]{imagenes/cronograma.png}
  \caption{Diagrama de Gantt del proyecto. Fuente: Elaboración propia.}
  \label{Fig.Cronograma}
\end{figure}


Este cronograma proporciona una guía para la ejecución del proyecto, destacando la superposición de algunos \textit{sprints} (como la investigación teórica con el diseño inicial, o el preprocesamiento de datos con el desarrollo web) para optimizar el tiempo. La naturaleza iterativa de la metodología ágil permite ajustes en la duración de los \textit{sprints} según las necesidades emergentes y la complejidad de las tareas. La dedicación principal al proyecto se concentró en la ejecución de estos \textit{sprints}, culminando con una fase intensiva de integración y pruebas antes de la entrega final.

\subsection{Estimación del Esfuerzo}

La estimación del esfuerzo para el presente Trabajo de Fin de Grado se ha realizado en conformidad con la carga académica establecida para un TFG de 12 créditos ECTS, lo que se traduce en un total de 300 horas de trabajo dedicado. Este volumen de esfuerzo se ha distribuido estratégicamente a lo largo de la duración del proyecto, que abarca desde Noviembre de 2024 hasta finales de Julio de 2025, como se detalla en el cronograma.

Para un perfil de Programador/Investigador \textit{Junior}, la dedicación requerida para un proyecto de esta envergadura se traduce en una asignación meticulosa de horas a cada fase, cubriendo desde la investigación y el diseño hasta la implementación, depuración y la elaboración de la documentación. Este enfoque es consistente con las prácticas comunes en la planificación de proyectos de \textit{software}.

El esfuerzo total de 300 horas se ha distribuido entre las distintas fases y actividades del proyecto, alineándose con la metodología de \textit{sprints} empleada. A continuación, se detalla la asignación de estas horas en las principales etapas:

\begin{itemize}

    \item\textbf{Análisis y Diseño}: Incluye la investigación teórica, la definición de requisitos funcionales y no funcionales, y la conceptualización de la arquitectura inicial del sistema.
    
    \item\textbf{Desarrollo e Implementación}: Abarca la codificación y construcción de los diversos módulos, como el capturador de tráfico, la herramienta de generación de \textit{datasets}, el componente de \textit{Machine Learning} y la interfaz de usuario web.
    
    \item\textbf{Pruebas y Validación}: Comprende la ejecución de baterías de pruebas, la validación del rendimiento del modelo de \textit{Machine Learning} y la verificación integral de la funcionalidad del sistema.
    
    \item\textbf{Documentación}: Incluye la redacción exhaustiva de esta memoria final de grado, la cual culminará a finales de julio de 2025, y la preparación de otros materiales de apoyo.
    
    Esta distribución del esfuerzo, similar a la aplicada en otros proyectos académicos que buscan cumplir con un número de créditos específicos, asegura una asignación proporcional de recursos al trabajo técnico, la investigación y la formalización documental. La planificación detallada y el progreso de estas actividades se visualizan en el Diagrama de Gantt del proyecto, presentado en la Figura ~\ref{Fig.Cronograma}.


\end{itemize}
\section{Estructura de la memoria}
La memoria final de este proyecto está organizada en una serie de capítulos que se ajustan a la siguiente estructura:

\begin{itemize}
    \item{Capítulo 2}: Capítulo en el que se discute el estado del arte de sistemas de ciberseguridad que aplican ML, como diversas líneas de investigación y desarrollo al respecto
    \item{Capítulo 3}: Capítulo en el que se abordan los materiales y métodos empleados en el desarrollo del proyecto, haciéndose especial énfasis en cada una de las funcionalidades desarrolladas tanto en el capturador de paquetes como en el propio modelo de \textit{machine learning} empleado para la construcción de la aplicación final para usuarios.
    \item{Capítulo 4}: Capítulo en el que se profundiza en el análisis del sistema propuesto. Se especifican los requisitos funcionales y no funcionales que definen las funcionalidades y limitaciones de la aplicación. Además, se presentan diagramas de casos de uso, de secuencia y de flujo, que ilustran a un alto nivel la arquitectura y el comportamiento del sistema.
    \item{Capítulo 5}: Este capítulo documenta el proceso de diseño detallado del sistema, traduciendo los requisitos en una arquitectura concreta. Se presentan los modelos de datos, los diagramas de clases y la estructura modular de la aplicación, definiendo cómo interactúan los componentes del capturador, el modelo de \textit{Machine Learning} y la interfaz de usuario para construir una solución robusta y escalable.
    \item{Capítulo 6}: En este capítulo se describen los resultados obtenidos tras la implementación y las pruebas del sistema. Se detallan las métricas de rendimiento del modelo de \textit{Machine Learning} y los hallazgos más relevantes de la evaluación. Se presentan los resultados de la validación del sistema para verificar que cumple con los requisitos iniciales.
    \item{Capítulo 7}: Este apartado presenta las conclusiones del proyecto. Se resumen los logros alcanzados, los desafíos superados y las principales lecciones aprendidas durante el desarrollo. También se discuten las limitaciones del sistema y se proponen posibles mejoras, extensiones y futuras líneas de investigación.
    \item{Capítulo 8}: Este capítulo incluye cualquier material adicional que, aunque no es esencial para la comprensión del texto principal, complementa la información de la memoria. Incluye detalles técnicos adicionales sobre la instalación y configuración del sistema, capturas de pantalla de la interfaz y manual de usuario.
    \item{Capítulo 9}: Este capítulo contiene un glosario de términos clave, acrónimos y abreviaturas utilizados en la memoria. Su propósito es ayudar al lector a comprender la terminología técnica del proyecto, garantizando la claridad y la consistencia a lo largo del documento.
    
\end{itemize}  
\clearpage\thispagestyle{empty}\cleardoublepage

\chapter{Antecedentes}
\section{Introducción}
El presente capítulo tiene como objetivo establecer el marco teórico y contextual necesario para comprender el alcance y la relevancia del Trabajo de Fin de Grado. Se abordarán los fundamentos de los modelos de clasificación de \textit{Machine Learning} que sustentan el enfoque propuesto, se realizará un exhaustivo recorrido por el estado del arte en sistemas de detección de intrusiones (IDS) y se presentarán trabajos relacionados clave que han contribuido al desarrollo de esta disciplina. Finalmente, se analizarán las tecnologías existentes que son relevantes para la implementación de un sistema de detección de anomalías en tráfico de red.

\section{Modelos de Clasificación para Machine Learning}

Los modelos de clasificación son un pilar fundamental en el ámbito del \textit{Machine Learning}, especialmente en aplicaciones donde es necesario categorizar datos en clases predefinidas. En el contexto de la ciberseguridad, estos modelos permiten discernir entre tráfico de red legítimo y malicioso, o incluso identificar el tipo específico de un ataque. A continuación, se describen los modelos de clasificación más relevantes y su aplicabilidad en este dominio.

\subsection{Random Forest}
\textit{Random Forest} ~\cite{breiman2001random} es un algoritmo de \textit{Machine Learning} basado en el concepto de \textit{ensemble learning}, específicamente mediante el método de \textit{bagging} (\textit{Bootstrap Aggregating}). Este modelo construye múltiples árboles de decisión durante la fase de entrenamiento y, para la clasificación, la salida es la clase que más votan los árboles individuales (o el promedio de las predicciones para regresión). Su robustez se debe a la combinación de predicciones de múltiples árboles, lo que reduce la varianza y el sobreajuste, problemas comunes en árboles de decisión únicos.

Lo que diferencia a este modelo de clasificación de otros de \textit{ensemble} es el hecho de que, para cada árbol de forma individual, se eligen una serie de características del conjunto de datos en lugar del conjunto completo de características. A continuación, a partir de los resultados proporcionados por los árboles individuales, se puede determinar qué etiqueta o clase es la más apropiada para cada dato.

En el ámbito de los IDS, \textit{Random Forest} ha demostrado ser particularmente efectivo debido a su capacidad para manejar grandes volúmenes de datos con alta dimensionalidad, su resistencia a características irrelevantes y su habilidad para estimar la importancia de cada característica, lo cual es crucial para la selección de atributos en el tráfico de red. La capacidad de este algoritmo para proporcionar una alta precisión y su relativamente bajo riesgo de sobreajuste lo hacen una opción atractiva para la detección de anomalías y ataques en redes, como se ha evidenciado en la implementación de este proyecto.

Véase en la siguiente ilustración~\ref{Fig.RandomForest} la arquitectura de clasificación del modelo \textit{Random Forest}.

\begin{figure}[ht!]
  \centering
  \includegraphics[width=0.8\textwidth]{imagenes/estructura_rf.png}
  \caption{Esquema del modelo Random Forest. Fuente: Elaboración propia.}
  \label{Fig.RandomForest}
\end{figure}


\subsection{Support Vector Machine (SVM)}
Las \textit{Support Vector Machines} (SVM)~\cite{aljamal2019hybridIDS} son modelos de aprendizaje supervisado utilizados tanto para problemas de clasificación como de regresión. El principio fundamental de SVM radica en encontrar el hiperplano óptimo que mejor separe las clases en un espacio de características de alta dimensionalidad. El "óptimo" se refiere al hiperplano que maximiza el margen entre las clases (la distancia entre el hiperplano y los puntos de datos más cercanos de cada clase, conocidos como vectores de soporte).

Las SVM son particularmente potentes en espacios de alta dimensionalidad y son efectivas en casos donde el número de dimensiones es mayor que el número de muestras. Sin embargo, su complejidad computacional puede ser una limitación en \textit{datasets} muy grandes, como los que se encuentran comúnmente en el análisis de tráfico de red en tiempo real.

Véase en la siguiente ilustración~\ref{Fig.SVM} la arquitectura de clasificación del modelo \textit{SVM}.

\begin{figure}[ht!]
  \centering
  \includegraphics[width=0.8\textwidth]{imagenes/diagramas/modelos_ml/svm.png}
  \caption{Esquema del modelo SVM.}
  \label{Fig.SVM}
\end{figure}

\subsection{Naive Bayes}
El clasificador \textit{Naive Bayes}~\cite{ibmNaiveBayes} es una familia de algoritmos de clasificación probabilística simple basada en el Teorema de Bayes con una "ingenua" suposición de independencia fuerte entre las características. A pesar de su simplicidad y de la suposición de independencia (que a menudo no se cumple en la realidad), los clasificadores \textit{Naive Bayes} han demostrado un rendimiento sorprendentemente bueno en muchas aplicaciones prácticas, especialmente en el procesamiento de lenguaje natural y la clasificación de texto.

Para la detección de intrusiones, su simplicidad y velocidad de entrenamiento lo hacen adecuado para conjuntos de datos grandes y para escenarios donde la eficiencia computacional es crítica. No obstante, la suposición de independencia entre las características de red (como puertos, protocolos, tamaños de paquete) puede limitar su precisión en comparación con modelos más complejos que capturan las interacciones entre ellas.

Véase en la siguiente ilustración~\ref{Fig.nb} la arquitectura de clasificación del modelo \textit{Naive Bayes}.

\begin{figure}[ht!]
  \centering
  \includegraphics[width=0.8\textwidth]{imagenes/diagramas/modelos_ml/naive bayes.png}
  \caption{Esquema del modelo Naive Bayes.}
  \label{Fig.nb}
\end{figure}

\subsection{K-Nearest Neighbors (KNN)}
\textit{K-Nearest Neighbors} (KNN)~\cite{ibmKNN} es un algoritmo de clasificación no paramétrico y basado en instancias. Funciona clasificando nuevos puntos de datos basándose en la mayoría de votos de sus "K" vecinos más cercanos en el espacio de características. La distancia entre los puntos de datos (p.\,ej., distancia euclidiana) es crucial para determinar la "cercanía" de los vecinos.

KNN es simple de entender e implementar y no requiere una fase de entrenamiento explícita, ya que memoriza el conjunto de datos de entrenamiento. Sin embargo, su principal desventaja en el contexto de IDS es la alta complejidad computacional durante la fase de predicción, ya que necesita calcular las distancias a todos los puntos de entrenamiento para cada nueva instancia. Esto lo hace menos adecuado para sistemas en tiempo real que manejan un flujo constante y elevado de tráfico de red.

Véase en la siguiente ilustración~\ref{Fig.knn} la arquitectura de clasificación del modelo \textit{KNN}.

\begin{figure}[ht!]
  \centering
  \includegraphics[width=0.8\textwidth]{imagenes/diagramas/modelos_ml/knn.png}
  \caption{Esquema del modelo KNN.}
  \label{Fig.knn}
\end{figure}

\section{Estado del Arte en Sistemas de Detección de Intrusiones (IDS)}

La ciberseguridad es un campo en constante evolución, y los Sistemas de Detección de Intrusiones (IDS) son componentes críticos para salvaguardar la integridad, confidencialidad y disponibilidad de los sistemas de información. El estado del arte actual se caracteriza por una convergencia creciente entre las metodologías tradicionales de detección y las capacidades emergentes del \textit{Machine Learning} y la Inteligencia Artificial.

\subsection{Evolución de los IDS: Desde Firmas a Anomalías}
Tradicionalmente, los IDS se han clasificado en dos categorías principales: basados en firmas (\textit{Signature-based IDS - SIDS}) y basados en anomalías (\textit{Anomaly-based IDS - AIDS}). Los SIDS detectan intrusiones buscando patrones o firmas predefinidos de ataques conocidos en el tráfico de red o en los registros del sistema. Si bien son muy efectivos para identificar ataques ya conocidos, su principal limitación es la incapacidad de detectar nuevas amenazas (ataques de día cero).

En contraste, los AIDS se centran en identificar desviaciones del comportamiento normal o esperado del sistema o la red. Para ello, construyen un perfil de "normalidad" y cualquier actividad que se desvíe significativamente de este perfil se marca como una anomalía, y potencialmente como una intrusión. La emergencia del \textit{Machine Learning} ha potenciado enormemente la capacidad de los AIDS para aprender patrones complejos de tráfico normal y detectar actividades maliciosas novedosas, superando las limitaciones de los SIDS.

\subsection{IDS Basados en Machine Learning: Taxonomía y Enfoques}
La aplicación de \textit{Machine Learning} en IDS ha revolucionado la forma en que se detectan las amenazas. Los enfoques de IDS basados en ML pueden clasificarse en:
\begin{itemize}
\item \textbf{Supervisados:} Utilizan conjuntos de datos etiquetados (tráfico normal y de ataque) para entrenar modelos que aprenden a clasificar nuevas instancias. Modelos como \textit{Random Forest}, SVM o Redes Neuronales son comunes aquí.
\item \textbf{No Supervisados:} Emplean algoritmos para descubrir patrones y estructuras ocultas en datos no etiquetados, identificando como anomalías aquellas instancias que no se ajustan a estos patrones. La detección de \textit{outliers} es un ejemplo.
\item \textbf{Semi-supervisados:} Combinan un pequeño conjunto de datos etiquetados con una gran cantidad de datos no etiquetados para el entrenamiento, siendo útil cuando la anotación de datos es costosa.
\item \textbf{Basados en Redes Profundas (\textit{Deep Learning}):} Representan una evolución de los modelos de ML, utilizando arquitecturas de redes neuronales con múltiples capas. Modelos como las Redes Neuronales Convolucionales (CNN) son eficientes en la extracción de características jerárquicas de datos secuenciales, y las Redes Neuronales Recurrentes (RNN), especialmente las variantes como LSTM o GRU, son idóneas para el análisis de series temporales de tráfico de red. Recientemente, los modelos basados en \textit{Transformers} también han mostrado un gran potencial en el análisis de secuencias de paquetes.
\end{itemize}
Los \textit{datasets} utilizados para el entrenamiento y evaluación de estos modelos son cruciales, siendo algunos de los más reconocidos CIC-IDS2017, CIC-CSE-IDS2018 y KDD Cup 1999.

\subsection{Detección de Ataques Específicos Mediante IA}
La capacidad del \textit{Machine Learning} permite ir más allá de la simple detección de intrusiones, facilitando la identificación de tipos específicos de ataques con un alto grado de precisión. Por ejemplo, los modelos de clasificación pueden ser entrenados para reconocer patrones asociados con:
\begin{itemize}
\item \textbf{Ataques de Denegación de Servicio Distribuido (DDoS):} Identificando volúmenes anómalos de tráfico o patrones de conexión coordinados.
\item \textbf{Escaneos de Puertos (\textit{Port Scan}):} Detectando intentos de sondear puertos abiertos en un sistema.
\item \textbf{Inyecciones SQL (\textit{SQL Injection}):} Analizando el contenido de las peticiones web en busca de secuencias de comandos maliciosas.
\end{itemize}
Estas detecciones específicas, como las que se presentan en interfaces de \textit{dashboard}, son vitales para la respuesta a incidentes y la implementación de contramedidas adecuadas.

\subsection{Desafíos y Futuro en la Detección de Intrusiones con IA}
A pesar de los avances significativos, la aplicación de IA en IDS enfrenta desafíos importantes. El "desplazamiento de concepto" (\textit{concept drift}), donde los patrones de ataque evolucionan y los modelos previamente entrenados pueden volverse obsoletos, exige la implementación de mecanismos de re-entrenamiento continuo. El desbalance de clases en los \textit{datasets} (donde el tráfico normal es significativamente más abundante que el tráfico malicioso) es otro obstáculo que puede llevar a modelos sesgados y con baja capacidad para detectar las clases minoritarias. Además, la interpretabilidad de los modelos complejos de \textit{Deep Learning} es un área activa de investigación, ya que comprender por qué un modelo clasifica una actividad como maliciosa es crucial para los analistas de seguridad en la toma de decisiones. El futuro de los IDS con IA se dirige hacia el desarrollo de sistemas más adaptativos, proactivos y explicables, capaces de operar eficazmente en entornos de alta velocidad y volumen, integrando aprendizaje continuo y capacidades de respuesta automatizada.

\section{Trabajos Relacionados}

La literatura científica y técnica ofrece un vasto cuerpo de trabajos relacionados que han explorado la aplicación de técnicas de \textit{Machine Learning} para la detección de intrusiones en redes. Muchos estudios se centran en la experimentación con diversos modelos de clasificación sobre \textit{datasets} de referencia como CIC-IDS2017 o KDD Cup 1999, evaluando su rendimiento en términos de precisión, \textit{recall} y \textit{F1-score} para identificar diferentes categorías de ataques.

Algunas investigaciones se han enfocado en la selección de características (\textit{feature selection}) para optimizar el rendimiento de los modelos, buscando los atributos más relevantes del tráfico de red que permiten una detección eficaz y reducen la complejidad computacional. Otros trabajos han abordado la problemática del desbalance de clases en los \textit{datasets} de ciberseguridad, proponiendo técnicas de sobremuestreo o submuestreo para mejorar la capacidad de detección de ataques minoritarios.

La integración de capturadores de tráfico en tiempo real con módulos de \textit{Machine Learning} es también un área de investigación activa, buscando cerrar la brecha entre la detección \textit{offline} y la capacidad de respuesta inmediata. Proyectos similares al presente, que buscan desarrollar sistemas capaces de procesar flujos de paquetes en tiempo real y aplicar algoritmos de clasificación para identificar anomalías, han allanado el camino para soluciones más proactivas en ciberseguridad.

\section{Tecnologías Existentes}

El desarrollo de un sistema de detección de intrusiones basado en \textit{Machine Learning} se apoya en un ecosistema de tecnologías y herramientas existentes que facilitan la captura de datos, su procesamiento, el entrenamiento de modelos y la visualización de resultados.

En el ámbito de la captura y análisis de tráfico de red, herramientas como Wireshark o tcpdump son estándares de la industria que permiten la interceptación y el análisis profundo de paquetes. Wireshark, en particular, ofrece capacidades de descifrado y una interfaz gráfica detallada para la inspección manual del tráfico.

Para el desarrollo de los algoritmos de \textit{Machine Learning}, Python se ha consolidado como el lenguaje de programación por excelencia en ciencia de datos, gracias a su rica colección de librerías. Scikit-learn proporciona una implementación robusta de una amplia gama de algoritmos de clasificación, incluyendo \textit{Random Forest}, SVM, \textit{Naive Bayes} y KNN. Para enfoques de \textit{Deep Learning}, TensorFlow y PyTorch son los \textit{frameworks} dominantes, ofreciendo flexibilidad y eficiencia computacional para la construcción y entrenamiento de redes neuronales complejas.

El procesamiento y manipulación de grandes volúmenes de datos de red se beneficia enormemente de librerías como Pandas para la gestión de \textit{DataFrames} y NumPy para operaciones numéricas. La visualización de los datos y los resultados del sistema, como un \textit{dashboard} de alertas, se puede lograr con \textit{frameworks} de desarrollo web como Flask, Django o reflex en Python, combinados con librerías de visualización \textit{front-end}.

Finalmente, el uso de entornos de desarrollo integrado (IDE) como VS Code o PyCharm, junto con sistemas de control de versiones como Git y plataformas como GitHub, son tecnologías esenciales que aseguran un desarrollo colaborativo, organizado y eficiente del proyecto.
\clearpage\thispagestyle{empty}\cleardoublepage

\chapter{ANÁLISIS DEL SISTEMA}
El análisis es una etapa crucial en el ciclo de vida del desarrollo de software, ya que establece una base sólida para el diseño y la implementación del sistema. En este apartado, se detallará la fase de análisis del proyecto, que incluye la identificación y especificación de los requisitos funcionales y no funcionales, la creación de los correspondientes diagramas de casos de uso, la especificación gramatical de cada caso de uso, la creación de un diagrama de flujo principal, de actividades y por último, la elaboración de diagramas de secuencia correspondientes.
\section{Requisitos del sistema}
Los requisitos del sistema describen las funciones, características y limitaciones del sistema de detección de intrusiones basado en Machine Learning. Estos se han dividido en dos categorías principales: requisitos funcionales y requisitos no funcionales.
\subsection{Requisitos funcionales}
Los requisitos funcionales definen las funcionalidades que el sistema debe proporcionar para cumplir con los objetivos del proyecto~\cite{iso29148}. Se presentan en la siguiente tabla ~\ref{Tabla.requisitos_funcionales}.

\input{tablas/tablaRF}
\subsection{Requisitos no funcionales}
Los requisitos no funcionales se refieren a las características de calidad del sistema, como la usabilidad, la fiabilidad y el rendimiento~\cite{iso29148}. En este subapartado, se detallan estos requisitos, proporcionando una visión clara de los estándares de calidad que se espera que la aplicación cumpla. A continuación, se detallan en la siguiente tabla ~\ref{Tabla.requisitos_no_funcionales}.

\input{tablas/tablaRNF}

\subsection{Diagrama de requisitos y trazabilidad inicial}

\begin{landscape}
\begin{figure}[p]
  \centering
  % Aire lateral de seguridad (ajústalo)
  \newlength{\sideair}\setlength{\sideair}{-2.4cm}

  % Ampliar más allá de textwidth pero sin llegar al borde de la página
  \begin{adjustwidth*}
    {-\dimexpr(\paperwidth-\textwidth)/2 + \sideair\relax}
    {-\dimexpr(\paperwidth-\textwidth)/2 + \sideair\relax}
    \centering
    \adjustbox{max totalsize={\dimexpr\paperwidth - 2\sideair\relax}{\textheight}}{%
      \begin{tabular}{@{}c@{}}
        \includegraphics[width=\linewidth,pagebox=cropbox]{imagenes/diagramas/requirements/requirementsPart1cropped.pdf}\\[-0.35em]
        \includegraphics[width=\linewidth,pagebox=cropbox]{imagenes/diagramas/requirements/requirementsPart2cropped.pdf}
      \end{tabular}%
    }
  \end{adjustwidth*}

  \caption{Diagrama de requisitos y trazabilidad (Parte 1 y Parte 2).}
  \label{fig:diagRequisitosSplit}
\end{figure}
\end{landscape}

\subsection*{Diagrama de requisitos y trazabilidad (resumen de RF y RNF)}
La Figura ~\ref{fig:diagRequisitosSplit} integra en una sola vista los requisitos funcionales (RFx) y no funcionales (RNFx) del sistema y su trazabilidad con los elementos de diseño (casos de uso, \texttt{CU\_*}). 

\paragraph{Qué muestra el diagrama}
\begin{itemize}
  \item Los nodos con estereotipo \texttt{\guillemotleft requirements\guillemotright} representan requisitos. Cada uno recoge: \textbf{Id}, \textbf{Text} (resumen), \textbf{Risk} (prioridad/impacto) y \textbf{Verification} (método de verificación: \emph{Test}, \emph{Inspection} o \emph{Analysis}).
  \item Los nodos con estereotipo \texttt{\guillemotleft element\guillemotright} representan elementos de diseño; en este proyecto corresponden a casos de uso (\emph{Type: usecase}).
\end{itemize}

\paragraph{Convenciones de relaciones}
\begin{itemize}
  \item \textbf{contains}: descomposición/agrupación. Un requisito de alto nivel agrupa a otros más concretos.
  \item \textbf{satisfies}: relación de cobertura. Un caso de uso implementa el comportamiento requerido por un RF.
  \item \textbf{traces}: relación de trazado. Un RNF condiciona a un caso de uso (p.\,ej., restricciones de seguridad, rendimiento o usabilidad) sin implicar implementación funcional directa.
\end{itemize}

\paragraph{Estructura de alto nivel}
\begin{itemize}
  \item \textbf{ROW1} y \textbf{ROW2} actúan como requisitos de organización, desde los que se descomponen y agrupan RF relacionados mediante \emph{contains}.
  \item Existen dependencias/refinamientos entre RF (flechas entre requisitos) que documentan precondiciones lógicas del flujo (p.\,ej., para clasificar es necesario antes capturar/filtrar).
\end{itemize}

\paragraph{Cobertura funcional (ejemplos destacados para entender los enlaces)}
\begin{itemize}
  \item \textbf{CU\_Iniciar} \emph{satisfies} \textbf{RF10} (inicio del sistema).
  \item \textbf{CU\_Parar} \emph{satisfies} \textbf{RF11} (parada controlada).
  \item \textbf{CU\_Dashboard} \emph{satisfies} \textbf{RF8} (panel/visualización).
  \item \textbf{CU\_Limpiar} \emph{satisfies} \textbf{RF9} (limpieza de datos/estado).
  \item \textbf{CU\_GenerarDataset} \emph{satisfies} \textbf{RF13} (generación de dataset).
  \item En la parte superior, \textbf{CU\_Clasificar}, \textbf{CU\_Capturar}, \textbf{CU\_Alerta} y \textbf{CU\_Filtro} cubren el flujo principal de tratamiento (detección, alertado, filtrado, etc.). En el clúster derecho, \textbf{CU\_Modelo} y \textbf{CU\_GenerarDataset} completan los RF de modelo y preparación de datos.
  \item Resultado: cada \textbf{RF} queda cubierto al menos por un \textbf{CU}; estos enlaces \emph{satisfies} permiten derivar casos de prueba de aceptación.
\end{itemize}

\paragraph{Trazabilidad de requisitos no funcionales (RNF)}
\begin{itemize}
  \item \textbf{RNF3} (usabilidad) \emph{traces} a \textbf{CU\_Dashboard}, donde se materializan decisiones de interfaz/UX.
  \item RNF transversales (p.\,ej., \emph{seguridad}, \emph{rendimiento}, \emph{calidad del modelo}) \emph{traces} a \textbf{CU\_Capturar}, \textbf{CU\_Clasificar}, \textbf{CU\_Modelo}, etc., dejando explícito dónde deben comprobarse dichas restricciones.
  \item La verificación de RNF se documenta en \emph{Verification} (\emph{Analysis}/\emph{Inspection}; \emph{Test} cuando procede, p.\,ej., tiempos de respuesta o controles de acceso).
\end{itemize}

\paragraph{Riesgo y priorización}
\begin{itemize}
  \item El campo \textbf{Risk} clasifica el impacto de no cumplimiento y guía la priorización de implementación y pruebas: los requisitos con riesgo \emph{High} se abordan antes y con mayor cobertura de validación.
\end{itemize}

\paragraph{Cómo leer el diagrama de requisitos y trazabilidad}
\begin{enumerate}
  \item Identificar, de izquierda a derecha, los grupos de RF bajo \textbf{ROW1}/\textbf{ROW2} para obtener el mapa de alcance.
  \item Para cada \textbf{RF}, seguir las flechas \emph{satisfies} hacia arriba para ver qué \textbf{CU} lo implementa (y dónde se prueba).
  \item Para cada \textbf{RNF}, seguir las flechas \emph{traces} para localizar los \textbf{CU} que deben cumplir la restricción.
  \item Consultar \textbf{Risk} y \textbf{Verification} para conocer la prioridad y el tipo de evidencia planificada en validación.
\end{enumerate}


Por lo tanto, el diagrama o figura proporciona \emph{trazabilidad bidireccional}: desde un requisito es posible localizar los elementos que lo implementan y, desde un caso de uso, los requisitos que satisface o que lo condicionan. Los campos \textbf{Risk} y \textbf{Verification} completan la planificación al vincular prioridad y evidencias de prueba.

\section{Diagrama de casos de uso}
Este apartado describe los casos de uso identificados para el Sistema IDS desarrollado. Cada caso de uso se alinea con uno o varios requisitos funcionales (RF) y no funcionales (RNF). Para la notación se emplea una plantilla homogénea con: Objetivo, Actores, Precondiciones, Postcondiciones, Flujo Principal, Flujos Alternativos / Excepciones, Requisitos Asociados, Datos y Frecuencia.

A continuación, se podrá visualización en la ilustración ~\ref{fig:diagCasosUso} el diagrama de casos de uso empleado.
\begin{figure}[ht!] \centering \includegraphics[ width=0.95\textwidth, trim=0mm 30mm 0mm 30mm]{
imagenes/diagramas/casos/casosClaro.pdf} \caption{Diagrama de casos de uso principal.} \label{fig:diagCasosUso} \end{figure}


\subsection{Listado de Casos de Uso}
\begin{itemize}
  \item CU\_Configurar Captura
  \item CU\_Iniciar Captura
  \item CU\_Capturar Tráfico
  \item CU\_Filtrar Captura
  \item CU\_Agregar Paquetes a Flujo
  \item CU\_Clasificar Flujo
  \item CU\_Generar Alerta
  \item CU\_Visualizar Dashboard
  \item CU\_Mostrar Detalles
  \item CU\_Limpiar Panel
  \item CU\_Parar Captura
  \item CU\_Generar Dataset (Exportar CSV/TXT)
  \item CU\_Gestionar Modelo ML
\end{itemize}

% =========================================================
\subsubsection{CU\_Configurar Captura}
\begin{description}
  \item[Objetivo:] Permitir al Administrador seleccionar interfaz y filtro inicial antes de iniciar la captura.
  \item[Actores:] Administrador (primario).
  \item[Precondiciones:] El sistema está operativo pero la captura no ha comenzado.
  \item[Postcondiciones:] Queda almacenada la configuración (interfaz, filtro BPF u otro) en el estado global.
  \item[Flujo Principal:]
    \begin{enumerate}
      \item El Administrador accede al panel de configuración.
      \item Selecciona la interfaz de red disponible.
      \item (Opcional) Introduce un filtro (por ejemplo ``tcp port 80'').
      \item El sistema valida sintaxis mínima (formato no vacío).
      \item Se actualiza el estado interno (State.set\_interface / set\_filter).
    \end{enumerate}
  \item[Flujos Alternativos / Excepciones:]
    \begin{enumerate}
      \item[A1] Interfaz inválida: se muestra mensaje y no se guarda.
      \item[A2] Filtro vacío: se acepta como “sin filtro”.
    \end{enumerate}
  \item[Requisitos Asociados:] RF5, RF12, RF10 (previo al inicio), RNF3.
  \item[Datos:] Nombre de interfaz, cadena de filtro.
  \item[Frecuencia:] Cada vez que se quiera cambiar la configuración antes o durante la operación (si se soporta cambio dinámico).
\end{description}

% =========================================================
\subsubsection{CU\_Iniciar Captura}
\begin{description}
  \item[Objetivo:] Comenzar la captura de paquetes en la interfaz configurada.
  \item[Actores:] Administrador.
  \item[Precondiciones:] Configuración válida establecida; capturador detenido.
  \item[Postcondiciones:] Hilo de captura (sniffer) en ejecución; worker activo.
  \item[Flujo Principal:]
    \begin{enumerate}
      \item El Administrador pulsa ``Iniciar''.
      \item El sistema crea instancia de FlowSession y reemplaza el writer por ConsoleWriter.
      \item Se lanza el hilo de sniffing (scapy AsyncSniffer o bucle sniff).
      \item Se lanza el worker (thread) de procesamiento de flujos.
      \item Se marca el estado capturing = True.
      \item La UI pasa a modo Online.
    \end{enumerate}
  \item[Flujos Alternativos:]
    \begin{enumerate}
      \item[A1] Error al crear FlowSession: se registra log y no se inicia (capturing permanece False).
      \item[A2] Interfaz sin tráfico: el sistema sigue esperando.
    \end{enumerate}
  \item[Requisitos Asociados:] RF10, RF1, RNF1, RNF5.
  \item[Datos:] Parámetros de inicialización (interfaz, filtro).
  \item[Frecuencia:] Varias veces a lo largo de la vida del sistema (tras paradas).
\end{description}

% =========================================================
\subsubsection{CU\_Capturar Tráfico}
\begin{description}
  \item[Objetivo:] Interceptar paquetes de red y asociarlos a flujos.
  \item[Actores:] Sniffer (sistema), Red / Atacante (fuentes externas).
  \item[Precondiciones:] Captura iniciada.
  \item[Postcondiciones:] Paquetes tratados y entregados a FlowSession.
  \item[Flujo Principal:]
    \begin{enumerate}
      \item El sniffer recibe un paquete.
      \item Aplica filtro (si existe).
      \item Invoca on\_packet\_received(packet).
      \item Determina clave de flujo (5-tuple + dirección).
    \end{enumerate}
  \item[Flujos Alternativos:]
    \begin{enumerate}
      \item[A1] Paquete no TCP ni UDP (si restricción activa) se descarta.
      \item[A2] Error parsing -> se ignora.
    \end{enumerate}
  \item[Requisitos Asociados:] RF1, RF5, RF12, RNF1.
  \item[Datos:] Paquetes en bruto (headers, tiempos).
  \item[Frecuencia:] Continuamente mientras haya tráfico.
\end{description}

% =========================================================
\subsubsection{CU\_Filtrar Captura}
\begin{description}
  \item[Objetivo:] Limitar el tráfico procesado según criterios (puerto, protocolo).
  \item[Actores:] Administrador.
  \item[Precondiciones:] Configuración aceptada; captura activa o próxima a iniciar.
  \item[Postcondiciones:] Nuevo filtro aplicado (en reinicio o inmediatamente si se soporta).
  \item[Flujo Principal:]
    \begin{enumerate}
      \item Administrador edita campo de filtro.
      \item Sistema almacena nuevo valor.
      \item En el siguiente reinicio de la captura, se usa el filtro.
    \end{enumerate}
  \item[Flujos Alternativos:]
    \begin{enumerate}
      \item[A1] Filtro sintácticamente inválido: se ignora y mantiene anterior.
    \end{enumerate}
  \item[Requisitos Asociados:] RF5, RNF3.
  \item[Datos:] Expresión BPF o similar.
  \item[Frecuencia:] Esporádica según necesidad.
\end{description}

% =========================================================
\subsubsection{CU\_Agregar Paquetes a Flujo}
\begin{description}
  \item[Objetivo:] Agregar cada paquete al flujo correspondiente y actualizar métricas.
  \item[Actores:] FlowSession, Flow.
  \item[Precondiciones:] Clave de flujo calculada; flujo existente o se creará uno nuevo.
  \item[Postcondiciones:] Estructura Flow actualizada; listas internas (paquetes, IAT, active/idle, bulk).
  \item[Flujo Principal:]
    \begin{enumerate}
      \item FlowSession localiza flujo existente o crea uno nuevo.
      \item Invoca flow.add\_packet(packet, direction).
      \item Actualiza timestamps, tamaños, flags, contadores.
      \item Evalúa condiciones de expiración (FIN, duración, inactividad).
    \end{enumerate}
  \item[Flujos Alternativos:]
    \begin{enumerate}
      \item[A1] Expiración superada -> se fuerza clean\_write\_flows y se abre un nuevo contador (count++).
    \end{enumerate}
  \item[Requisitos Asociados:] RF1, RNF1.
  \item[Datos:] Listas de paquetes y features incrementales.
  \item[Frecuencia:] Por cada paquete.
\end{description}

% =========================================================
\subsubsection{CU\_Clasificar Flujo}
\begin{description}
  \item[Objetivo:] Etiquetar cada flujo como Normal o Malicioso.
  \item[Actores:] Worker (primario), Modelo RF (externo lógico), Administrador (interesado).
  \item[Precondiciones:] Flujo finalizado (expiración, FIN o límite de paquetes); modelo cargado.
  \item[Postcondiciones:] Flujo con etiqueta y probabilidad en el estado global.
  \item[Flujo Principal:]
    \begin{enumerate}
      \item El writer encola datos del flujo.
      \item Worker consume de la cola.
      \item Mapea/ordena características al formato CIC.
      \item Escala / transforma según scaler guardado.
      \item Invoca predict().
      \item Recibe etiqueta y probabilidad y actualiza contadores.
    \end{enumerate}
  \item[Flujos Alternativos:]
    \begin{enumerate}
      \item[A1] Modelo no cargado: etiqueta \emph{Unknown}.
      \item[A2] Excepción de predicción: etiqueta \emph{Error}, probabilidad 0.
    \end{enumerate}
  \item[Requisitos Asociados:] RF1, RF2, RF7, RNF1, RNF5.
  \item[Datos:] Vector de características estandarizadas.
  \item[Frecuencia:] Cada flujo finalizado.
\end{description}

% =========================================================
\subsubsection{CU\_Generar Alerta}
\begin{description}
  \item[Objetivo:] Registrar una alerta cuando un flujo es malicioso.
  \item[Actores:] Worker / Estado, Administrador (visualiza).
  \item[Precondiciones:] Flujo clasificado como Malicious (etiqueta binaria).
  \item[Postcondiciones:] Alerta añadida a la lista de alertas.
  \item[Flujo Principal:]
    \begin{enumerate}
      \item Worker detecta predicción Malicious.
      \item Crea estructura de alerta (timestamp, IP origen/destino, severidad).
      \item Inserta en lista de alertas (máx. 50 manteniendo las más recientes).
      \item UI muestra contador actualizado.
    \end{enumerate}
  \item[Flujos Alternativos:]
    \begin{enumerate}
      \item[A1] Lista supera capacidad -> se descarta la más antigua.
    \end{enumerate}
  \item[Requisitos Asociados:] RF4, RF1, RNF3.
  \item[Datos:] IPs, probabilidad, severidad (actualmente simulada).
  \item[Frecuencia:] Cada flujo malicioso.
\end{description}

% =========================================================
\subsubsection{CU\_Visualizar Dashboard}
\begin{description}
  \item[Objetivo:] Presentar métricas, flujos y alertas en tiempo (casi) real.
  \item[Actores:] Administrador.
  \item[Precondiciones:] Sistema en ejecución (capturando o con datos cargados).
  \item[Postcondiciones:] Información visual actualizada cada intervalo (2 s).
  \item[Flujo Principal:]
    \begin{enumerate}
      \item La UI solicita refresco (timer).
      \item Estado devuelve totales y colecciones.
      \item Se renderizan tarjetas (flujos, alertas, normales, ataques).
    \end{enumerate}
  \item[Flujos Alternativos:]
    \begin{enumerate}
      \item[A1] Sin flujos todavía: se muestra mensaje de espera.
    \end{enumerate}
  \item[Requisitos Asociados:] RF3, RF8, RNF3.
  \item[Datos:] Listas en memoria (no persistidas).
  \item[Frecuencia:] Cada ciclo de refresco.
\end{description}

% =========================================================
\subsubsection{CU\_Mostrar Detalles}
\begin{description}
  \item[Objetivo:] Inspeccionar características detalladas de un flujo o alerta.
  \item[Actores:] Administrador.
  \item[Precondiciones:] Existen flujos y/o alertas en la UI.
  \item[Postcondiciones:] Modal desplegada con datos completos.
  \item[Flujo Principal:]
    \begin{enumerate}
      \item Administrador pulsa “Detalles”.
      \item Sistema localiza el objeto por ID.
      \item Copia información en selected\_flow / selected\_alert.
      \item Abre modal con métricas y flags.
    \end{enumerate}
  \item[Flujos Alternativos:]
    \begin{enumerate}
      \item[A1] ID no encontrado (eliminado por poda) -> se cancela.
    \end{enumerate}
  \item[Requisitos Asociados:] RF3, RF8.
  \item[Datos:] Campos completos del flujo (features) y la predicción.
  \item[Frecuencia:] A demanda.
\end{description}

% =========================================================
\subsubsection{CU\_Limpiar Panel}
\begin{description}
  \item[Objetivo:] Vaciar las listas de flujos y alertas para reinicio visual.
  \item[Actores:] Administrador.
  \item[Precondiciones:] Existen flujos/alertas en memoria.
  \item[Postcondiciones:] Listas vacías y contadores a cero.
  \item[Flujo Principal:]
    \begin{enumerate}
      \item Administrador pulsa “Limpiar”.
      \item Sistema asigna listas vacías y reinicia contadores.
      \item UI refleja estado vacío.
    \end{enumerate}
  \item[Flujos Alternativos:] No aplica.
  \item[Requisitos Asociados:] RF9.
  \item[Datos:] N/A (operación destructiva en memoria).
  \item[Frecuencia:] Esporádica.
\end{description}

% =========================================================
\subsubsection{CU\_Parar Captura}
\begin{description}
  \item[Objetivo:] Finalizar la captura y forzar escritura de flujos pendientes.
  \item[Actores:] Administrador.
  \item[Precondiciones:] Captura activa.
  \item[Postcondiciones:] Hilos detenidos; flujos procesados; estado Offline.
  \item[Flujo Principal:]
    \begin{enumerate}
      \item Administrador pulsa “Parar”.
      \item Sistema marca capturing = False.
      \item Lanza limpieza de flujos (clean\_write\_flows).
      \item Worker vacía cola pendiente.
      \item UI muestra estado Offline.
    \end{enumerate}
  \item[Flujos Alternativos:]
    \begin{enumerate}
      \item[A1] Error en limpieza: se registra log, se continúa la parada.
    \end{enumerate}
  \item[Requisitos Asociados:] RF11, RF1, RNF1.
  \item[Datos:] Flujos en cola final.
  \item[Frecuencia:] Cada ciclo de captura.
\end{description}

% =========================================================
\subsubsection{CU\_Generar Dataset}
\begin{description}
  \item[Objetivo:] Exportar los flujos capturados a formato CSV o TXT para entrenamiento futuro o análisis.
  \item[Actores:] Administrador.
  \item[Precondiciones:] Existen flujos capturados; el writer soporta el formato.
  \item[Postcondiciones:] Archivo generado en el sistema de ficheros.
  \item[Flujo Principal:]
    \begin{enumerate}
      \item Administrador solicita exportación (o esta ocurre al finalizar un lote).
      \item Writer serializa las características en filas.
      \item Se cierra (o sincroniza) el descriptor de archivo.
      \item Se notifica éxito (log/UI).
    \end{enumerate}
  \item[Flujos Alternativos:]
    \begin{enumerate}
      \item[A1] Error de escritura: se avisa, pudiendo reintentar.
    \end{enumerate}
  \item[Requisitos Asociados:] RF13, RF6 (si se considera histórico), RNF6.
  \item[Datos:] Todas las features por flujo (ver Anexo de Features).
  \item[Frecuencia:] Bajo demanda o al parar.
\end{description}

% =========================================================
\subsubsection{CU\_Gestionar Modelo ML}
\begin{description}
  \item[Objetivo:] Cargar, validar y (potencialmente) actualizar el modelo de Random Forest.
  \item[Actores:] Administrador (dispara entrenamiento externo), Sistema (carga al arrancar).
  \item[Precondiciones:] Ficheros del modelo (.pkl) disponibles.
  \item[Postcondiciones:] Modelo y scaler listos para inferencia (flag is\_loaded).
  \item[Flujo Principal:]
    \begin{enumerate}
      \item Al iniciar la aplicación se intenta load\_model().
      \item Se cargan modelo, scaler, mapping y lista de features.
      \item Se valida coherencia (número de features).
      \item Se expone is\_loaded=True.
    \end{enumerate}
  \item[Flujos Alternativos:]
    \begin{enumerate}
      \item[A1] Archivos no encontrados: is\_loaded=False, se muestra mensaje.
      \item[A2] Error de deserialización: se registra log, mantiene estado no cargado.
    \end{enumerate}
  \item[Requisitos Asociados:] RF7, RNF5, RNF6.
  \item[Datos:] Ficheros .pkl (modelo, scaler, mapping, features).
  \item[Frecuencia:] Al arranque; manual tras reentrenamiento.
\end{description}

% =========================================================
\subsection{Mapa Caso de Uso $\leftrightarrow$ Requisitos (Resumen)}
\begin{center}
\begin{tabular}{p{3.2cm} p{8.5cm}}
\textbf{Caso de Uso} & \textbf{Requisitos Principales}\\\hline
CU\_Configurar & RF5, RF12, RNF3 \\
CU\_Iniciar & RF10, RF1, RNF1 \\
CU\_Capturar & RF1, RF5, RF12, RNF1 \\
CU\_Filtrar & RF5, RNF3 \\
CU\_Agregar Paquetes & RF1, RNF1 \\
CU\_Clasificar & RF1, RF2, RF7, RNF1, RNF5 \\
CU\_Generar Alerta & RF4, RF1 \\
CU\_Visualizar Dashboard & RF3, RF8, RNF3 \\
CU\_Mostrar Detalles & RF3, RF8 \\
CU\_Limpiar Panel & RF9 \\
CU\_Parar Captura & RF11, RF1 \\
CU\_Generar Dataset & RF13, RF6 \\
CU\_Gestionar Modelo ML & RF7, RNF5, RNF6 \\
\end{tabular}
\end{center}

\bigskip
Con esto queda descrita la interacción funcional del sistema.
El desarrollo de un sistema de detección de intrusiones basado en Machine Learning se apoya en un ecosistema de tecnologías y herramientas existentes que facilitan la captura de datos, su procesamiento, el entrenamiento de modelos y la visualización de resultados.


\section{Gramática del caso de uso}

Una gramática de un caso~\cite{cockburn2000writing} de uso pretende dar una descripción de los componentes, acciones y reacciones del caso de uso, recogiendo los posibles escenarios que pueden producirse. Se estructura con: nombre del caso de uso, protagonista (actor primario), sistema, participantes (conjunto de actores o componentes), nivel del caso de uso (objetivo de usuario o subfunción), precondiciones, operaciones básicas del escenario principal y alternativas/excepciones.

A continuación se presentan dos gramáticas representativas: (i) un caso de objetivo de usuario (CU\_Iniciar Captura) en la tabla ~\ref{tab:gramatica_iniciar}, y (ii) un caso de subfunción interna (CU\_Clasificar Flujo) en la tabla ~\ref{tab:gramatica_clasificar}.

\begin{table}[H]
\centering
\small
\renewcommand{\arraystretch}{1.15}
\begin{tabular}{p{4.0cm} p{10.5cm}}
\hline
\textbf{Caso de uso} & CU\_Iniciar Captura \\
\textbf{Protagonista} & Administrador (usuario) \\
\textbf{Sistema} & IDS (aplicación web + capturador) \\
\textbf{Participantes} & UI (Reflex), Estado, Sniffer (Scapy), FlowSession, Worker \\
\textbf{Nivel} & Objetivo de usuario \\
\textbf{Precondición} & Configuración válida (interfaz y, opcionalmente, filtro BPF) y captura detenida. \\
\hline
\multicolumn{2}{l}{\textbf{Operaciones básicas}}\\
1 & Pulsar el botón «Iniciar» en la UI. \\
2 & Crear \textit{FlowSession} y preparar el escritor (ConsoleWriter para la UI). \\
3 & Lanzar el hilo de sniffing (Scapy) con la interfaz y filtro configurados. \\
4 & Lanzar el \textit{worker} de procesamiento de flujos. \\
5 & Marcar \texttt{capturing = True} y poner la UI en estado «Online». \\
6 & Comenzar el refresco periódico de la información (cada 2 s). \\
\hline
\multicolumn{2}{l}{\textbf{Operaciones alternativas / Excepciones}}\\
1.A & ¿Configuración válida? \\
1.A.1 & Si sí, continuar con el paso 2. \\
1.A.2 & Si no, mostrar mensaje de error y no iniciar (permanecer detenido). \\
3.A & ¿Interfaz o permisos de captura incorrectos? \\
3.A.1 & Si sí, registrar log y abortar el inicio (seguir Offline). \\
3.A.2 & Si no, continuar. \\
4.A & ¿Fallo al arrancar el \textit{worker}? \\
4.A.1 & Si sí, registrar log y abortar (deshacer recursos del sniffer). \\
4.A.2 & Si no, continuar. \\
5.A & ¿No llega tráfico? El sistema permanece a la espera (sin errores). \\
\hline
\end{tabular}
\caption{Gramática del caso de uso CU\_Iniciar Captura (objetivo de usuario).}
\label{tab:gramatica_iniciar}
\end{table}

\paragraph{Explicación}
Este caso de uso representa la acción principal del usuario para poner el sistema en marcha. La responsabilidad de crear y coordinar los hilos (sniffer y \textit{worker}) recae en la aplicación, y el éxito se refleja en el cambio de estado a «Online» y el inicio del ciclo de refresco.

\begin{table}[H]
\centering
\small
\renewcommand{\arraystretch}{1.15}
\begin{tabular}{p{4.0cm} p{10.5cm}}
\hline
\textbf{Caso de uso} & CU\_Clasificar Flujo \\
\textbf{Protagonista} & Sistema (Worker) \\
\textbf{Sistema} & IDS (componente de inferencia) \\
\textbf{Participantes} & Writer/Cola de flujos, CICIDSPredictor (Scaler + Modelo RF), Estado/UI \\
\textbf{Nivel} & Subfunción del sistema \\
\textbf{Precondición} & Existe un flujo finalizado en cola; artefactos cargados (modelo, \textit{scaler}, \textit{mapping}). \\
\hline
\multicolumn{2}{l}{\textbf{Operaciones básicas}}\\
1 & El \textit{writer} entrega (encola) los datos del flujo. \\
2 & El \textit{worker} consume el flujo desde la cola. \\
3 & Mapear/renombrar y completar características al formato CIC (duplicados incluidos). \\
4 & Sustituir NaN/inf por 0 y aplicar \texttt{StandardScaler}. \\
5 & Invocar \texttt{predict}/\texttt{predict\_proba} (Random Forest). \\
6 & Registrar en el estado la etiqueta y la probabilidad; actualizar contadores. \\
7 & Si la etiqueta es «Malicious», solicitar la generación de alerta. \\
\hline
\multicolumn{2}{l}{\textbf{Operaciones alternativas / Excepciones}}\\
1.A & ¿Hay flujo listo en cola? \\
1.A.1 & Si sí, continuar con el paso 2. \\
1.A.2 & Si no, esperar (bloqueo con \textit{timeout}) y reintentar. \\
3.A & ¿Faltan columnas requeridas por CIC? \\
3.A.1 & Si sí, imputar/corregir (duplicados: \texttt{Fwd Seg Size Avg} $\leftarrow$ \texttt{Fwd Pkt Len Mean}, etc.). \\
3.A.2 & Si no se puede corregir, descartar el flujo y registrar log. \\
4.A & ¿Scaler no cargado o inconsistencia de dimensiones? \\
4.A.1 & Intentar recarga; si falla, etiquetar «Unknown» y probabilidad 0. \\
5.A & ¿Excepción en predicción? \\
5.A.1 & Etiquetar «Error», probabilidad 0 y registrar log. \\
7.A & ¿Etiqueta = «Malicious»? \\
7.A.1 & Si sí, crear alerta con severidad según probabilidad y notificar a la UI. \\
7.A.2 & Si no, finalizar sin alerta. \\
\hline
\end{tabular}
\caption{Gramática del caso de uso CU\_Clasificar Flujo (subfunción del sistema).}
\label{tab:gramatica_clasificar}
\end{table}

\paragraph{Explicación}
Este caso de uso detalla el ciclo de inferencia del IDS. La clave es mantener la compatibilidad estricta con el formato CIC (nombres y orden de columnas) y el uso del \texttt{StandardScaler} entrenado. Los fallos se gestionan con etiquetas «Unknown»/«Error» y registro de eventos, y las detecciones maliciosas derivan en la creación de alertas visibles en el \textit{dashboard}.

\section{Diagrama de actividades}\label{Sec.DiagActividades}
\subsection{Exportación del dataset}

\begin{figure}[H] \centering \includegraphics[ height=0.85\textheight, trim=0mm 0mm 100mm 0mm, clip ]{imagenes/diagramas/actividades/activityexportdataset/aed.pdf} \caption{Diagrama de actividades de exportación del dataset.} \label{fig:diagActED} \end{figure}

\textbf{Propósito.} Este proceso, asociado a \textbf{CU\_GenerarDataset} y \textbf{RF13}, permite extraer un conjunto de datos desde los flujos mantenidos por el IDS para su análisis externo o entrenamiento de modelos. El principal objetivo de este proceso es obtener un conjunto de datos que pueda ser utilizado en un futuro como una fuente de base de datos de recopilación de ataques modernos.

En la ilustración ~\ref{fig:diagActED} se puede visualizar el diagrama de exportación del dataset.

\textbf{Descripción del flujo.}
\begin{enumerate}
  \item \textbf{Solicitar exportación.} El operador dispara la acción desde la interfaz, un endpoint de administración o desde la terminal de comandos. Se registran parámetros de exportación (rango temporal, esquema, formato CSV/TXT, destino).
  \item \textbf{¿Captura activa?} Se comprueba si el proceso de captura está en ejecución.
  \begin{itemize}
    \item \emph{Sí:} Se ofrece la \emph{opción de parar temporalmente} la captura o tomar un \emph{snapshot} consistente (congelar la vista de la tabla de flujos) para evitar condiciones de carrera durante la lectura.
    \item \emph{No:} Se continúa directamente.
  \end{itemize}
  \item \textbf{Tomar flujos en memoria.} Se obtiene el conjunto de flujos a exportar a partir de la estructura de agregación (p.\,ej., tabla hash por 5-tuple) o del buffer persistente, aplicando los filtros solicitados.
  \item \textbf{Recorrer flujos.} Iteración secuencial o en \emph{batches} sobre los flujos. En este paso se completan/derivan campos (duración, bytes/paquetes, timestamps normalizados) y se valida el esquema de salida.
  \item \textbf{Serializar a CSV/TXT.} Conversión de cada flujo a un registro plano con cabecera y orden de columnas definido (alineado con el \emph{feature map} usado por el modelo). Se recomienda escritura atómica: archivo temporal + renombrado final.
  \item \textbf{¿Error de escritura?} Se controla E/S (permisos, espacio, locking).
  \begin{itemize}
    \item \emph{Sí:} Se registra el error y se aplica política de \emph{reintento} con \emph{backoff}, hasta un máximo de intentos. Si se agota, se marca la exportación como \emph{fallida}.
    \item \emph{No:} Se confirma la generación (ruta, tamaño, número de registros, checksum).
  \end{itemize}
  \item \textbf{Fin de exportación.} Se notifica al usuario y se dejan métricas (tiempo, throughput, errores).
\end{enumerate}

\textbf{Entradas y salidas.}
\begin{itemize}
  \item Entradas: tabla/buffer de flujos, parámetros de exportación, esquema de columnas.
  \item Salidas: archivo(s) CSV/TXT, registro de auditoría y métricas de proceso.
\end{itemize}

\textbf{RNF y consideraciones.}
\begin{itemize}
  \item \emph{Consistencia}: snapshot o pausa breve si la captura está activa, evitando truncados/inconsistencias.
  \item \emph{Reproducibilidad}: incluir cabecera, versión de esquema y zona horaria; fijar formato de timestamp.
  \item \emph{Rendimiento}: escritura en \emph{buffer} y compresión opcional si el volumen es alto.
\end{itemize}

\textbf{Verificación.} Prueba de conteos (flujos seleccionados $\approx$ filas exportadas), validación de esquema/encabezados, comprobación de integridad (checksum) y prueba de reintento ante disco lleno/permisos.

\textbf{Trazabilidad.} Satisface \textbf{RF13}. Condiciona a \textbf{CU\_Modelo} (entrenamiento) al proporcionar datasets consistentes.

\subsection{Captura y clasificación}

\begin{figure}[H] \centering \includegraphics[ height=0.90\textheight, trim=0mm 0mm 100mm 0mm, clip ]{imagenes/diagramas/actividades/activitycaptureclassify/acc.pdf} \caption{Diagrama de actividades de captura y clasificación.} \label{fig:diagActCyC} \end{figure}

\textbf{Propósito.} Este flujo implementa el \emph{pipeline} en línea de detección, cubriendo \textbf{CU\_Capturar} (\textbf{RF2}), \textbf{CU\_Filtro} (\textbf{RF5}), \textbf{CU\_Clasificar} (\textbf{RF1}) y la generación de alertas \textbf{CU\_Alerta} (\textbf{RF4}), con persistencia \textbf{RF6} y actualización del panel \textbf{RF8}.

En la ilustración ~\ref{fig:diagActCyC} se puede visualizar el diagrama de captura y clasificación.
\textbf{Descripción del flujo.}
\begin{enumerate}
  \item \textbf{Iniciar captura.} El sistema configura el \emph{sniffer} (interfaz, BPF) y pasa a estado operativo.
  \item \textbf{Recibir paquete.} Cada paquete entrante se procesa en tiempo real.
  \item \textbf{¿Existe flujo?} Búsqueda en la tabla de flujos por 4-tupla.
  \begin{itemize}
    \item \emph{No:} \textbf{Crear flujo}. Se inicializa estructura con contadores, timestamps y contexto.
    \item \emph{Sí:} \textbf{Actualizar flujo}. Se actualizan métricas (bytes, paquetes, flags, tiempos).
  \end{itemize}
  \item \textbf{¿Fin/expira/máx?} Se decide si el flujo puede cerrarse por: (i) FIN/RST observado, (ii) \emph{timeout} de inactividad, (iii) tamaño/duración máximos alcanzados.
  \begin{itemize}
    \item \emph{No:} Se vuelve a \textbf{Recibir paquete} (flujo continúa).
    \item \emph{Sí:} El flujo se \emph{cierra} y pasa a etapa de \emph{feature engineering}.
  \end{itemize}
  \item \textbf{Calcular features.} Extracción de características a nivel de flujo (duración, tasas, variabilidad inter-arrival, flags, etc.), alineadas con el \emph{feature map} del modelo.
  \item \textbf{Encolar flujo.} Se introduce en una cola que desacopla captura (productor) de inferencia (consumidor), protegiendo la latencia de captura.
  \item \textbf{Worker extrae.} Un proceso/ hilo consumidor toma el flujo de la cola y continúa el pipeline.
  \item \textbf{Mapear y escalar.} Se reordenan columnas al esquema del modelo y se aplica el \emph{scaler} persistido (media/desv. estándar, min–max, etc.).
  \item \textbf{Modelo RF: predict.} Inferencia con el modelo de \emph{Random Forest} activo, obteniendo probabilidad/clase e incorporando metadatos (versión de modelo, hash de \emph{feature map}).
  \item \textbf{¿Malicioso?} Decisión por umbral.
  \begin{itemize}
    \item \emph{Sí:} \textbf{Generar alerta}. Se construye alerta con contexto; se aplican deduplicación y \emph{rate limiting}.
    \item \emph{No:} \textbf{Marcar normal}. Se registra clasificación para métricas y análisis.
  \end{itemize}
  \item \textbf{Actualizar \emph{dashboard}.} Se reflejan KPIs (tasa, colas, ratio de alertas) y eventos recientes.
  \item \textbf{¿Captura activa?} Si el operador detiene la captura, el bucle finaliza; en caso contrario, el control retorna a \textbf{Recibir paquete}.
\end{enumerate}

\textbf{Entradas y salidas.}
\begin{itemize}
  \item Entradas: paquetes de red, configuración de captura, modelo y \emph{scaler} activos.
  \item Salidas: etiquetas por flujo, alertas, métricas operativas y registros para auditoría.
\end{itemize}

\textbf{Concurrencia y resiliencia.}
\begin{itemize}
  \item \emph{Desacoplo productor–consumidor}: la cola evita que la inferencia afecte a la captura. Política ante \emph{backpressure}: aumentar tamaño, \emph{drop} controlado o \emph{spill} a disco.
  \item \emph{Tiempo de vida del flujo}: la decisión fin/expira/máx impide \emph{flows} eternos y controla memoria.
  \item \emph{Trazabilidad}: cada predicción conserva versión del modelo y del \emph{feature map}, facilitando auditorías y análisis de deriva.
\end{itemize}

\textbf{RNF y seguridad.}
\begin{itemize}
  \item \emph{Rendimiento}: latencia de inferencia dentro del presupuesto; captura sin pérdidas (monitor de drops).
  \item \emph{Usabilidad}: el panel refleja estado en tiempo real (RNF de dashboard).
  \item \emph{Seguridad}: protección de artefactos de modelo, sanitización de datos en alertas.
\end{itemize}

\textbf{Verificación.} Pruebas con \emph{pcaps} de referencia; validación de colas (sin bloqueos), exactitud de predicción con \emph{golden set}, y pruebas de deduplicación/umbralado de alertas.

\textbf{Trazabilidad.} Satisface \textbf{RF2}, \textbf{RF5}, \textbf{RF1}, \textbf{RF4}, \textbf{RF6} y \textbf{RF8}; condicionado por RNF de rendimiento y disponibilidad.

\subsection{Entrenamiento}


\begin{figure}[H] \centering \includegraphics[ height=0.90\textheight, trim=0mm 0mm 90mm 0mm, clip ]{imagenes/diagramas/actividades/activitytraining/at.pdf} \caption{Diagrama de actividades de entrenamiento.} \label{fig:diagActE} \end{figure}

\textbf{Propósito.} Este proceso, vinculado a \textbf{CU\_Modelo} (\textbf{RF7}) y alimentado por \textbf{CU\_GenerarDataset} (\textbf{RF13}), genera y valida el modelo de clasificación empleado en producción.

En la ilustración ~\ref{fig:diagActE} se puede visualizar el diagrama de exportación del dataset.

\textbf{Descripción del flujo.}
\begin{enumerate}
  \item \textbf{Iniciar entrenamiento.} Se parametriza la tarea (fuente del dataset, \emph{seed} de aleatoriedad, configuración de evaluación).
  \item \textbf{Cargar CSV.} Lectura del dataset exportado (CSV) con validación de esquema y tipos.
  \item \textbf{Preprocesar/limpiar.} Gestión de nulos, outliers, categorías desconocidas; normalización de formatos.
  \item \textbf{Mapear features CIC.} Alineación de columnas con el conjunto de \emph{features} esperado (p.\,ej., mapeo compatible con \emph{CICFlowMeter}/datasets CIC).
  \item \textbf{Convertir protocol/timestamp.} Codificación de \emph{protocol} y normalización de tiempos (zona horaria, precisión) para consistencia entre entrenamiento e inferencia.
  \item \textbf{\emph{Split} train–test.} División estratificada para preservar el balance de clases; posibilidad de conjunto de validación separado o \emph{cross-validation}.
  \item \textbf{Escalar features.} Cálculo de parámetros del \emph{scaler} (p.\,ej., StandardScaler) sobre el entrenamiento y aplicación consistente a train/test.
  \item \textbf{Entrenar RandomForest.} Ajuste del clasificador principal (número de árboles, profundidad, \emph{max features}, criterio, etc.) con la semilla fijada para reproducibilidad.
  \item \textbf{Evaluar métricas.} Cálculo de F1, precisión/recobrado, AUC y, cuando procede, MCC, de acuerdo con los criterios de la sección de evaluación del TFG. Se consideran tanto métricas globales como por clase (normal/malicioso).
  \item \textbf{¿Aceptable?} Comparación con umbrales definidos en la memoria.
  \begin{itemize}
    \item \emph{Sí:} \textbf{Guardar modelo/scaler/mapping}. Se persisten artefactos (modelo serializado, parámetros del \emph{scaler}, orden y nombres de columnas), junto con metadatos (versión, fecha, \emph{seed}, métricas).
    \item \emph{No:} \textbf{Ajustar hiperparámetros}. Se explora el espacio de hiperparámetros (p.\,ej., rejilla o búsqueda aleatoria) y se itera desde el escalado, manteniendo controles contra sobreajuste (validación cruzada, \emph{hold-out}).
  \end{itemize}
  \item \textbf{Fin.} Publicación de un informe de entrenamiento y registro de métricas base para detectar deriva futura.
\end{enumerate}

\textbf{Entradas y salidas.}
\begin{itemize}
  \item Entradas: dataset CSV exportado, configuración de entrenamiento y evaluación.
  \item Salidas: artefactos versionados (modelo RF, \emph{scaler}, \emph{feature map}) y reporte de métricas.
\end{itemize}

\textbf{RNF y buenas prácticas.}
\begin{itemize}
  \item \emph{Reproducibilidad}: fijar \emph{seed}; guardar versiones de datos/esquema; registrar librerías.
  \item \emph{Equilibrio de clases}: técnicas de balance si procede (p.\,ej., \emph{class\_weight}).
  \item \emph{Trazabilidad}: enlazar artefactos con commit/configuración; facilitar \emph{rollback}.
\end{itemize}

\textbf{Verificación.} Validación de consistencia entre entrenamiento e inferencia (mismo \emph{feature map}/\emph{scaler}); repetibilidad del resultado con igual \emph{seed}; métricas por encima de umbrales definidos.

\textbf{Trazabilidad.} Satisface \textbf{RF7} y depende de \textbf{RF13}. Sus resultados condicionan \textbf{RF1} y \textbf{RF4} en producción.

\section{Diagrama de flujo}
Un \textbf{diagrama de flujo} modela la secuencia de pasos de un proceso mediante símbolos estandarizados. En nuestra representación utilizamos la notación tradicional de flujogramas\cite{iso5807}) y, como referencia vigente para modelado de procesos, BPMN 2.0~\cite{iso19510bpmn}. En este trabajo mantenemos una notación mínima orientada a claridad operativa:
\begin{itemize}
  \item Rectángulos: actividades o procesos (p.\,ej., \emph{Captura}, \emph{Construir}).
  \item Conectores/flechas: dirección del flujo de control o de datos.
  \item Cilindros: almacenes o \emph{buffers} (en este contexto, una \textbf{cola} que desacopla productores y consumidores).
  \item (Opcional) Diamantes: decisiones. En este flujograma concreto no se representan bifurcaciones condicionales.
\end{itemize}
A diferencia de los diagramas de \emph{actividades} o de \emph{secuencia}, el flujograma busca la \emph{claridad operativa} del proceso por encima del detalle temporal o de la interacción entre objetos.

En la siguiente ilustración ~\ref{fig:diagFlujo} se puede observar el diagrama de flujo.

\subsection{Descripción del diagrama de flujo}
La Figura~\ref{fig:diagFlujo} resume el recorrido de los datos desde su origen hasta su consumo por los módulos de inferencia y visualización.
\begin{enumerate}
  \item \textbf{Tráfico.} Entrada de paquetes procedentes de la red objetivo. Constituye la fuente de datos primaria del IDS.
  \item \textbf{Captura.} Un \emph{sniffer} ingiere los paquetes desde una interfaz y aplica, si procede, filtros BPF. La salida son eventos a nivel de paquete.
  \item \textbf{FlowSession.} Capa de \emph{sessionización} que agrega paquetes en \emph{flows} (p.\,ej., por 5--tupla), manteniendo contadores, marcas temporales y estado del flujo.
  \item \textbf{Construir.} Transformación del \emph{flow} en un registro estructurado: cálculo de características (duración, tasas, flags, inter-arrival), normalización de tipos y orden de columnas conforme al \emph{feature map}.
  \item \textbf{Cola.} \emph{Buffer} productor--consumidor que desacopla la etapa de captura/transformación del procesamiento posterior. Amortigua picos de carga y protege la latencia de captura.
  \item \textbf{Worker.} Proceso consumidor que:
  \begin{itemize}
    \item aplica el mapeo y el \emph{scaler} persistidos,
    \item invoca al \textbf{Modelo} para obtener predicción,
    \item actualiza el \textbf{Estado} del sistema y emite resultados.
  \end{itemize}
  \item \textbf{Modelo.} Clasificador (p.\,ej., \emph{Random Forest}) que devuelve etiqueta y probabilidad para cada \emph{flow}. Sus artefactos (modelo, \emph{scaler}, \emph{feature map}) están versionados.
  \item \textbf{Estado.} Módulo que consolida métricas operativas (tasas, colas, \% de alertas) y orquesta acciones derivadas (p.\,ej., \emph{rate limiting} de alertas, retención de datos).
  \item \textbf{Dashboard.} Interfaz que presenta en tiempo real el estado del pipeline y los eventos/alertas al operador.
  \item \textbf{Export CSV.} Ruta de salida para la \emph{exportación de dataset} (CSV/TXT) destinada a análisis o entrenamiento offline.
\end{enumerate}

\paragraph{Aspectos clave}
\begin{itemize}
  \item \textbf{Desacoplo y robustez.} La \textbf{cola} separa captura e inferencia, evitando bloqueos y permitiendo políticas de \emph{backpressure} (tamaño, \emph{drop} controlado o volcado a disco).
  \item \textbf{Consistencia de \emph{features}.} El bloque \emph{Construir} y el \emph{Worker} garantizan que el mismo \emph{feature map}/\emph{scaler} se use en entrenamiento e inferencia.
  \item \textbf{Observabilidad.} \emph{Estado} y \emph{Dashboard} exponen métricas y logs necesarios para operar y auditar el sistema.
  \item \textbf{Trazabilidad.} Cada registro procesado debe quedar asociado a la versión del modelo y del \emph{feature map} empleados.
\end{itemize}

\paragraph{Trazabilidad con requisitos y casos de uso}
El flujograma cubre: \textbf{CU\_Capturar} (RF2), \textbf{CU\_Filtro} (RF5), \textbf{CU\_Clasificar} (RF1), \textbf{CU\_Alerta} (RF4), \textbf{CU\_Dashboard} (RF8) y \textbf{CU\_GenerarDataset} (RF13). Sus elementos se desarrollan con mayor detalle en los diagramas de actividades (Sección~\ref{Sec.DiagActividades}) y de secuencia (Sección~\ref{Sec.DiagSecuencia}).

\begin{figure}[H]
  \centering
  \includegraphics[height=0.90\textheight, trim = 0mm 0mm 36mm 0mm, clip]{imagenes/diagramas/flujo/diagramaFlujocropped.pdf}
  \caption{Diagrama de flujo general del pipeline del IDS.}
  \label{fig:diagFlujo}
\end{figure}

\section{Diagrama de secuencia}\label{Sec.DiagSecuencia}
Un diagrama de secuencia muestra la interacción entre objetos organizados en
una secuencia de tiempo~\cite{omgUML251}. Puede dibujarse con distintos niveles de detalle y para satisfacer distintos objetivos en las diversas etapas del ciclo de vida del desarrollo del sistema. Normalmente se utiliza para representar la interacción entre objetos que se produce en un caso de uso o para una operación. Cuando se utiliza para modelar el comportamiento dinámico de un caso de uso puede considerarse como una especificación detallada del caso de uso.

La dimensión vertical representa el tiempo. Los objetos involucrados en la
interacción están distribuidos horizontalmente en el diagrama. Cada objeto está representado por una línea de vida, que se representa por una línea vertical discontinua y con un símbolo de objeto en su parte superior. Un mensaje se representa por una flecha horizontal que va desde una línea de vida a otra.

Cuando se envía un mensaje a un objeto se está llamando a una operación de
ese objeto. El nombre del mensaje generalmente coincide con el de la operación que se está llamando. El periodo de tiempo durante el que se está ejecutando una operación se conoce como una ocurrencia de ejecución o de activación. Dicho periodo de tiempo se representa mediante un bloque rectangular situado a lo largo de la línea de vida.
\subsection{Principal}

\begin{landscape} \begin{figure}[p] \centering \includegraphics[height=1\textheight, trim = 0mm 82mm 0mm 82mm ,clip]{imagenes/diagramas/secuencia/principal/mainSecuence.pdf} \caption{Diagrama de secuencia principal.} \label{fig:seqPrincipal} \end{figure} \end{landscape}

\paragraph{Propósito}
Modelar el flujo en línea desde que el operador inicia la captura hasta la clasificación de flujos y la actualización periódica del panel.

\paragraph{Participantes}
\begin{description}
  \item[Admin:] usuario que desencadena la acción.
  \item[UI (Reflex):] interfaz que orquesta las órdenes del usuario y refresca métricas.
  \item[State:] componente que gestiona el estado global del IDS y el ciclo de vida de hilos.
  \item[Sniffer:] proceso/hilo de captura que recibe paquetes.
  \item[Session:] gestor de sesiones/tabla de \emph{flows} (hash por 4--tupla).
  \item[Flow:] entidad que acumula métricas por flujo y decide su cierre.
  \item[Writer:] serializa el flujo cerrado a un registro estructurado.
  \item[Cola:] \emph{buffer} productor--consumidor entre escritura y clasificación.
  \item[Worker:] consumidor que prepara \emph{features} y pide predicciones.
  \item[Model:] clasificador (RandomForest) que devuelve \emph{label} y probabilidad.
\end{description}

A continuación, pasaremos a explicar el flujo de la secuencia principal mostrado en la Figura~\ref{fig:seqPrincipal}.
\paragraph{Secuencia principal}
\begin{enumerate}
  \item \textbf{Admin} hace \emph{Clic Iniciar}. La \textbf{UI} invoca \texttt{start\_capture()} sobre \textbf{State} (CU\_Iniciar, RF10).
  \item \textbf{State} lanza el hilo de captura y activa el \textbf{Sniffer}.
  \item Ante cada paquete, \textbf{Sniffer} dispara \texttt{on\_packet\_received(pkt)} hacia \textbf{Session} (callback asíncrona).
  \item \textbf{Session} localiza o crea el \textbf{Flow} correspondiente y llama a \texttt{add\_packet(pkt)} sobre \textbf{Flow}; este actualiza métricas (contadores, tiempos, flags).
  \item \textbf{alt} [Flujo expira/FIN]. Si se observa FIN/RST, \emph{timeout} o límite de tamaño/duración:
    \begin{enumerate}
      \item \textbf{Flow} calcula y empaqueta \texttt{flow\_data} y lo envía a \textbf{Writer.write(\textit{flow\_data})}.
      \item \textbf{Writer} transforma a registro (orden de columnas consistente con el \emph{feature map}) y hace \textbf{Cola.put(\textit{flow\_data})} (desacoplando captura y clasificación).
      \item \textbf{Worker} realiza \textbf{Cola.get()}, mapea/escala \emph{features} según artefactos persistidos y solicita \textbf{Model.predict(\textit{features})}.
      \item \textbf{Model} devuelve \texttt{label, prob} al \textbf{Worker}; el resultado se persiste y/o se encola para alertado. Este retorno está representado en la figura como respuesta a \texttt{predict(features)}.
    \end{enumerate}
  \item En paralelo, la \textbf{UI} realiza un \emph{refresco periódico} de contadores/estado de \emph{pipeline} (``Actualiza contadores'') mostrando tasas de captura, número de \emph{flows} cerrados y alertas (CU\_Dashboard, RF8).
\end{enumerate}

\paragraph{Consideraciones técnicas}
\begin{itemize}
  \item \textbf{Concurrencia:} el hilo de captura produce registros y el \textbf{Worker} los consume. La \textbf{Cola} absorbe picos (\emph{backpressure}). Políticas: tamaño máximo, \emph{drop} controlado o volcado a disco.
  \item \textbf{Consistencia:} el cierre de flujo en \textbf{Flow} garantiza que las métricas estén completas antes de serializar.
  \item \textbf{Trazabilidad:} \textbf{Writer} y \textbf{Worker} deben anotar versión de modelo, \emph{scaler} y \emph{feature map} para auditoría.
  \item \textbf{Seguridad:} sanitizar datos sensibles en rutas de log/alertas.
\end{itemize}

\paragraph{Requisitos no funcionales (RNF)}
Latencia de inferencia dentro del presupuesto para no bloquear la captura (RNF de rendimiento); disponibilidad (reintentos sobre la \textbf{Cola} o el \textbf{Model}); observabilidad (métricas y logs).

\paragraph{Verificación}
Pruebas con \emph{pcaps} de referencia, tests de integridad de la \textbf{Cola}, comprobación de exactitud con \emph{golden set}, y validación del refresco de la \textbf{UI}.

\subsection{Expiración y cierre de flujos}

\paragraph{Propósito}
Describir el apagado ordenado cuando el operador detiene la captura, asegurando que todos los \emph{flows} pendientes se cierran y persisten correctamente.

\paragraph{Participantes}
\begin{description}
  \item[Admin y UI:] desencadenan la parada y muestran confirmación.
  \item[State:] transiciona el sistema a estado \emph{Offline} y coordina limpieza.
  \item[Session:] recorre la tabla de \emph{flows} para cerrar los que queden abiertos.
  \item[Writer y Archivo:] serializan y adjuntan los registros finales a almacenamiento.
\end{description}

\begin{figure}[ht!] \centering \includegraphics[ height=0.50\textheight, trim = 0mm 0mm 0mm 0mm, clip]{imagenes/diagramas/secuencia/stopExport/stopExport.pdf} \caption{Diagrama de secuencia de expiración y cierre de flujos.} \label{fig:SeqECF} \end{figure}

A continuación, pasaremos a explicar el flujo de la secuencia principal mostrado en la Figura~\ref{fig:SeqECF}.
\paragraph{Secuencia principal}
\begin{enumerate}
  \item \textbf{Admin} ejecuta \emph{Clic Parar}. La \textbf{UI} invoca \texttt{stop\_capture()} sobre \textbf{State} (CU\_Parar, RF11).
  \item \textbf{State} entra en proceso de parada y llama a \texttt{clean\_write\_flows()} sobre \textbf{Session} para parada/controlado de \emph{flows}.
  \item \textbf{Session} itera (\textbf{loop} [Flujos pendientes]) sobre la estructura de \emph{flows} abiertos; para cada uno:
    \begin{enumerate}
      \item compone el \texttt{data} final del flujo y llama a \textbf{Writer.write(\textit{data})};
      \item \textbf{Writer} realiza \textbf{Archivo.append(\textit{linea})}, preferiblemente con escritura atómica y \emph{buffers}.
    \end{enumerate}
  \item Tras vaciar la cola y completar la escritura, \textbf{State} marca \emph{Estado Offline} y comunica confirmación a la \textbf{UI}, que la muestra a \textbf{Admin}.
\end{enumerate}

\paragraph{Consideraciones técnicas}
\begin{itemize}
  \item \textbf{Idempotencia:} \texttt{clean\_write\_flows()} debe tolerar invocaciones repetidas sin duplicar registros (clave natural por \textit{flow-id} + \emph{timestamps}).
  \item \textbf{Integridad:} evitar pérdidas de datos asegurando \texttt{flush()} a disco y cierre controlado de descriptores.
  \item \textbf{Orden:} escribir primero \emph{flows} más antiguos o por prioridad para acotar tiempo de parada.
  \item \textbf{Telemetría:} contabilizar \emph{flows} drenados, tiempo total y posibles errores de E/S para diagnóstico.
\end{itemize}

\paragraph{Requisitos no funcionales (RNF)}
Disponibilidad (parada en tiempo acotado), confiabilidad (no perder \emph{flows}), trazabilidad (log de cierre).

\paragraph{Verificación}
Pruebas de parada con colas llenas, simulación de fallo de escritura (reintento y \emph{fallback}), y comprobación de que la \textbf{UI} no confirma hasta finalizar la limpieza.
\clearpage\thispagestyle{empty}\cleardoublepage

\chapter{DISEÑO}
\section{Arquitectura del sistema}

En esta sección se describe la arquitectura~\cite{iso42010} global del sistema de detección de intrusiones (IDS) mostrada en la Figura~\ref{fig:Arquitectura2}. El diagrama presenta las entradas, salidas y los módulos responsables del procesamiento~\cite{bass2021sap,kruchten1995} desde la captura del tráfico hasta la clasificación, visualización y exportación de datos. La organización en bloques facilita distinguir el \textbf{módulo de Captura}, el \textbf{\textit{Backend}} (orquestación, clasificación y estado) y el \textbf{\textit{Frontend}} (panel de control), así como los canales de interacción del \textbf{Administrador} y los artefactos de salida (\emph{CSV/TXT}).

\begin{landscape} \begin{figure}[p] \centering \includegraphics[height=1\textheight, trim = 0mm 11mm 0mm 11mm ,clip]{imagenes/diagramas/arquitectura/arquitectura.pdf} \caption{Diagrama de arquitectura del sistema.} \label{fig:Arquitectura2} \end{figure} 
\end{landscape}

\subsection*{Descripción de componentes}

\noindent\textbf{Atacante / Red.} Fuente de tráfico de red (paquetes) sobre la que opera el IDS. Puede representar tanto actividad legítima como potencialmente maliciosa.

\noindent\textbf{Captura.} Bloque encargado de transformar paquetes en \emph{flows} enriquecidos con características:
\begin{itemize}
  \item \textbf{\textit{Sniffer} Scapy.} Proceso/hilo de captura basado en Scapy; aplica filtros BPF y entrega paquetes al agregador de sesiones.
  \item \textbf{\textit{FlowSession}.} Mantenimiento de la tabla de sesiones/\textit{flows} (p.\,ej., por 5--tupla), con contadores, marcas temporales y estado del flujo.
  \item \textbf{\textit{Flow} + \textit{Features}.} Cálculo de \emph{features} a nivel de flujo (duración, tasas, \textit{flags}, estadísticos temporales) alineados con el \emph{feature map} del modelo.
  \item \textbf{\textit{ConsoleWriter}.} Serialización del flujo cerrado a un registro estructurado (para depuración/observabilidad) y envío al siguiente estadio.
\end{itemize}

\noindent\textbf{Cola de flujos.} \emph{Buffer} productor–consumidor que desacopla la captura/transformación de la clasificación. Amortigua picos y protege la latencia de captura, permitiendo políticas de \emph{backpressure}.

\noindent\textbf{\textit{Backend}.} Núcleo de procesado y orquestación:
\begin{itemize}
  \item \textbf{\textit{Worker} de clasificación.} Consumidor de la cola que aplica el mapeo/escala de \emph{features} y solicita predicciones al modelo.
  \item \textbf{CICIDSPredictor.} Servicio/modelo de inferencia (p.\,ej., \textit{Random Forest}) entrenado sobre \emph{features} compatibles con CICFlowMeter/CICIDS.
  \item \textbf{\textit{State}.} Gestor de estado y telemetría: controla ciclo de vida (iniciar/parar), expone métricas al \textit{Frontend} y coordina exportaciones.
\end{itemize}

\noindent\textbf{\textit{Frontend}.} \textbf{\textit{Dashboard} Reflex} que presenta KPIs, flujo de eventos y alertas, y permite la interacción del \textbf{Administrador} (iniciar/parar, filtros de captura, exportación).

\noindent\textbf{CSV/TXT (\textit{Export}).} Artefacto de salida para generación de \textit{datasets} (análisis/entrenamiento \textit{offline}), con esquema y metadatos versionados.

\subsection*{Flujo de datos extremo a extremo}

\begin{enumerate}
  \item El \emph{Sniffer Scapy} ingiere paquetes desde la interfaz de red y los entrega a \emph{FlowSession}.
  \item \emph{FlowSession} agrega paquetes en \emph{flows}; al expirar o cerrarse (FIN/RST/límites), el \emph{Flow} calcula \emph{features}.
  \item \emph{ConsoleWriter} serializa el \emph{flow} y lo publica en la \textbf{cola de flujos}.
  \item El \textbf{\textit{Worker}} consume cada \emph{flow}, aplica mapeo/escala y consulta a \textbf{CICIDSPredictor} para obtener \emph{label} y probabilidad.
  \item \textbf{\textit{State}} registra métricas, decide acciones (alerta, persistencia) y expone el estado al \textbf{\textit{Dashboard} Reflex}.
  \item El \textbf{Administrador} opera el sistema desde el \emph{dashboard} (iniciar/parar, filtros). Cuando se requiere, se dispara la ruta de \textbf{\textit{Export}} para generar \emph{CSV/TXT}.
\end{enumerate}

\subsection*{Aspectos transversales de diseño}

\begin{itemize}
  \item \textbf{Concurrencia y resiliencia.} Desacoplo productor–consumidor mediante la cola; reintentos y \emph{rate limiting} en rutas de alerta/exportación.
  \item \textbf{Reproducibilidad.} Versionado del modelo, \emph{scaler} y \emph{feature map} embebidos en cada predicción/exportación.
  \item \textbf{Observabilidad.} Métricas de captura (pps, \% \textit{drops}), tamaño/latencia de cola, \textit{throughput} de inferencia, ratio de alertas y estado del modelo.
  \item \textbf{Seguridad.} Control de acceso al panel; sanitización de datos en \textit{logs}/\textit{alertas}; protección de claves/artefactos de modelo.
  \item \textbf{Escalabilidad.} Posibilidad de múltiples \emph{workers} y balanceo de la cola; captura distribuida si la tasa lo requiere.
\end{itemize}

\subsection*{Trazabilidad con requisitos y casos de uso}
\begin{itemize}
  \item Captura y agregación: CU\_Capturar (RF2), CU\_Filtro (RF5).
  \item Clasificación y alertado: CU\_Clasificar (RF1), CU\_Alerta (RF4).
  \item Exportación: CU\_GenerarDataset (RF13).
  \item Orquestación y panel: CU\_Iniciar/CU\_Parar (RF10/RF11), CU\_Dashboard (RF8).
\end{itemize}

\section{Wireframe}

Los \emph{wireframes} representan la estructura y jerarquía visual sin detalle gráfico final. Se emplean como artefactos de \emph{diseño centrado en el usuario} de baja fidelidad para explorar la organización de la interfaz antes del diseño visual~\cite{iso9241-210}. A partir del \emph{mockup} existente del \emph{Dashboard} (Sección~\ref{sec:mockups}), se propone el siguiente \textbf{\textit{wireframe} de la vista principal} mostrado en la Figura~\ref{fig:wireframe-dashboard}, con una distribución en rejilla de 12 columnas~\cite{tidwell2019designing,cooper2014aboutface}: cabecera, banda de KPIs, panel de tráfico (izquierda) y columna derecha con alertas y configuración.

\begin{figure}[H]
    \centering
    \includegraphics[width=\linewidth, trim = 0mm 100mm 0mm 0mm, clip]{imagenes/diagramas/wireframe/wireframe.drawio.pdf}
    \caption{Wireframe de la vista principal del Dashboard.}
    \label{fig:wireframe-dashboard}
\end{figure}

\subsection*{Explicación del wireframe}
\begin{itemize}
  \item \textbf{Barra superior.} Muestra el título y el estado del sistema, junto con acciones globales (iniciar/parar/limpiar).
  \item \textbf{KPIs.} Cuatro tarjetas resumen: total de \emph{flows}, alertas, normales y ataques.
  \item \textbf{Tráfico de Red (izquierda).} Lista principal con los \emph{flows} recientes; cada ítem ofrece acceso a ``Detalles''.
  \item \textbf{Alertas (derecha).} Tarjetas con severidad, \textit{score} y contexto; acciones de gestión.
  \item \textbf{Configuración (derecha).} Selector de interfaz, filtro BPF y acción de exportación de \textit{dataset} (CSV).
\end{itemize}

\section{Vistas necesarias}
Además del \textbf{\textit{Dashboard} principal}, se consideran las siguientes vistas:
\begin{itemize}
  \item \textbf{Detalle de flujo.} Objetivo: inspección de atributos y \emph{features} de un \emph{flow}. Elementos: 4--tupla, tiempos, bytes/paquetes, \textit{flags}, decisión del modelo y \emph{score}.
  \item \textbf{Detalle de alerta.} Objetivo: análisis de una detección. Elementos: severidad, regla/modelo, evidencia (\textit{flow(s)} implicados), acciones (\textit{ack}, exportación).
  \item \textbf{Configuración de captura.} Objetivo: elegir interfaz y filtro BPF; previsualizar estado de captura. Elementos: selectores y validación.
  \item \textbf{Exportación de \textit{dataset}.} Objetivo: seleccionar rango y esquema; lanzar exportación CSV/TXT; ver progreso y resultado.
  \item \textbf{Estado del sistema.} Objetivo: telemetría (tasa de captura, tamaño de cola, latencias de inferencia, ratio de alertas).
  \item \textbf{Gestión de modelo (opcional).} Objetivo: ver versión activa, métricas y posibilidad de recarga/\textit{rollback} controlado.
\end{itemize}

La definición de vistas responde a una separación de preocupaciones orientada a tareas del usuario y a niveles de información, alineada con buenas prácticas de arquitectura de la información e interacción~\cite{garrett2010elements,tidwell2019designing}.

\section{Mockups}
\label{sec:mockups}

Como \emph{mockup}~\cite{cooper2014aboutface,tidwell2019designing} se emplea la implementación visual existente del \textbf{\textit{Dashboard}} (Figura~\ref{fig:mockup-dashboard}). Este \textit{mockup} ilustra el estilo final (tema oscuro, iconografía, colores por severidad)~\cite{iso9241-210} y sirve de referencia para el \emph{wireframe} de la Figura~\ref{fig:wireframe-dashboard}, del que deriva su estructura. En la siguiente ilustración ~\ref{fig:mockup-dashboard} se puede visualizar el mockup en el estado inicial.

\begin{figure}[H]
  \centering
  % Sustituir la ruta por la ubicación real del PNG/JPG del mockup
  \includegraphics[width=\linewidth]{imagenes/diagramas/mockups/mockupstart.png}
  \caption{Mockup de la interfaz del Dashboard (implementación actual).}
  \label{fig:mockup-dashboard}
\end{figure}

Ahora, también podemos visualizar el \textit{mockup} de la aplicación pero en estado de ejecución donde se podrá apreciar cómo el panel se va autorrellenando cuando se está ejecutando con el botón "Iniciar" en la siguiente Figura ~\ref{fig:mockup-dashboard-run}

\begin{figure}[H]
  \centering
  \includegraphics[width=\linewidth]{imagenes/diagramas/mockups/mockuprun.png}
  \caption{Mockup de la interfaz del Dashboard (implementación actual) en estado de ejecución.}
  \label{fig:mockup-dashboard-run}
\end{figure}

\subsection*{Relación wireframe–mockup}
El \emph{wireframe} define la jerarquía y disposición de los elementos (zonas funcionales y navegación), mientras que el \emph{mockup} concreta estilo visual (tipografía, colores, espaciados). Ambos representan la misma vista: el primero como guía estructural temprana; el segundo, como anticipo fiel del resultado final.
\clearpage\thispagestyle{empty}\cleardoublepage

\chapter{MATERIALES Y MÉTODOS}
Este capítulo describe los recursos utilizados y los procedimientos seguidos para construir el sistema IDS con captura por flujos (CICFlowMeter), componente de clasificación (Random Forest) e interfaz web (Reflex). Se incluyen decisiones de diseño, algoritmos en pseudo‑código,listados representativos del código fuente así como las funcionalidades más importantes desarrolladas en el proyecto.

\section{Materiales empleados}
\subsection{Hardware y entorno de desarrollo}

El desarrollo se ha realizado en un equipo personal de propósito general, usando \textbf{Visual Studio Code} como entorno de desarrollo. La tabla~\ref{tab:hw-mm} recoge las especificaciones del ordenador relevantes para la experimentación y el desarrollo.

\begin{longtable}{p{4cm}p{10cm}}
\textbf{Componente} & \textbf{Especificación} \\
\hline
CPU & Intel(R) Core(TM) i7-1065G7 CPU @ 1.5GHz \\
Memoria RAM & 12 GB \\
Disco Duro & SSD 512GB \\
Tarjeta Gráfica & Intel(R) Iris(R) Plus Graphics 128 MB \\
Sistema Operativo & \textit{(Windows 10 Home/Linux)} \\
IDE & Visual Studio Code \\
Red de pruebas & Segmento aislado de laboratorio; generador de tráfico \\
\hline
\caption{Hardware y entorno.}\label{tab:hw-mm}\\
\end{longtable}

\paragraph{Elección del modelo de captura de paquetes:}
Se estudiaron diferentes alternativas:
\begin{itemize}
  \item \textbf{Wireshark/TShark}: excelente para inspección/validación, pero su código base y acoplamiento interno dificultan extraer un módulo de captura independiente alineado con el objetivo de construir \textit{features} de flujo de forma programática.
  \item \textbf{tcpdump/libpcap}: opción muy eficiente y estándar para captura y filtros BPF; adecuada como comparador/validación.
  \item \textbf{nmap}: no es un capturador, sino un generador/escáner; se ha usado para \emph{probar} el IDS (escaneos de puertos), no para capturar.
  \item \textbf{Scapy + CICFlowMeter} (elegido): Scapy aporta una API de alto nivel para ingerir paquetes y CICFlowMeter agrega en \textit{flows} y calcula 83 características compatibles con datasets CIC‑IDS, simplificando la creación de datasets y la alimentación del modelo.
\end{itemize}
El criterio de elección ponderó: expresividad y rapidez de desarrollo (Python), disponibilidad de \textit{features} de flujo, integración con el pipeline del modelo de IA y validación con herramientas consolidadas (TShark/tcpdump).

Nótese que consideré un baremo amplio de opciones a elegir para desarrollar el capturador pero me resultó sumamente complejo decantarme por una opción factible. Como bien he comentado, encontré una API que tenía multitud de paquetes con métodos de utilidad que me ayudaron al desarrollo de dicho capturador. La API: \url{https://www.osgeo.cn/scapy/index.html} tuvo un gran impacto en el desarrollo del capturador y la extracción de características. A partir de ella, logré extraer prácticamente todas las características requeridas por el capturador, además de que también proporciona librerías para la captura de paquetes a bajo nivel, su manejo, la agrupación de flujos, entre otros. Aunque también este documento de scapy: \url{https://app.readthedocs.org/projects/scapy/downloads/pdf/stable/} me sirvió de gran utilidad.

\subsection{Software y herramientas}
En la siguiente tabla ~\ref{tab:sf-h} se pueden visualizar las herramientas y el software utilizado en la elaboración y desarrollo del proyecto.

\begin{longtable}{p{4cm}p{10cm}}
\textbf{Herramienta} & \textbf{Uso y versión} \\
\hline
Python & Lenguaje principal (captura, backend e IA) \\
Reflex & Framework de interfaz web (dashboard) \\
Scapy & Ingesta de paquetes y filtros \\
CICFlowMeter (src/cicflowmeter) & Agregación de \textit{flows} y cálculo de \textit{features} \\
scikit‑learn / sklearnex & Modelo Random Forest, escalado y aceleración CPU \\
RAPIDS/cuML (opcional) & Entrenamiento acelerado por GPU (si disponible) \\
pandas / numpy & Manipulación de datos \\
imbalanced‑learn & SMOTE y técnicas de balanceo \\
joblib & Persistencia de artefactos (modelo, escalador, mapeos) \\
Wireshark/TShark y tcpdump & Validación e inspección del tráfico \\
\hline
\caption{Software y herramientas.}\label{tab:sf-h}\\
\end{longtable}

\subsection{Conjuntos de datos}
Se analizaron los siguientes datasets de referencia:
\begin{itemize}
  \item \textbf{KDD Cup 99 y KDD99}: históricos y ampliamente usados; presentan obsolescencia y sesgos bien documentados.
  \item \textbf{NSL‑KDD}: corrige algunos problemas de KDD99, pero sigue alejado del tráfico moderno.
  \item \textbf{CIC‑BoT‑IoT}: centrado en IoT, con distribución específica.
  \item \textbf{CIC‑IDS 2017/2018/2019}: capturas recientes, con etiquetas y escenarios variados. Los ficheros se publican segmentados por días y, en ocasiones, por franjas horarias.
\end{itemize}

Se eligió \textbf{CIC‑IDS 2018} por variedad de ataques (fuerza bruta, DoS, ataques web, infiltración), disponibilidad de campos compatibles con CICFlowMeter y representatividad del tráfico.

Dado que CIC‑IDS 2018 está particionado en múltiples ficheros (por días y eventos), se implementó una \textbf{estrategia de unificación} en un único CSV:

\begin{enumerate}
  \item Carga de cada día seleccionado.
  \item Limpieza de filas espurias (p.\,ej., \texttt{Label == "Label"}).
  \item Muestreo estratificado por clase por día (tamaño ajustado por prioridad del escenario).
  \item Concatenación en un DataFrame único.
  \item Balanceo con una estrategia híbrida (submuestreo de clases dominantes + SMOTE con límites de tamaño).
\end{enumerate}

El resultado es un \textbf{dataset unificado y balanceado} (\texttt{cic\_ids\_unified\_balanced.csv}) apto para entrenamiento robusto.

El conjunto \textbf{CIC-CSE-IDS2018} fue creado de forma colaborativa por el \emph{Canadian Institute for Cybersecurity} (CIC) y el \emph{Communications Security Establishment} (CSE). Se diseñaron múltiples escenarios realistas con víctimas y servidores de diversos servicios. Para cada máquina se registró el tráfico y, a partir de éste, se extrajeron más de 80 variables por flujo mediante \textbf{CICFlowMeter}. A diferencia de datasets previos, además de las métricas de flujo clásicas, se incluyen explícitamente el \textbf{protocolo} de transporte y la \textbf{marca temporal} (timestamp) del evento, haciendo el total de variables superior a 80.

\paragraph{Variables disponibles}
A diferencia de CIC‑IDS2017, CIC‑CSE‑IDS2018 añade dos campos: \textbf{Protocolo} y \textbf{Marca de tiempo}. El conjunto de columnas completas se muestra en la Tabla~\ref{tab:cicids2018-features}.


\begin{longtable}{p{0.31\textwidth}p{0.31\textwidth}p{0.31\textwidth}}
\hline
\textbf{Columna 1} & \textbf{Columna 2} & \textbf{Columna 3} \\
\hline
\endfirsthead
\multicolumn{3}{c}{\small\itshape (continuación)}\\
\hline
\textbf{Columna 1} & \textbf{Columna 2} & \textbf{Columna 3} \\
\hline
\endhead
\hline
\multicolumn{3}{r}{\small\itshape (continúa en la siguiente página)}\\
\endfoot

\endlastfoot
Protocolo  & Marca de tiempo (timestamp) & Puerto de destino \\
Duración del flujo & Total de paquetes fwd & Total de paquetes bwd \\
Longitud total de paquetes fwd & Longitud total de paquetes bwd & Longitud máxima de paquetes fwd \\
Longitud mínima de paquetes fwd & Longitud máxima de paquetes bwd & Longitud mínima de paquetes bwd \\
Longitud media de paquetes fwd & Desviación estándar de la longitud de paquetes fwd & Longitud media de paquetes bwd \\
Desviación estándar de la longitud de paquetes bwd & Flujo en Bytes/s & Flujo en paquetes/s \\
Flujo IAT medio & Desviación estándar del flujo IAT & Máximo flujo IAT \\
Mínimo flujo IAT & Flujo IAT total fwd & Flujo IAT medio fwd \\
Desviación estándar del flujo IAT fwd & Flujo IAT fwd máximo & Flujo IAT fwd mínimo \\
Flujo IAT total bwd & Flujo IAT medio bwd & Desviación estándar del flujo IAT bwd \\
Flujo IAT bwd máximo & Flujo IAT bwd mínimo & Flags PSH fwd \\
Flags PSH bwd & Flags URG fwd & Flags URG bwd \\
Longitud de la cabecera fwd & Longitud de la cabecera bwd & Paquetes/s fwd \\
Paquetes/s bwd & Longitud mínima del paquete & Longitud máxima del paquete \\
Longitud media del paquete & Desviación estándar de la longitud del paquete & Varianza de la longitud del paquete \\
Conteo de flags FIN & Conteo de flags SYN & Conteo de flags RST \\
Conteo de flags PSH & Conteo de flags ACK & Conteo de flags URG \\
Conteo de flags CWR & Conteo de flags ECE & Ratio de bajada/subida \\
Tamaño medio de paquete & Tamaño medio de segmento fwd & Tamaño medio de segmento bwd \\
Longitud de la cabecera fwd media & Bytes/bulk promedio fwd & Packets/bulk medios fwd \\
Ratio de bulk fwd medio & Bytes/bulk medios bwd & Packets/bulk medios bwd \\
Ratio de bulk bwd medio & Subflujo de paquetes fwd & Subflujo de bytes fwd \\
Subflujo de paquetes bwd & Subflujo de bytes bwd & Número total de bits fwd en la ventana inicial (fwd) \\
Número total de bits fwd en la ventana inicial (bwd) & Conteo de paquetes fwd con payload TCP & Tamaño mínimo de segmentos fwd \\
Tiempo mínimo que un segmento estuvo activo & Tiempo medio que un segmento estuvo activo & Tiempo máximo que un segmento estuvo activo \\
Desviación estándar de los tiempos activos & Tiempo mínimo que un segmento estuvo inactivo & Tiempo medio que un segmento estuvo inactivo \\
Tiempo máximo que un segmento estuvo inactivo & Desviación estándar de los tiempos inactivos & \\
\hline
\caption{Variables del dataset CIC-CSE-IDS2018 (incluye Protocolo y Marca de tiempo añadidas respecto a CIC-IDS2017).}
\label{tab:cicids2018-features}
\end{longtable}

\paragraph{Ejemplos del dataset}
Mostraremos dos ejemplos de este conjunto de datos para visualizar su estructura:

\begin{lstlisting}[style=csvline,caption={Ejemplo 1 (CIC‑CSE‑IDS2018, fila CSV abreviada)},label=List.CICIDS2018Example1]
0,0,14/02/2018 08:36:39,112638623,3,0,0,0,0,0,0,0,0,0,0,0,0,0.0266338483,56319311.5,301.9345955667,56319525,56319098,112638623,56319311.5,301.9345955667,56319525,56319098,0,0,0,0,0,0,0,0,0,0,0,0.0266338483,0,0,0,0,0,0,0,0,0,0,0,0,0,0,0,0,0,0,0,0,0,0,0,0,3,0,0,0,-1,-1,0,0,0,0,0,0,56319311.5,301.9345955667,56319525,56319098,Benign
\end{lstlisting}

\begin{lstlisting}[style=csvline,caption={Ejemplo 2 (CIC‑CSE‑IDS2018, fila CSV abreviada)},label=List.CICIDS2018Example2]
22,6,14/02/2018 08:40:13,6453966,15,10,1239,2273,744,0,82.6,196.7412368715,976,0,227.3,371.6778922072,544.1615279659,3.8735871865,268915.25,247443.778966007,673900,22,6453966,460997.571428571,123109.423587757,673900,229740,5637902,626433.555555556,455082.21422401,1167293,554,0,0,0,0,488,328,2.3241523119,1.5494348746,0,976,135.0769230769,277.8347599674,77192.1538461539,0,0,0,1,0,0,0,0,0,140.48,82.6,227.3,0,0,0,0,0,0,15,1239,10,2273,65535,233,6,32,0,0,0,0,0,0,0,0,Benign
\end{lstlisting}

\paragraph{Días y ataques utilizados}
En este trabajo se han utilizado los días y ataques indicados en la Tabla~\ref{tab:cicids2018-dias}, siguiendo el script \texttt{unify\_datasets.py}.

\begin{longtable}{p{2.8cm}p{8.8cm}p{2cm}}
\textbf{Día} & \textbf{Ataques incluidos} & \textbf{Prioridad} \\
\hline
02-14-2018 & FTP-BruteForce, SSH-Bruteforce & alta \\
02-15-2018 & DoS GoldenEye, DoS Slowloris & alta \\
02-23-2018 & Brute Force Web, XSS, SQL Injection & alta \\
03-01-2018 & Infiltration & media \\
\hline
\caption{Días y ataques empleados de CIC-CSE-IDS2018 (según \texttt{unify\_datasets.py}).}\label{tab:cicids2018-dias}
\end{longtable}

\paragraph{Distribución por tipo de tráfico}
Los recuentos principales del dataset aparecen en la Tabla~\ref{tab:cicids2018-detalles}.

\begin{longtable}{p{3.2cm}p{7.5cm}p{3.2cm}}
\hline
\textbf{Tipo de tráfico} & \textbf{Tipo de ataque} & \textbf{Número de ejemplos} \\
\hline
\endfirsthead
\multicolumn{3}{c}{\small\itshape (continuación)}\\
\hline
\textbf{Tipo de tráfico} & \textbf{Tipo de ataque} & \textbf{Número de ejemplos} \\
\hline
\endhead
\multicolumn{3}{r}{\small\itshape (fin)}\\
\endfoot

\textbf{Benigno} & \textbf{-} & \textbf{2.110.356} \\
Ataque & FTP-Bruteforce & 193.360 \\
Ataque & SSH-Bruteforce & 187.589 \\
Ataque & DoS-GoldenEye & 41.508 \\
Ataque & DoS-Slowloris & 10.990 \\
Ataque & DoS-SlowHttpTest & 91.434 \\
Ataque & DoS-AttackHulk & 461.912 \\
\textbf{Ataque} & \textbf{Suma total} & \textbf{986.793} \\
\caption{Detalles de los datos del dataset CIC-CSE-IDS2018.}
\label{tab:cicids2018-detalles}\\
\end{longtable}

\section{Métodos de adquisición y preprocesamiento de datos}

\subsection*{Captura, \textit{sessionización} de flujos y arquitectura del capturador}
La lógica de captura se apoya en \textbf{CICFlowMeter}, cuya estructura reside en \texttt{/src/cicflowmeter}:
\begin{itemize}
  \item \texttt{flow\_session.py}: mantiene la tabla de flujos (\texttt{self.flows}) e implementa \texttt{on\_packet\_received()} y \texttt{clean\_write\_flows()}.
  \item \texttt{flow.py}: representa un \emph{flow} y computa las 83 \textit{features} en \texttt{get\_data()}.
  \item \texttt{features/}: módulos para longitudes, tiempos, flags, bytes, etc. Lo componen:
  \begin{itemize}
    \item \texttt{packet\_time.py}: Extrae medidas temporales a nivel de flujo y por dirección.
    \item \texttt{packet\_length.py}: Agrega estadísticas de longitudes de paquete (totales o por dirección).
    \item \texttt{packet\_count.py}: Cuenta paquetes y calcula tasas y relaciones.
    \item \texttt{flow\_bytes.py}: Métricas de bytes totales y por dirección, y bytes de cabecera.
    \item \texttt{flag\_count.py}: Cuenta flags TCP por dirección o en total (SYN, ACK, FIN, RST, URG, PSH, CWR, ECE).
    \item \texttt{response\_time.py}: Mide tiempos de respuesta entre un paquete saliente y el siguiente paquete entrante.
  \end{itemize}
  \item \texttt{writer.py}: salida en CSV/TXT.
  \item \texttt{constants.py}: parámetros de expiración y limpieza (\texttt{EXPIRED\_UPDATE}, \texttt{MAX\_COLLECT\_PACKETS}, \texttt{FLOW\_DURATION}).
\end{itemize}

\paragraph{Nota sobre la clave del flujo (implementación en este proyecto)}
Aunque en la literatura es habitual definir flujos con una \textbf{5‑tupla} (IP/puerto origen, IP/puerto destino y protocolo), en este proyecto la clave interna es una \textbf{4‑tupla normalizada por dirección} (src\_ip, src\_port, dst\_ip, dst\_port), \emph{sin incluir el protocolo}, en coherencia con el código. Esto funciona correctamente en los escenarios evaluados; si se previesen colisiones TCP/UDP para la misma 4‑tupla, puede ampliarse fácilmente a 5‑tupla añadiendo \texttt{protocol} a la clave.

\subsection{Módulos de extracción de características (features/)}

A continuación se detallan los módulos del directorio \texttt{features/} que computan las variables empleadas por el sistema. Cada módulo opera sobre una instancia de \texttt{Flow} (que contiene la lista de paquetes con su dirección FORWARD/REVERSE) y expone métodos para obtener medidas agregadas. Estas medidas son después ensambladas en \texttt{flow.get\_data()}.

\paragraph{packet\_time.py}
\begin{itemize}
  \item \textbf{Entradas}: \texttt{flow.packets} con marcas de tiempo (\texttt{packet.time}).
  \item \textbf{Salidas} principales:
    \begin{itemize}
      \item \texttt{get\_duration()}: duración del flujo en segundos (max tiempo relativo - min tiempo relativo).
      \item \texttt{get\_packet\_iat(dir)}: lista de IATs (microsegundos) entre paquetes consecutivos (totales o por dirección).
      \item \texttt{get\_timestamp()}: marca de tiempo legible del primer paquete.
      \item Estadísticos: \texttt{get\_mean()}, \texttt{get\_std()}, \texttt{get\_median()}, \texttt{get\_mode()}, \texttt{get\_skew()}, \texttt{get\_cov()} sobre tiempos relativos.
    \end{itemize}
\end{itemize}

\begin{lstlisting}[language=Python,caption={IATs y duración del flujo (extracto)},label=List.PacketTime]
def get_packet_iat(self, packet_direction=None):
    if packet_direction is not None:
        packets = [p for p,d in self.flow.packets if d == packet_direction]
    else:
        packets = [p for p,_ in self.flow.packets]
    return [1e6 * float(packets[i].time - packets[i-1].time)
            for i in range(1, len(packets))]

def get_duration(self):
    times = self._get_packet_times()
    return max(times) - min(times)
\end{lstlisting}

\paragraph{packet\_length.py}

\begin{itemize}
  \item \textbf{Entradas}: \texttt{len(packet)} y cabeceras IP (\texttt{ihl}).
  \item \textbf{Salidas}: mínimos, máximos, media, desviación estándar, varianza, suma total por dirección; métricas derivadas como \texttt{pkt\_len\_mean/std/var}, \texttt{pkt\_len\_max/min}, \texttt{totlen\_fwd/bwd\_pkts}.
\end{itemize}

\begin{lstlisting}[language=Python,caption={Estadísticas de longitudes (extracto)},label=List.PacketLength]
def get_packet_length(self, packet_direction=None):
    if packet_direction is not None:
        return [len(p) for p,d in self.flow.packets if d == packet_direction]
    return [len(p) for p,_ in self.flow.packets]

def get_total(self, packet_direction=None):
    return sum(self.get_packet_length(packet_direction))

def get_std(self, packet_direction=None):
    import numpy as np
    var = self.get_packet_length(packet_direction)
    return float(np.sqrt(np.var(var)))
\end{lstlisting}

\paragraph{packet\_count.py}

\begin{itemize}
  \item \textbf{Salidas}: \texttt{get\_total(dir)}, \texttt{get\_rate(dir)} = paquetes/segundo, \texttt{get\_down\_up\_ratio()} = bwd/fwd, \texttt{has\_payload(dir)} = conteo de paquetes con carga útil.
\end{itemize}

\begin{lstlisting}[language=Python,caption={Conteos y tasas de paquetes (extracto)},label=List.PacketCount]
def get_total(self, packet_direction=None):
    if packet_direction is not None:
        return len([1 for _,d in self.flow.packets if d == packet_direction])
    return len(self.flow.packets)

def get_rate(self, packet_direction=None):
    duration = PacketTime(self.flow).get_duration()
    return self.get_total(packet_direction) / duration if duration > 0 else 0.0

@staticmethod
def get_payload(packet):
    if "TCP" in packet: return packet["TCP"].payload
    if "UDP" in packet: return packet["UDP"].payload
    return 0
\end{lstlisting}

\paragraph{flow\_bytes.py}

\begin{itemize}
  \item \textbf{Salidas}: \texttt{get\_bytes()}, \texttt{get\_bytes\_sent()} (FORWARD), \texttt{get\_bytes\_received()} (REVERSE), tasas por segundo \texttt{get\_rate()/get\_sent\_rate()/get\_received\_rate()}, bytes de cabecera forward/reverse, tamaños mínimos de cabecera, ratios \texttt{get\_header\_in\_out\_ratio()}, métricas \emph{bulk}: bytes/paquetes por bulk y tasa de bulk.
  \item El tamaño de cabecera se estima con \texttt{IP.ihl * 4} si hay TCP, en caso contrario 8 bytes base.
\end{itemize}

\begin{lstlisting}[language=Python,caption={Bytes y tasas; cabeceras y bulk (extracto)},label=List.FlowBytes]
def get_bytes(self):
    return sum(len(p) for p,_ in self.flow.packets)

def get_rate(self):
    dur = PacketTime(self.flow).get_duration()
    return self.get_bytes() / dur if dur > 0 else 0.0

def get_forward_header_bytes(self):
    from scapy.layers.inet import IP, TCP
    def hdr_size(p): return p[IP].ihl * 4 if TCP in p else 8
    return sum(hdr_size(p) for p,d in self.flow.packets if d is PacketDirection.FORWARD)

def get_bytes_per_bulk(self, direction):
    if direction is PacketDirection.FORWARD and self.flow.forward_bulk_count:
        return self.flow.forward_bulk_size / self.flow.forward_bulk_count
    if direction is PacketDirection.REVERSE and self.flow.backward_bulk_count:
        return self.flow.backward_bulk_size / self.flow.backward_bulk_count
    return 0.0
\end{lstlisting}

\paragraph{flag\_count.py}

\begin{itemize}
  \item \textbf{Salida}: \texttt{count(flag, dir)} devuelve el número de paquetes cuyo campo \texttt{TCP.flags} contiene el flag indicado.
\end{itemize}

\begin{lstlisting}[language=Python,caption={Conteo de flags TCP (extracto)},label=List.FlagCount]
def count(self, flag, packet_direction=None):
    cnt = 0
    if packet_direction is not None:
        packets = (p for p,d in self.flow.packets if d == packet_direction)
    else:
        packets = (p for p,_ in self.flow.packets)
    for p in packets:
        if flag[0] in p.sprintf("%TCP.flags%"):
            cnt += 1
    return cnt
\end{lstlisting}

\paragraph{response\_time.py}

\begin{itemize}
  \item \textbf{Salida principal}: \texttt{get\_dif()} produce una lista de diferencias de tiempo; sobre ella se calculan media, mediana, varianza, desviación, sesgo y coeficiente de variación.
\end{itemize}

\begin{lstlisting}[language=Python,caption={Diferencias de tiempo solicitud-respuesta (extracto)},label=List.ResponseTime]
def get_dif(self):
    diffs, temp_p, temp_d = [], None, None
    for p,d in self.flow.packets:
        if temp_d == PacketDirection.FORWARD and d == PacketDirection.REVERSE:
            diffs.append(float(p.time - temp_p.time))
        temp_p, temp_d = p, d
    return diffs
\end{lstlisting}

\begin{algorithm}[H]
\DontPrintSemicolon
\SetAlgoLined
\SetKwInOut{KwIn}{Entrada}
\SetKwInOut{KwOut}{Salida}
\KwIn{\texttt{flow.packets} con marcas de tiempo y dirección}
\KwOut{Diccionario con 83 variables por flujo}
\Begin{ 
  Crear instancias: \texttt{FlowBytes}, \texttt{PacketCount}, \texttt{PacketLength}, \texttt{PacketTime}, \texttt{FlagCount}\;
  Calcular duración y tasas globales: \texttt{flow\_byts\_s}, \texttt{flow\_pkts\_s}\;
  Calcular métricas por dirección (FORWARD/REVERSE): totales, media, min, max, desviación de longitudes; IATs; bytes/cabeceras; paquetes con payload\;
  Calcular flags TCP por dirección y totales\;
  Calcular ratios: \texttt{down\_up\_ratio}, tamaños medios de paquete\;
  Calcular métricas de actividad/idle y \emph{bulk} (bytes/paquetes/tasa por bulk)\;
  Rellenar campos duplicados de compatibilidad CIC (p.\,ej., \texttt{fwd\_seg\_size\_avg})\;
}
\caption{Construcción del vector de características por flujo.}
\label{alg:assemble_features}
\end{algorithm}

\paragraph{Consideraciones y casos límite}
\begin{itemize}
  \item Si la \textbf{duración} del flujo es cero, las \textbf{tasas} devuelven 0 para evitar divisiones por cero.
  \item Si no hay suficientes elementos, los \textbf{estadísticos} (varianza, desviación) vuelven 0.
  \item Las \textbf{métricas bulk} solo se computan cuando se supera \texttt{BULK\_BOUND}; en otro caso devuelven 0.
  \item La \textbf{dirección} se usa para separar FORWARD y REVERSE, manteniendo la coherencia de sub-métricas.
\end{itemize}

\begin{algorithm}[H]
\SetAlgoLined
\KwResult{Tabla de flujos y registros listos para escritura}
\textbf{Entrada}: paquetes de red, interfaz y filtro BPF\;
Inicializar diccionario \texttt{flows} y contador \texttt{count}\;
\ForCada{paquete recibido}{
  $key \leftarrow$ 4‑tupla normalizada con dirección FORWARD\;
  $flow \leftarrow flows[(key,count)]$ si existe; en caso contrario probar REVERSE\;
  \If{$flow == \varnothing$}{
    Crear \texttt{Flow(paquete, FORWARD)} y registrar en \texttt{flows[(key,0)]}\;
  }
  \ElseIf{paquete llega tras \texttt{EXPIRED\_UPDATE}}{
    Incrementar \texttt{count} (nueva sesión de la misma 4‑tupla)\;
    Crear/recuperar \texttt{flows[(key,count)]}\;
  }
  \ElseIf{paquete TCP con FIN/RST}{
    Añadir paquete al \emph{flow} y llamar a \texttt{clean\_write\_flows()}\;
    \textbf{continuar}\;
  }
  Añadir paquete al \emph{flow} (\texttt{flow.add\_packet})\;
  \If{\#paquetes procesados múltiplo de \texttt{MAX\_COLLECT\_PACKETS} \textbf{o} $flow.duration > \texttt{FLOW\_DURATION}$}{
    \texttt{clean\_write\_flows()}\;
  }
}

\caption{Sessionización y control de ciclo de vida de flujos (resumen de \texttt{on\_packet\_received}).}
\label{alg:on_packet_received}
\end{algorithm}

\paragraph{Explicación}
\textbf{Evitar duplicados}: al buscar primero en FORWARD y después en la clave invertida (REVERSE) se agrupa A$\leftrightarrow$B en un único \texttt{Flow}. \\
\textbf{Sub‑sesiones con \texttt{count}}: cuando hay pausas largas (\texttt{EXPIRED\_UPDATE}) o una conversación supera \texttt{FLOW\_DURATION}, se \emph{trocea} en sub‑sesiones numeradas (\texttt{(key,0)}, \texttt{(key,1)}, \ldots). Ejemplo: 10.0.0.1:50000$\rightarrow$10.0.0.2:80 habla 40s, se para 200s y reanuda; se continúa en \texttt{(key,1)}. Así se evitan “flujos estancados”, se controlan memoria y métricas por tramo ejecutándose el método  \texttt{clean\_write\_flows()} periódicamente o por exceder la duración máxima. \\
\textbf{Cierre inmediato}: FIN/RST fuerzan \texttt{clean\_write\_flows()}.


\begin{algorithm}[H]
\SetAlgoLined
\KwResult{Escritura de flujos expirados o cerrados}
\textbf{Entrada}: $latest\_time$ (\texttt{None} o tiempo de referencia)\;
\ForCada{clave $k$ en \texttt{flows}}{
  $flow \leftarrow flows[k]$\;
  \If{$latest\_time \neq \varnothing$ \textbf{y} $latest\_time - flow.latest\_timestamp < \texttt{EXPIRED\_UPDATE}$ \textbf{y} $flow.duration < 90$}{
    \textbf{continuar}\;
  }
  $data \leftarrow flow.get\_data()$ \tcp*{83 características (CIC)}
  \texttt{writer.write(data)}\;
  Eliminar \texttt{flows[k]}\;
}
\caption{Extracción y escritura de flujos (\texttt{clean\_write\_flows}).}
\label{alg:clean_write_flows}
\end{algorithm}

\paragraph{Explicación}
Se evita cerrar flujos que aún parecen activos (recientes y de corta duración). Para el resto, \texttt{get\_data()} calcula métricas de longitudes, tiempos, tasas y conteos de flags, y las escribe por CSV/TXT.

Actúa como \textit{colector}: mantiene \textbf{activos} los flujos recientes y cortos, y vuelca/elimina los \textbf{expirados o largos}. Esto estabiliza el consumo de memoria y produce registros con todas las \textit{features} calculadas por \texttt{get\_data()}.

\begin{lstlisting}[style=tfgpython,caption={Cálculo de features a nivel de flujo (extracto)},label=List.FlowGetData]
# flow.py (extracto)
def get_data(self) -> dict:
    flow_bytes = FlowBytes(self)
    flag_count = FlagCount(self)
    packet_count = PacketCount(self)
    packet_length = PacketLength(self)
    packet_time = PacketTime(self)
    flow_iat = get_statistics(self.flow_interarrival_time)
    forward_iat = get_statistics(packet_time.get_packet_iat(PacketDirection.FORWARD))
    backward_iat = get_statistics(packet_time.get_packet_iat(PacketDirection.REVERSE))

    data = {
        "src_ip": self.src_ip, "dst_ip": self.dest_ip,
        "src_port": self.src_port, "dst_port": self.dest_port,
        "protocol": self.protocol,
        "timestamp": packet_time.get_timestamp(),
        "flow_duration": 1e6 * packet_time.get_duration(),
        "flow_byts_s": flow_bytes.get_rate(),
        "flow_pkts_s": packet_count.get_rate(),
        "fwd_pkts_s": packet_count.get_rate(PacketDirection.FORWARD),
        "bwd_pkts_s": packet_count.get_rate(PacketDirection.REVERSE),
        "tot_fwd_pkts": packet_count.get_total(PacketDirection.FORWARD),
        "tot_bwd_pkts": packet_count.get_total(PacketDirection.REVERSE),
        "totlen_fwd_pkts": packet_length.get_total(PacketDirection.FORWARD),
        "totlen_bwd_pkts": packet_length.get_total(PacketDirection.REVERSE),
        "fwd_pkt_len_max": packet_length.get_max(PacketDirection.FORWARD),
        "bwd_pkt_len_max": packet_length.get_max(PacketDirection.REVERSE),
        # ... (resto de campos, incluidos flags e IATs)
    }
    # Duplicados necesarios para compatibilidad CIC
    data["fwd_seg_size_avg"] = data["fwd_pkt_len_mean"]
    data["bwd_seg_size_avg"] = data["bwd_pkt_len_mean"]
    data["subflow_fwd_pkts"] = data["tot_fwd_pkts"]
    # ...
    return data
\end{lstlisting}

\paragraph{Explicación}
Cada submódulo encapsula una familia de \emph{features}. Por ejemplo, \texttt{PacketLength} calcula máximos, mínimos, medias y desviaciones; \texttt{PacketTime} produce IATs y duración; \texttt{FlagCount} recorre los flags TCP. Al final se rellenan campos duplicados que CIC-IDS espera por compatibilidad ya que los CSV de los CIC incluyen pares de columnas equivalentes con distinto nombre (\emph{p.ej.}, \texttt{fwd\_seg\_size\_avg} = \texttt{fwd\_pkt\_len\_mean}; \texttt{subflow\_fwd\_pkts} = \texttt{tot\_fwd\_pkts}). El \textbf{StandardScaler} y el \textbf{modelo} se entrenan con la lista completa de nombres, y en inferencia deben recibirse \emph{exactamente} esos nombres y en el mismo orden. Por eso se rellenan explícitamente estos duplicados, garantizando compatibilidad y evitando errores de columnas ausentes o desalineación del vector.

\subsection*{Preprocesamiento para inferencia}
El predictor transforma el diccionario del capturador al vector requerido por el modelo:
\begin{algorithm}[H]
\SetAlgoLined
\KwResult{Vector de \textit{features} escaladas y predicción}
Inicializar diccionario \texttt{features} con todas las columnas esperadas a 0\;
\ForCada{(clave\_capturador, clave\_CIC) en \texttt{feature\_mapping}}{
  \If{clave\_capturador $\in$ \texttt{flow\_data}}{
    Copiar valor y normalizar: \texttt{Protocol} (TCP=6, UDP=17, ICMP=1), \texttt{Timestamp} numérico\;
    Asignar en \texttt{features[clave\_CIC]}\;
  }
}
Ordenar columnas según \texttt{cic\_features} y construir \texttt{DataFrame}\;
Reemplazar \texttt{NaN}/inf por 0 y aplicar \texttt{StandardScaler}\;
Inferir con \texttt{RandomForest.predict}/\texttt{predict\_proba} y devolver (etiqueta, probabilidad)\;
\caption{Transformación y predicción en el componente \texttt{CICIDSPredictor}.}
\label{alg:predictor}
\end{algorithm}

\begin{lstlisting}[style=tfgpython,caption={Transformar y predecir (extracto)},label=List.Predictor]
# cicidspredictor.py (extracto)
def predict(self, flow_data: Dict) -> Tuple[str, float]:
    features = self.transform_flow_data(flow_data)   # mapping + orden + limpieza
    features_scaled = self.scaler.transform(features)
    yhat = self.model.predict(features_scaled)[0]
    proba = self.model.predict_proba(features_scaled)[0]
    if len(proba) == 2:
        return ("Normal", proba[0]) if yhat == 0 else ("Malicious", proba[1])
    return ("Normal" if yhat == 0 else "Malicious",
            proba[yhat] if yhat < len(proba) else 0.0)
\end{lstlisting}

\paragraph{Explicación}
Se garantiza el orden correcto de columnas, se reescala con el \texttt{StandardScaler} entrenado y se devuelve etiqueta/probabilidad. El código es defensivo frente a modelos binarios o con múltiples clases.

\section{Métodos para la generación de datasets}
En esta sección, se abordará la explicación de cada uno de los métodos que nos han sido de utilidad para la generación de datasets, el objetivo principal de este proyecto.

\subsection*{Herramienta de línea de comandos \texttt{sniffer.py} y modo dataset}
El archivo \texttt{sniffer.py} implementa una \textbf{CLI} para ejecutar el capturador y/o generar datasets. Su función es doble: (i) captura en vivo con interfaz y filtro BPF configurables; (ii) \textbf{creación de datasets} con la opción \texttt{-c}, escribiendo ficheros \texttt{.csv} o \texttt{.txt} con todas las columnas CIC.

\paragraph{Opciones principales}
\begin{itemize}
  \item \texttt{-i/--interface}: interfaz de red (p.\,ej., \texttt{eth0}).
  \item \texttt{-t/--txt}: Flujo de salida con formato de fichero \textbf{.txt}.
  \item \texttt{-c/--csv}: Flujo de salida con formato de fichero \textbf{.csv}.
  \item \texttt{-f / --file}: Captura los datos desde un archivo dado.
  \item \texttt{-h/--help}: Muestra este mensaje de ayuda y sale.
  \item \texttt{-v/--verbose}: Añade más detalle al flujo de ejecución del script.
\end{itemize}

Cabe destacar que en el \textbf{README.md} del proyecto del capturador, se reflejan las opciones explicadas con varios ejemplos para su entendimiento y uso del mismo.

\paragraph{Uso típico}
\begin{lstlisting}[style=tfgbash,caption={Invocaciones representativas de sniffer.py},label=List.SnifferCLI]
# Captura en vivo y genera dataset CSV con columnas CIC
capturador -i eth0 -c flows.csv

# Igual pero a TXT 
capturador -i eth0 -t flows.txt

# Procesar un PCAP y volcar a CSV (útil para reproducir ataques/días concretos)
capturador -f example.pcap -c flows.csv

# Solo muestra un panel de ayuda donde figuran las opciones disponibles a ejecutar
capturador -h
\end{lstlisting}

\paragraph{Funcionamiento interno en modo \texttt{-c}}
Inicializa el \texttt{Writer} (\texttt{CSVWriter}/\texttt{TXTWriter}), ejecuta el bucle de sessionización (Algoritmo~\ref{alg:on_packet_received}) y delega en \texttt{clean\_write\_flows()} (Algoritmo~\ref{alg:clean_write_flows}) el cálculo de \texttt{get\_data()} y la escritura de cabecera + registros. El fichero resultante es \textbf{directamente consumible} por el pipeline de entrenamiento (\texttt{setup\_system.py}) y por el componente de inferencia (mismos nombres y orden de columnas).

\subsection*{Salida desde la captura}
La generación de datasets desde la captura usa los \texttt{Writer} disponibles (CSV/TXT). La salida incluye todas las columnas compatibles con CIC para facilitar entrenamiento y análisis.

\begin{lstlisting}[style=tfgpython,caption={Fábrica de escritores y salida en CSV (extracto)},label=List.WriterFactory]
class CSVWriter(OutputWriter):
    def __init__(self, output_file) -> None:
        super().__init__(output_file)
        self.line = 0
        self.writer = csv.writer(self.file)

    def write(self, data: dict) -> None:
        if self.line == 0:
            self.writer.writerow(data.keys()) # cabecera
        self.writer.writerow(data.values()) # registro
        self.line += 1
\end{lstlisting}

\paragraph{Explicación}
El primer registro escribe la cabecera; los siguientes, los valores de cada flujo. Este CSV es compatible con el preprocesado del entrenador.

\subsection*{Unificación y balanceo de CIC-IDS2018}
Para consolidar CIC‑IDS2018 se emplea el \textit{unificador}:

\begin{algorithm}[H]
\SetAlgoLined
\KwResult{Archivo \texttt{cic\_ids\_unified\_balanced.csv}}
\ForCada{día en \{02-14, 02-15, 02-23, 03-01, ...\}}{
  Cargar CSV; eliminar filas espurias (\texttt{Label == "Label"})\;
  Muestrear estratificadamente por clase (tamaño según prioridad del día)\;
  Añadir columna \texttt{dataset\_origin} y acumular\;
}
Concatenar muestras en un único DataFrame\;
Limpiar datos: convertir numéricos; mapear \texttt{Protocol}; gestionar \texttt{Timestamp}\;
Balanceo híbrido: submuestrear clases dominantes y aplicar SMOTE con límites (\texttt{MAX\_SMOTE\_SAMPLES})\;
Guardar CSV resultante y reportar distribución final\;
\caption{Unificación y balanceo de CIC‑IDS2018.}
\label{alg:unify_balance}
\end{algorithm}


\begin{lstlisting}[style=tfgpython,caption={Balanceo híbrido (esqueleto simplificado)},label=List.HybridBalance]
from imblearn.over_sampling import SMOTE
from imblearn.under_sampling import RandomUnderSampler

def balance_dataset(df):
    X, y = preparar_xy(df)
    unders = RandomUnderSampler(sampling_strategy=targets_intermedios,
                                random_state=42)
    X_u, y_u = unders.fit_resample(X, y)

    smote = SMOTE(sampling_strategy=targets_finales,
                  random_state=42, k_neighbors=3)
    X_b, y_b = smote.fit_resample(X_u, y_u)
    return X_b, y_b
\end{lstlisting}


\section{Métodos del componente de detección de intrusiones}
\subsection{Selección y justificación de modelo de machine learning}
Se utiliza \textbf{Random Forest} por su rendimiento en datos tabulares, robustez ante \textit{features} heterogéneas, baja sensibilidad a escalado y cierta interpretabilidad (importancias). Se consideraron conjuntos base (heurísticas de umbrales, regresión logística, Naive Bayes) como referencias.

\subsection{Proceso de entrenamiento y validación del modelo}
\begin{algorithm}[H]
\SetAlgoLined
\KwResult{Modelo y artefactos persistidos}
Cargar CSV (unificado o día concreto) por \textit{chunks} para limitar memoria\;
Preprocesar: mapear \texttt{Protocol}, normalizar \texttt{Timestamp}, garantizar \texttt{cic\_features}\;
Etiquetado binario: \texttt{y = (Label != "Benign")}\;
Dividir en \texttt{train/test} (20\%, estratificado)\;
Ajustar \texttt{StandardScaler} con \texttt{train} y transformar \texttt{train/test}\;
Entrenar \texttt{RandomForest} (\texttt{n\_estimators=100}, \texttt{max\_depth=20}, \texttt{class\_weight='balanced'})\;
Evaluar (\texttt{accuracy}, \texttt{classification\_report})\;
Guardar \texttt{cic\_ids\_model.pkl}, \texttt{cic\_ids\_scaler.pkl}, \texttt{feature\_mapping.pkl}, \texttt{cic\_features.pkl}\;
\caption{Entrenamiento y evaluación (CPU/GPU con aceleraciones cuando estén disponibles).}
\label{alg:train}
\end{algorithm}

\begin{lstlisting}[style=tfgpython,caption={Entrenamiento optimizado (extracto)},label=List.Trainer]
from sklearn.model_selection import train_test_split
from sklearn.preprocessing import StandardScaler
from sklearn.ensemble import RandomForestClassifier
from sklearn.metrics import accuracy_score, classification_report
import joblib

X_train, X_test, y_train, y_test = train_test_split(
    X, y, test_size=0.2, random_state=42, stratify=y
)

scaler = StandardScaler().fit(X_train)
X_train_scaled = scaler.transform(X_train)
X_test_scaled  = scaler.transform(X_test)

model = RandomForestClassifier(
    n_estimators=100, max_depth=20,
    class_weight='balanced', n_jobs=-1,
    random_state=42
)
model.fit(X_train_scaled, y_train)

y_pred = model.predict(X_test_scaled)
print("accuracy:", accuracy_score(y_test, y_pred))
print(classification_report(y_test, y_pred))

joblib.dump(model, 'cic_ids_model.pkl')
joblib.dump(scaler, 'cic_ids_scaler.pkl')
\end{lstlisting}

\subsection{Métodos de integración del modelo}
La integración en tiempo real se realiza en \texttt{state.py}:
\begin{itemize}
  \item \textbf{ConsoleWriter} publica cada flujo en una \textbf{cola} (\texttt{queue.Queue}).
  \item Un \textbf{worker} (\texttt{process\_flows\_worker}) consume la cola y llama a \texttt{add\_real\_flow\_direct()}, que:
  \begin{enumerate}
    \item obtiene la predicción y probabilidad de \texttt{CICIDSPredictor};
    \item construye el objeto de flujo para la UI;
    \item genera una alerta si el flujo es malicioso.
  \end{enumerate}
\end{itemize}

La app web consume flujos desde una cola y aplica predicción antes de mostrarlos.

\begin{lstlisting}[style=tfgpython,caption={Worker de procesamiento y aplicación de la IA (extracto)},label=List.Worker]
# state.py (extracto)
def process_flows_worker():
    while True:
        try:
            flow_data = global_flow_queue.get(timeout=2.0)
            if global_state_instance:
                global_state_instance.add_real_flow_direct(flow_data)
            global_flow_queue.task_done()
        except queue.Empty:
            if not global_capturing:
                break

def add_real_flow_direct(self, flow_data):
    prediction, probability = cic_predictor.predict(flow_data)
    flow = {
        "timestamp": datetime.now().strftime("%H:%M:%S"),
        "src_ip": flow_data.get('src_ip','N/A'),
        "dst_ip": flow_data.get('dst_ip','N/A'),
        "protocol": flow_data.get('protocol','Unknown'),
        "prediction": prediction,
        "probability": probability,
        # ... resto de campos mostrados por la UI
    }
    self.flows.append(flow)
    if prediction == "Malicious":
        self.add_alert(flow)
\end{lstlisting}
\paragraph{Explicación}
El escritor del capturador encola cada flujo. Un \emph{worker} dedicado consume la cola y llama a la predicción. El resultado se refleja en el dashboard y, en caso de malicioso, genera una alerta.


\section{Métodos de diseño e implementación de la interfaz de usuario}
La UI se implementa con \textbf{Reflex}~\cite{rootstackReflex} en \texttt{ids\_web.py}.Reflex compila el frontend en una aplicación Next.js de una sola página y la sirve en un puerto (por defecto 3000) al que puedes acceder en tu navegador. La función del frontend es reflejar el estado de la aplicación y enviar eventos al backend cuando el usuario interactúa con la interfaz de usuario.

En Reflex, solo el frontend se compila en Javascript y se ejecuta en el navegador del usuario, mientras que todo el estado y la lógica permanecen en Python y se ejecutan en el servidor. Cuando iniciamos un servidor FastAPI (por defecto en el puerto 8000) al que se conecta el frontend a través de un websocket: \url{https://fastapi.tiangolo.com/}.

La página principal incluye:
\begin{itemize}
  \item \textbf{Cabecera}: título, estado (Online/Offline) y acciones (Iniciar/Parar/Limpiar).
  \item \textbf{KPIs}: total de \textit{flows}, alertas, normales y ataques.
  \item \textbf{Tráfico}: lista en vivo con botón de detalles.
  \item \textbf{Alertas}: panel con severidad, probabilidad y navegación a detalles.
  \item \textbf{Configuración}: interfaz de red y filtro de captura.
  \item \textbf{Auto‑refresh}: temporizador cada 2 s cuando \texttt{capturing=True}.
\end{itemize}

\begin{lstlisting}[style=tfgpython,caption={Auto‑refresh y cabecera },label=List.ReflexHeader]
rx.moment(interval=2000, on_change=State.refresh_data)
rx.hstack(
  rx.badge("Online") if State.capturing else rx.badge("Offline"),
  rx.button("Parar", on_click=State.stop_capture) if State.capturing
           else rx.button("Iniciar", on_click=State.start_capture),
)
\end{lstlisting}

\subsection*{Componente de configuración y flujo operativo UI$\rightarrow$Estado$\rightarrow$Hilos}
La configuración permite elegir interfaz (\texttt{interface}) y filtro BPF (\texttt{bpf\_filter}); el botón \textit{Iniciar} valida ambos, carga artefactos si es necesario y lanza los hilos de \textbf{captura} y \textbf{proceso}. El temporizador de auto\-refresh sólo funciona cuando \texttt{capturing=True}. El botón \textit{Parar} sincroniza el cierre y \textit{Limpiar} vacía flujos/alertas.

\begin{lstlisting}[style=tfgpython,caption={Estado mínimo para configuración y captura},label=List.ConfigState]
class State(rx.State):
    interface: str = "eth0"
    bpf_filter: str = ""
    capturing: bool = False
    model_loaded: bool = False

    def set_interface(self, iface: str):
        self.interface = iface

    def set_bpf(self, filt: str):
        self.bpf_filter = filt

    def start_capture(self):
        if self.capturing:
            return
        assert self.interface in list_interfaces(), "Interfaz no válida"
        assert validate_bpf(self.bpf_filter), "Filtro BPF inválido"
        if not self.model_loaded:
            load_artifacts()  # scaler, modelo y mapeos
            self.model_loaded = True
        launch_capture_threads(self.interface, self.bpf_filter)
        self.capturing = True

    def stop_capture(self):
        if not self.capturing:
            return
        signal_stop_capture()
        join_capture_threads()
        self.capturing = False

    def clear_data(self):
        clear_flows_and_alerts()
\end{lstlisting}

\begin{lstlisting}[style=tfgpython,caption={Utilidades: listar interfaces y validar filtros BPF},label=List.ConfigUtils]
import subprocess
from scapy.all import get_if_list

def list_interfaces():
    try:
        return get_if_list()
    except Exception:
        return []

def validate_bpf(bpf: str) -> bool:
    if not bpf or not bpf.strip():  # vacío = sin filtro
        return True
    try:
        res = subprocess.run(
            ["tcpdump", "-ddd", bpf],
            stdout=subprocess.PIPE, stderr=subprocess.PIPE,
            text=True, timeout=3
        )
        return res.returncode == 0
    except Exception:
        bad = set(";|&$`")
        return not any(ch in bad for ch in bpf)
\end{lstlisting}

\subsection*{Diagrama de alto nivel del pipeline}
\begin{center}
\resizebox{\linewidth}{!}{%
\begin{tikzpicture}
  % Cajas con ancho controlado
  \node[draw,rounded corners,fill=gray!10,inner sep=6pt,text width=3.6cm,align=center] (cap)  at (0,0)   {Captura\\(Scapy + BPF)};
  \node[draw,rounded corners,fill=gray!10,inner sep=6pt,text width=4.8cm,align=center] (agg)  at (5.5,0)  {Agregación en flows\\(CICFlowMeter)};
  \node[draw,rounded corners,fill=gray!10,inner sep=6pt,text width=5.1cm,align=center] (pred) at (12.5,0) {CICIDSPredictor\\(Scaler + Random Forest)};
  \node[draw,rounded corners,fill=gray!10,inner sep=6pt,text width=3.6cm,align=center] (ui)   at (19.2,0) {UI (Reflex)};

  % Flechas
  \draw[->] (cap) -- (agg);
  \node[fill=white,inner sep=1pt] at ($(cap)!0.5!(agg)+(0,0.6)$) { };

  \draw[->] (agg) -- (pred);
  \node[fill=white,inner sep=1pt] at ($(agg)!0.5!(pred)+(0,0.65)$) {83 features};

  \draw[->] (pred) -- (ui);
  \node[fill=white,inner sep=1pt] at ($(pred)!0.5!(ui)+(0,0.65)$) {etiqueta, prob.};
\end{tikzpicture}%
}
\end{center}

\section{Reproducibilidad y despliegue}
\begin{itemize}
  \item \textbf{Dependencias}: \texttt{gpu\_deps.py} intenta instalar scikit‑learn, pandas, numpy, joblib, imbalanced‑learn, extensiones Intel y paquetes RAPIDS/cuML (si están disponibles).
  \item \textbf{Setup}: \texttt{setup\_system.py} guía el entrenamiento (GPU/CPU/muestreo/unificado) y verifica los artefactos (\texttt{.pkl}).
  \item \textbf{Artefactos versionados}: \texttt{cic\_ids\_model.pkl}, \texttt{cic\_ids\_scaler.pkl}, \texttt{feature\_mapping.pkl}, \texttt{cic\_features.pkl}.
  \item \textbf{Ejecución}: \texttt{reflex run} lanza el dashboard.
\end{itemize}

\begin{lstlisting}[caption={Ejecución típica desde terminal},language=bash,label=List.Run]
python setup_system.py       # Entrena y guarda artefactos
reflex run                   # Levanta el dashboard web
\end{lstlisting}

\section{Configuración experimental y métricas}
\textbf{Escenarios}: tráfico normal (HTTP/HTTPS/DNS/ICMP), escaneos \texttt{nmap} (\texttt{-sS/-sT/-sU}) y reproducción de ataques de CIC‑IDS mediante pcaps o generación. \\
\textbf{Métricas de sistema}: pps, \% drops, tamaño/latencia de cola, latencia de inferencia, ratio de alertas. \\
\textbf{Métricas del modelo}: accuracy, precisión, recall, F1, AUC, matriz de confusión.

\subsection*{Medición de latencias y rendimiento}
\begin{lstlisting}[style=tfgpython,caption={Medición rápida de latencia de inferencia},label=List.MetricsLatency]
import time
def timed_predict(flow_dict):
    t0 = time.perf_counter()
    label, proba = cic_predictor.predict(flow_dict)
    t1 = time.perf_counter()
    return label, proba, (t1 - t0) * 1000.0  # ms
\end{lstlisting}

\subsection*{Selección de umbral y severidad de alertas}
\begin{lstlisting}[style=tfgpython,caption={Mapa simple de severidad en función de probabilidad},label=List.AlertSeverity]
def alert_severity(label: str, proba: float) -> str:
    if label != "Malicious": return "Info"
    if proba >= 0.95: return "Critical"
    if proba >= 0.80: return "High"
    if proba >= 0.60: return "Medium"
    return "Low"
\end{lstlisting}

\subsection*{Validación temporal y robustez}
Además del \textit{train/test} aleatorio, se recomienda una validación \emph{leave‑one‑day‑out} (entrenar con varios días y evaluar en un día no visto) para medir \emph{domain shift} por fecha/escenario.

\subsection*{Importancia de características y trazabilidad}
\begin{lstlisting}[style=tfgpython,caption={Exportar top‑20 importancias de características},label=List.TopImportances]
import numpy as np
def top_feature_importances(model, feature_names, k=20):
    imps = np.asarray(model.feature_importances_)
    idx = np.argsort(imps)[::-1][:k]
    return [(feature_names[i], float(imps[i])) for i in idx]
\end{lstlisting}

\subsection*{Auditoría de predicciones maliciosas}
\begin{lstlisting}[style=tfgpython,caption={Registro de auditoría opcional},label=List.AuditLog]
import json, time
def audit_malicious(flow, model_version: str):
    rec = {"ts": time.time(), "model": model_version, "flow": flow}
    with open("ids_audit.log", "a", encoding="utf-8") as f:
        f.write(json.dumps(rec) + "\n")
\end{lstlisting}

\section{Riesgos, limitaciones y consideraciones éticas}
En el desarrollo de este TFG he identificado varios aspectos que condicionan el comportamiento y la explotación del sistema:

\begin{itemize}
  \item \textbf{Generalización del modelo.} El entrenamiento se ha realizado con datos CIC--IDS. Aunque es un referente académico, su distribución no coincide siempre con la de una red productiva; por tanto, cabe esperar pérdida de rendimiento fuera del dominio de entrenamiento. Para mitigar este riesgo, he mantenido el pipeline de datos desacoplado (permite reentrenar con nuevas capturas) y he documentado el procedimiento de actualización del modelo.

  \item \textbf{Desbalanceo y sesgos de clase.} La proporción de tráfico benigno frente a algunas familias de ataque es muy desigual. He aplicado un balanceo híbrido (submuestreo + SMOTE) y he monitorizado métricas por clase (recall/precision) para evitar optimizar únicamente la \emph{accuracy}. Aun así, pueden persistir sesgos si aparecen ataques no representados.

  \item \textbf{Privacidad y protección de datos.} La exportación de flujos contiene metadatos de red (IP de origen/destino, puertos y marcas temporales). En los ficheros de análisis y las evidencias adjuntas he priorizado la anonimización o el uso en entornos aislados. En casos de demostración pública, se recomienda enmascarar IPs y limitar la persistencia.

  \item \textbf{Rendimiento y robustez.} El capturador funciona en tiempo real y podría sufrir picos (tráfico a ráfagas). Para evitar pérdidas he implementado límites y limpieza periódica de flujos (\texttt{EXPIRED\_UPDATE}, \texttt{FLOW\_DURATION}, \texttt{MAX\_COLLECT\_PACKETS}) y un \emph{worker} dedicado al procesado. Aun así, en entornos de alto caudal sería necesario dimensionar hardware y/o activar mecanismos de \emph{backpressure} y muestreo.
\end{itemize}
\clearpage\thispagestyle{empty}\cleardoublepage

\chapter{RESULTADOS}\label{ch:resultados}

\section{Evaluación del Modelo de Detección de Intrusiones}

En este capítulo se presentan los resultados obtenidos de la evaluación del modelo Random Forest entrenado sobre el dataset CIC-IDS2018. El análisis comprende métricas generales de rendimiento, evaluación por tipo de ataque específico y un estudio detallado de las limitaciones identificadas.

\subsection{Métricas Generales de Rendimiento}

El modelo fue evaluado sobre un conjunto de prueba de \textbf{309,523 muestras}, manteniendo la proporción original del dataset balanceado. La Tabla \ref{tab:main_results} presenta las métricas principales obtenidas.

\begin{table}[H]
\centering
\begin{tabular}{lr}
\toprule
\textbf{Métrica} & \textbf{Valor} \\
\midrule
Accuracy & 0.9293 (92.93\%) \\
Precision & 0.9981 (99.81\%) \\
Recall & 0.9185 (91.85\%) \\
F1-Score & 0.9566 (95.66\%) \\
AUC-ROC & 0.9851 (98.51\%) \\
\bottomrule
\end{tabular}
\caption{Métricas principales del modelo Random Forest en CIC-IDS2018}
\label{tab:main_results}
\end{table}

Los resultados demuestran un rendimiento excepcional del modelo, con una precisión del 99.81\% que indica una tasa extremadamente baja de falsos positivos. El valor de AUC-ROC de 0.9851 confirma la excelente capacidad discriminatoria del modelo entre tráfico benigno y malicioso.

\subsection{Reporte de Clasificación Detallado}

La Tabla \ref{tab:classification_report} presenta el reporte detallado de clasificación por clase:

\begin{table}[H]
\centering
\begin{tabular}{lrrrr}
\toprule
\textbf{Clase} & \textbf{Precision} & \textbf{Recall} & \textbf{F1-Score} & \textbf{Support} \\
\midrule
Benign & 0.6832 & 0.9902 & 0.8085 & 46,667 \\
Attack & 0.9981 & 0.9185 & 0.9566 & 262,856 \\
\midrule
\textbf{Accuracy} & & & \textbf{0.9293} & \textbf{309,523} \\
\textbf{Macro avg} & \textbf{0.8407} & \textbf{0.9544} & \textbf{0.8826} & \textbf{309,523} \\
\textbf{Weighted avg} & \textbf{0.9506} & \textbf{0.9293} & \textbf{0.9343} & \textbf{309,523} \\
\bottomrule
\end{tabular}
\caption{Reporte de clasificación detallado del modelo Random Forest}
\label{tab:classification_report}
\end{table}

\subsection{Matriz de Confusión}

La matriz de confusión detallada (Tabla \ref{tab:confusion_matrix}) proporciona una visión granular del comportamiento del modelo:

\begin{table}[H]
\centering
\begin{tabular}{c|cc|c}
\toprule
& \multicolumn{2}{c|}{\textbf{Predicción}} & \\
\textbf{Real} & Benign & Attack & \textbf{Total} \\
\midrule
Benign & 46,211 & 456 & 46,667 \\
Attack & 21,428 & 241,428 & 262,856 \\
\midrule
\textbf{Total} & 67,639 & 241,884 & 309,523 \\
\bottomrule
\end{tabular}
\caption{Matriz de confusión del modelo Random Forest}
\label{tab:confusion_matrix}
\end{table}

La Figura \ref{fig:confusion_matrix} proporciona una visualización más intuitiva de estos resultados mediante un mapa de calor que facilita la interpretación de los datos.

\begin{figure}[H]
\centering
\includegraphics[width=0.8\textwidth]{imagenes/diagramas/resultados/confusion_matrix.png}
\caption{Visualización de la matriz de confusión del modelo Random Forest.}
\label{fig:confusion_matrix}
\end{figure}
\paragraph{Explicación}
Los colores más intensos indican mayor concentración de predicciones. Se observa una alta concentración en la diagonal principal (predicciones correctas) con 46,211 casos benignos correctamente clasificados y 241,428 ataques detectados correctamente. Los falsos positivos (456) y falsos negativos (21,428) aparecen en menor intensidad fuera de la diagonal.

\textbf{Análisis de errores:}
\begin{itemize}
\item \textbf{Falsos positivos}: 456 casos (0.98\% del tráfico benigno)
\item \textbf{Falsos negativos}: 21,428 casos (8.15\% de los ataques)
\item \textbf{Tasa de error global}: 7.07\%
\end{itemize}

El modelo muestra una tendencia conservadora, priorizando la minimización de falsos positivos a costa de algunos falsos negativos, lo cual es apropiado para sistemas de detección de intrusiones en entornos de producción.

\subsection{Curvas de Rendimiento}

\paragraph{Curva ROC}

La Figura \ref{fig:roc_curve} muestra la curva ROC del modelo, que ilustra la relación entre la tasa de verdaderos positivos y la tasa de falsos positivos.

\begin{figure}[H]
\centering
\includegraphics[width=0.8\textwidth]{imagenes/diagramas/resultados/roc_curve.png}
\caption{Curva ROC del modelo Random Forest.}
\label{fig:roc_curve}
\end{figure}

\paragraph{Explicación}
El área bajo la curva (AUC = 0.9851) indica un rendimiento excelente. La curva se aproxima mucho a la esquina superior izquierda, lo que demuestra que el modelo logra altas tasas de detección (verdaderos positivos) manteniendo bajas tasas de falsos positivos. La línea diagonal representa el rendimiento de un clasificador aleatorio.

\paragraph{Curva Precision-Recall}

La Figura \ref{fig:precision_recall} presenta la curva Precision-Recall, especialmente relevante para datasets desbalanceados como el de detección de intrusiones.

\begin{figure}[H]
\centering
\includegraphics[width=0.8\textwidth]{imagenes/diagramas/resultados/precision_recall_curve.png}
\caption{Curva Precision-Recall del modelo Random Forest.}
\label{fig:precision_recall}
\end{figure}

\paragraph{Explicación}
Esta curva es particularmente importante en detección de intrusiones debido al desbalance de clases. El modelo mantiene alta precisión (>95\%) incluso con altos valores de recall, indicando que puede detectar la mayoría de ataques sin generar excesivas alarmas falsas. El área bajo esta curva complementa la información de la curva ROC.

\subsection{Distribución de Probabilidades}

La Figura \ref{fig:probability_distribution} muestra cómo el modelo asigna probabilidades a las diferentes clases, proporcionando insights sobre la confianza de las predicciones.

\begin{figure}[H]
\centering
\includegraphics[width=0.9\textwidth]{imagenes/diagramas/resultados/probability_distribution.png}
\caption{Distribución de probabilidades por clase real del modelo Random Forest.}
\label{fig:probability_distribution}
\end{figure}

\paragraph{Explicación}
La gráfica revela un comportamiento característico: el tráfico benigno (verde) se concentra fuertemente en probabilidades bajas (cercanas a 0), indicando alta confianza en su clasificación como no-ataque. Los ataques (rojo) muestran una distribución bimodal con una concentración masiva cerca de la probabilidad 1.0, reflejando la alta confianza del modelo para detectar la mayoría de ataques. El pico pronunciado del rojo en 1.0 explica la alta precisión del modelo (99.81\%).

Esta distribución explica varios aspectos clave del rendimiento:
\begin{itemize}
\item La \textbf{concentración verde en 0.0-0.2} indica que el modelo identifica claramente el tráfico benigno
\item El \textbf{pico rojo masivo en 1.0} demuestra alta confianza en la detección de ataques
\item La \textbf{separación clara} entre las distribuciones confirma la capacidad discriminatoria del modelo
\item Las \textbf{pequeñas superposiciones} en el rango medio (0.3-0.7) corresponden a casos de mayor incertidumbre
\end{itemize}

\subsection{Distribución del Dataset Completo}

El análisis por tipo de ataque se realizó sobre el dataset completo de \textbf{1,547,611 muestras} que incluye 9 tipos diferentes de ataques. La Tabla \ref{tab:dataset_distribution} muestra la distribución original:

\begin{table}[H]
\centering
\begin{tabular}{lrr}
\toprule
\textbf{Tipo de Ataque} & \textbf{Muestras} & \textbf{Porcentaje} \\
\midrule
Benign & 233,332 & 15.08\% \\
FTP-BruteForce & 193,360 & 12.49\% \\
SSH-Bruteforce & 187,589 & 12.12\% \\
Brute Force Web & 155,555 & 10.05\% \\
Brute Force XSS & 155,555 & 10.05\% \\
DoS attacks-Slowloris & 155,555 & 10.05\% \\
DoS attacks-GoldenEye & 155,555 & 10.05\% \\
Infiltration & 155,555 & 10.05\% \\
SQL Injection & 155,555 & 10.05\% \\
\midrule
\textbf{Total} & \textbf{1,547,611} & \textbf{100.00\%} \\
\bottomrule
\end{tabular}
\caption{Distribución original del dataset CIC-IDS2018 balanceado}
\label{tab:dataset_distribution}
\end{table}

\subsection{Análisis por Tipo de Ataque}

La evaluación detallada por tipo de ataque revela variaciones significativas en la capacidad de detección del modelo (Tabla \ref{tab:attack_detection}):

\begin{table}[H]
\centering
\begin{tabular}{lrrrr}
\toprule
\textbf{Tipo de Ataque} & \textbf{Total} & \textbf{Detectado} & \textbf{Tasa} & \textbf{Confianza} \\
\midrule
FTP-BruteForce & 193,360 & 193,360 & 100.0\% & 0.998 \\
SSH-Bruteforce & 187,589 & 187,586 & 100.0\% & 0.999 \\
Brute Force XSS & 155,555 & 155,555 & 100.0\% & 0.989 \\
DoS GoldenEye & 155,555 & 155,554 & 100.0\% & 0.999 \\
SQL Injection & 155,555 & 155,521 & 100.0\% & 0.977 \\
Brute Force Web & 155,555 & 155,538 & 100.0\% & 0.977 \\
DoS Slowloris & 155,555 & 155,059 & 99.7\% & 0.995 \\
Benign & 233,332 & 232,200 & 99.5\% & 0.914 \\
Infiltration & 155,555 & 51,094 & 32.8\% & 0.445 \\
\midrule
\textbf{Promedio ponderado} & \textbf{1,547,611} & \textbf{1,441,467} & \textbf{93.14\%} & \textbf{0.911} \\
\bottomrule
\end{tabular}
\caption{Tasa de detección por tipo de ataque específico}
\label{tab:attack_detection}
\end{table}

La Figura \ref{fig:attack_analysis} proporciona una visualización comprehensiva del rendimiento por tipo de ataque en múltiples dimensiones.

\begin{figure}[H]
\centering
\includegraphics[width=1.0\textwidth]{imagenes/diagramas/resultados/attack_type_analysis.png}
\caption{Análisis completo por tipo de ataque del modelo Random Forest.}
\label{fig:attack_analysis}
\end{figure}

\paragraph{Explicación}
El panel superior izquierdo muestra las tasas de detección, donde se observa un rendimiento excelente (>99\%) para la mayoría de ataques excepto infiltración (32.8\%). El panel superior derecho presenta la distribución de muestras, mostrando el balanceamiento del dataset. El panel inferior izquierdo ilustra la confianza promedio del modelo, donde infiltración muestra la menor confianza (0.445). El panel inferior derecho destaca los errores absolutos, donde infiltración presenta significativamente más errores que otros tipos.
 
La Figura \ref{fig:attack_heatmap} complementa este análisis con un mapa de calor que facilita la comparación entre tipos de ataque.

\begin{figure}[H]
\centering
\includegraphics[width=0.9\textwidth]{imagenes/diagramas/resultados/attack_performance_heatmap.png}
\caption{Mapa de calor del rendimiento por tipo de ataque.}
\label{fig:attack_heatmap}
\end{figure}

\paragraph{Explicación}
Los colores más cálidos (rojos/naranjas) indican mejor rendimiento. Se observa claramente que los ataques de fuerza bruta y DoS muestran rendimiento excelente (colores rojos intensos), mientras que infiltración presenta rendimiento deficiente (colores más fríos). Las tres métricas mostradas son: tasa de detección, confianza promedio y volumen de muestras normalizado.

\subsection{Análisis de Errores por Tipo de Ataque}

La Tabla \ref{tab:error_analysis} presenta un desglose detallado de los errores por tipo de ataque:

\begin{table}[H]
\centering
\begin{tabular}{lrrr}
\toprule
\textbf{Tipo de Ataque} & \textbf{Total Muestras} & \textbf{Errores} & \textbf{Tasa de Error} \\
\midrule
Benign & 233,332 & 1,132 & 0.5\% \\
SSH-Bruteforce & 187,589 & 3 & 0.0\% \\
Brute Force Web & 155,555 & 17 & 0.0\% \\
DoS attacks-Slowloris & 155,555 & 496 & 0.3\% \\
DoS attacks-GoldenEye & 155,555 & 1 & 0.0\% \\
SQL Injection & 155,555 & 34 & 0.0\% \\
Infiltration & 155,555 & 104,461 & 67.2\% \\
FTP-BruteForce & 193,360 & 0 & 0.0\% \\
Brute Force XSS & 155,555 & 0 & 0.0\% \\
\midrule
\textbf{Total} & \textbf{1,547,611} & \textbf{106,144} & \textbf{6.86\%} \\
\bottomrule
\end{tabular}
\caption{Análisis de errores por tipo de ataque}
\label{tab:error_analysis}
\end{table}

\textbf{Categorización del rendimiento:}

\textbf{Excelente detección ($\geq$99.5\%):}
\begin{itemize}
\item Ataques de fuerza bruta (FTP, SSH, Web, XSS): 100\% de detección
\item Ataques de denegación de servicio: 99.7-100\%
\item Inyección SQL: 100\%
\item Tráfico benigno: 99.5\%
\end{itemize}

\textbf{Detección problemática (<50\%):}
\begin{itemize}
\item Ataques de infiltración: 32.8\% de detección (67.2\% de errores)
\end{itemize}

\subsection{Importancia de Características}

El análisis de importancia de características identifica los atributos más relevantes para la clasificación. La Tabla \ref{tab:feature_importance} presenta las 20 características más importantes:

\begin{table}[H]
\centering
\begin{tabular}{rlr}
\toprule
\textbf{Ranking} & \textbf{Característica} & \textbf{Importancia} \\
\midrule
1 & Init Fwd Win Byts & 0.1094 \\
2 & Fwd Seg Size Min & 0.1020 \\
3 & Fwd Header Len & 0.0505 \\
4 & Flow IAT Min & 0.0459 \\
5 & Init Bwd Win Byts & 0.0333 \\
6 & Bwd Pkts/s & 0.0279 \\
7 & Fwd Pkt Len Std & 0.0256 \\
8 & Subflow Fwd Byts & 0.0255 \\
9 & TotLen Fwd Pkts & 0.0253 \\
10 & Fwd Pkt Len Mean & 0.0230 \\
11 & Flow Byts/s & 0.0224 \\
12 & Fwd Pkts/s & 0.0217 \\
13 & Flow IAT Max & 0.0201 \\
14 & Fwd IAT Min & 0.0194 \\
15 & Flow IAT Mean & 0.0193 \\
16 & Flow IAT Std & 0.0191 \\
17 & Flow Pkts/s & 0.0190 \\
18 & Tot Fwd Pkts & 0.0174 \\
19 & Tot Bwd Pkts & 0.0174 \\
20 & Fwd Act Data Pkts & 0.0169 \\
\bottomrule
\end{tabular}
\caption{Top 20 características más importantes según Random Forest}
\label{tab:feature_importance}
\end{table}

La Figura \ref{fig:feature_importance} visualiza estas importancias de manera más intuitiva.

\begin{figure}[H]
\centering
\includegraphics[width=0.9\textwidth]{imagenes/diagramas/resultados/feature_importance.png}
\caption{Importancia de características del modelo Random Forest.}
\label{fig:feature_importance}
\end{figure}

\paragraph{Explicación}
El gráfico muestra una distribución desigual donde las dos características principales (Init Fwd Win Byts y Fwd Seg Size Min) dominan con importancias de 10.94\% y 10.20\% respectivamente. Esto indica que el modelo se basa fuertemente en características relacionadas con el tamaño de ventana inicial y la segmentación de paquetes para distinguir entre tráfico benigno y malicioso. La disminución gradual sugiere que muchas características aportan información complementaria.

Las características relacionadas con el \textbf{tamaño de ventana inicial} y la \textbf{segmentación de paquetes} emergen como los discriminadores más potentes, representando conjuntamente el 21.14\% de la importancia total del modelo.

\subsection{Análisis Específico: Limitaciones en Ataques de Infiltración}

Dado el rendimiento significativamente inferior en la detección de ataques de infiltración (32.8\%), se realizó un análisis específico para identificar las causas subyacentes.

\paragraph{Distribución de Probabilidades}

El análisis de las probabilidades asignadas por el modelo reveló una distribución problemática:

\begin{itemize}
\item \textbf{Infiltración}: Promedio = 0.445, Desviación estándar = 0.327
\item \textbf{Tráfico benigno}: Promedio = 0.083, Desviación estándar = 0.090
\end{itemize}

La alta desviación estándar en ataques de infiltración (0.327) indica una \textbf{incertidumbre significativa} del modelo, contrastando con la confianza mostrada en tráfico benigno.

\paragraph{Análisis de Umbrales de Clasificación}

La evaluación bajo diferentes umbrales de clasificación (Tabla \ref{tab:threshold_analysis}) sugiere oportunidades de optimización:

\begin{table}[H]
\centering
\begin{tabular}{ccc}
\toprule
\textbf{Umbral} & \textbf{Detección Infiltración} & \textbf{Falsos Positivos Benigno} \\
\midrule
0.9 & 22.8\% & 0.1\% \\
0.7 & 26.4\% & 0.1\% \\
0.5 (actual) & 32.8\% & 0.3\% \\
0.3 & 53.6\% & 2.9\% \\
\bottomrule
\end{tabular}
\caption{Impacto del umbral de clasificación en la detección de infiltración}
\label{tab:threshold_analysis}
\end{table}

La Figura \ref{fig:infiltration_analysis} proporciona un análisis visual comprehensivo de las limitaciones en la detección de infiltración.

\begin{figure}[H]
\centering
\includegraphics[width=1.0\textwidth]{imagenes/diagramas/resultados/infiltration_analysis.png}
\caption{Análisis detallado de ataques de infiltración.}
\label{fig:infiltration_analysis}
\end{figure}

\paragraph{Explicación}
El panel superior izquierdo muestra las distribuciones de probabilidad superpuestas: infiltración (rojo) presenta una distribución más dispersa con menor concentración en valores altos, mientras que benigno (verde) se concentra fuertemente cerca de 0. El panel superior derecho ilustra cómo diferentes umbrales afectan la detección: umbrales más bajos mejoran la detección de infiltración. El panel inferior izquierdo identifica las características más distintivas, donde las diferencias temporales son predominantes. El boxplot inferior derecho confirma la mayor incertidumbre del modelo para infiltración.

Un \textbf{ajuste del umbral de 0.5 a 0.3} resultaría en:
\begin{itemize}
\item \textbf{Mejora del 63.4\%} en detección de infiltración (de 32.8\% a 53.6\%)
\item \textbf{Incremento tolerable} de falsos positivos (de 0.3\% a 2.9\%)
\end{itemize}

\paragraph{Características Distintivas de Infiltración}

El análisis comparativo entre ataques de infiltración y tráfico benigno identificó las 10 diferencias temporales más significativas (Tabla \ref{tab:infiltration_features}):

\begin{table}[H]
\centering
\begin{tabular}{lrrr}
\toprule
\textbf{Característica} & \textbf{Infiltración} & \textbf{Benigno} & \textbf{Diferencia} \\
\midrule
Fwd IAT Tot ($\mu$s) & 10,027,847 & 14,597,875 & -4,570,028 \\
Flow Duration ($\mu$s) & 10,461,854 & 14,964,644 & -4,502,789 \\
Flow IAT Max ($\mu$s) & 3,752,259 & 7,276,331 & -3,524,072 \\
Fwd IAT Max ($\mu$s) & 3,536,283 & 7,027,942 & -3,491,659 \\
Fwd IAT Mean ($\mu$s) & 1,308,751 & 4,475,634 & -3,166,882 \\
Flow IAT Mean ($\mu$s) & 1,021,423 & 4,121,081 & -3,099,658 \\
Fwd IAT Min ($\mu$s) & 622,602 & 3,697,850 & -3,075,248 \\
Flow IAT Min ($\mu$s) & 558,770 & 3,604,459 & -3,045,688 \\
Idle Min ($\mu$s) & 3,130,105 & 5,176,845 & -2,046,740 \\
Idle Mean ($\mu$s) & 3,346,631 & 5,362,075 & -2,015,444 \\
\bottomrule
\end{tabular}
\caption{Top 10 características temporales distintivas de ataques de infiltración}
\label{tab:infiltration_features}
\end{table}

\textbf{Hallazgo clave}: Los ataques de infiltración se caracterizan por \textbf{patrones temporales más compactos} que el tráfico benigno, con duraciones de flujo y intervalos entre paquetes sistemáticamente menores. Esta similitud con ciertos patrones de tráfico legítimo explica la dificultad del modelo para distinguirlos.

\subsection{Rendimiento Computacional}\label{res:rendimientocom}

El modelo demostró una eficiencia computacional adecuada para despliegue en producción:

\begin{itemize}
\item \textbf{Velocidad de procesamiento}: 152,761 muestras/segundo
\item \textbf{Tiempo promedio por muestra}: 0.007 ms
\item \textbf{Tiempo total para 309,523 muestras}: 2.028 segundos
\end{itemize}

Estos valores indican que el modelo es viable para análisis en tiempo real de tráfico de red de alta velocidad.

\section{Síntesis de Resultados}

El modelo Random Forest desarrollado demuestra un \textbf{rendimiento excepcional} en la detección de intrusiones, con una precisión global del 92.93\% y una precisión del 99.81\%. Las fortalezas principales incluyen:

\begin{enumerate}
\item \textbf{Detección perfecta} de ataques de fuerza bruta y denegación de servicio
\item \textbf{Tasa extremadamente baja} de falsos positivos (0.98\%)
\item \textbf{Eficiencia computacional} adecuada para tiempo real
\item \textbf{Identificación clara} de características discriminatorias
\end{enumerate}

La \textbf{limitación principal} radica en la detección de ataques de infiltración (32.8\%), atribuible a su similitud temporal con tráfico benigno. No obstante, esta limitación puede mitigarse mediante \textbf{ajuste de umbrales} o \textbf{enriquecimiento del conjunto de características} con información contextual adicional.

Los resultados posicionan al modelo como una \textbf{solución robusta y práctica} para sistemas de detección de intrusiones, con un balance favorable entre efectividad de detección y tasa de falsos positivos.

\subsection{Archivos de Soporte Generados}

Los siguientes archivos gráficos fueron generados para complementar el análisis:

\begin{itemize}
\item \texttt{confusion\_matrix.png} - Visualización de la matriz de confusión
\item \texttt{roc\_curve.png} - Curva ROC del modelo
\item \texttt{precision\_recall\_curve.png} - Curva Precision-Recall
\item \texttt{feature\_importance.png} - Importancia de características
\item \texttt{probability\_distribution.png} - Distribución de probabilidades por clase
\item \texttt{attack\_type\_analysis.png} - Análisis comparativo por tipo de ataque
\item \texttt{infiltration\_analysis.png} - Análisis específico de infiltración
\item \texttt{attack\_performance\_heatmap.png} - Mapa de calor del rendimiento
\end{itemize}
\clearpage\thispagestyle{empty}\cleardoublepage

\chapter{CONCLUSIONES}

\section{Conclusión final del trabajo}
Este TFG planteó como \textbf{objetivo general} la creación de una \textbf{herramienta de captura en tiempo real} capaz de reconstruir flujos, extraer características y \textbf{generar \textit{datasets} exportables} en formatos (\emph{CSV/TXT}) compatibles con esquemas tipo CIC; y, como demostrador, un \textbf{prototipo de IDS} con un modelo de ML operando en tiempo real. A continuación, se sintetiza el grado de cumplimiento con evidencias:

\begin{itemize}
  \item \textbf{Captura y exportación de datos (cumplido).} Se desarrolló el capturador (sniffer + \textit{FlowSession} + \textit{features}) y la \textbf{exportación a (\emph{CSV/TXT})} invocable desde la consola o terminal de comandos, permitiendo la construcción sistemática de \textit{datasets}. Este era el objetivo prioritario del trabajo y quedó implementado.
  \item \textbf{Modelo de ML integrado (cumplido).} Se entrenó e integró un modelo \textbf{Random Forest} con el conjunto de datos \textbf{CIC-IDS2018} (tras estudiar otros datasets en el estado del arte). La evaluación muestra \textbf{Accuracy 0.9293}, \textbf{Precision 0.9981}, \textbf{Recall 0.9185}, \textbf{F1 0.9566} y \textbf{AUC-ROC 0.9851} (Tabla~\ref{tab:main_results}); véanse también el \textit{reporte de la clasificación} (Tabla~\ref{tab:classification_report}) y la \textbf{curva ROC} (Figura~\ref{fig:roc_curve}), entre otros.
  \item \textbf{Análisis por tipo de ataque (cumplido y documentado).} Se obtuvo \textbf{detección prácticamente perfecta} en fuerza bruta, DoS y SQLi, y \textbf{limitaciones} en \textit{Infiltration} (32.8\% de detección) (Tabla~\ref{tab:attack_detection}, Figura~\ref{fig:attack_heatmap}). Se analizó la causa (distribuciones y \textbf{umbrales}), proponiendo que bajar el umbral a 0.3 eleva la detección al \textbf{53.6\%} con aumento controlado de FP (Tabla~\ref{tab:threshold_analysis}, Figura~\ref{fig:infiltration_analysis}).
  \item \textbf{Viabilidad operativa (cumplido).} El prototipo alcanza \textbf{152,761 muestras/s} y \textbf{2.028 s} para 309,523 muestras, lo que evidencia \textbf{viabilidad en tiempo prácticamente real}. Véase en el apartado (Resultados, ~\ref{res:rendimientocom}).
  \item \textbf{Observabilidad y UI (cumplido).} Se implementó un \textbf{dashboard} con métricas, alertas y control de captura, cerrando el \textit{loop} de operación del IDS.
\end{itemize}

\paragraph{Alcance y decisiones}
- Aunque se valoraron múltiples datasets en el estado del arte como el CIC-IDS2017, CIC‑IDS2018, CIC‑IDS2019, CIC-DDoS2019, la \textbf{validación experimental} se centró en \textbf{CIC-IDS2018}, unificando ficheros para cubrir 9 tipos de ataque (Tabla~\ref{tab:dataset_distribution}). Esta decisión es coherente con el objetivo prioritario (capturador + generación de datasets) y permitió concentrar esfuerzos en la integración y la evaluación profunda.
- Se reportaron \textbf{más métricas y visualizaciones} de las inicialmente previstas (añadiendo, p.\,ej., análisis de distribución de probabilidades y \textbf{feature importance}, Figura~\ref{fig:feature_importance}), fortaleciendo la evidencia de desempeño.

\paragraph{Conclusión general}
El sistema construido cumple el objetivo central de \textbf{generar datasets reproducibles} desde captura en tiempo real y demuestra, mediante un \textbf{prototipo para un IDS} basado en Random Forest, una \textbf{detección robusta} con muy baja tasa de falsos positivos. Las limitaciones detectadas en \textit{Infiltration} están diagnosticadas y cuentan con \textbf{líneas claras de mejora} (ajuste de umbral y enriquecimiento de \emph{features}), lo que sitúa la solución como una \textbf{base sólida y operativa} para evolución futura.

\section{Valoración Personal}
Esta memoria, culmen de un trayecto académico de cuatro años, representa no solo la materialización de un proyecto final de grado, sino también el reflejo de un profundo aprendizaje, compromiso y constancia. Este Trabajo de Fin de Grado constituye el desafío final para la obtención del título de Ingeniería Informática, encapsulando la evolución formativa de toda la carrera.

Mi interés por el ámbito de la seguridad informática se manifestó incluso antes de iniciar mis estudios universitarios. La fascinación por comprender cómo los sistemas podían ser controlados y manipulados de diversas formas despertó en mí una constante inquietud por identificar y mitigar vulnerabilidades. Esta pasión inicial se consolidó durante el bachillerato, impulsada por el intercambio de conocimientos con un compañero con afinidad por la seguridad, lo que nos llevó a una inmersión más profunda en la investigación de este campo.

Ya en el grado, la asignatura de ``Seguridad en Tecnologías de la Información'' en segundo curso avivó aún más mi deseo de especializarme en esta disciplina. Desde la complejidad de vulnerabilidades a nivel de \textit{hardware} hasta la sencillez de otras detectables con una inspección somera en entornos web, cada aspecto me cautivó. Paralelamente, la introducción a la Inteligencia Artificial y las Metaheurísticas en cursos subsiguientes despertó un nuevo interés hacia estos modelos avanzados, capaces de emular el aprendizaje humano para optimizar resultados. Esta confluencia de intereses me llevó a la elección de un proyecto final de grado que integrara ambas disciplinas: ciberseguridad e inteligencia artificial. Adicionalmente, la estrecha relación entre la seguridad y el mundo de las redes, reforzada por la asignatura de ``Redes e Infraestructuras'', me llevó finalmente a decantarme por la implementación de un capturador de tráfico en tiempo real, concebido como la base para un sistema de detección de intrusiones.

Las etapas iniciales del desarrollo del capturador estuvieron marcadas por desafíos significativos, particularmente la dificultad para encontrar APIs y librerías que facilitaran la implementación. La frustración inicial por la falta de información relevante me llevó a explorar incluso el funcionamiento interno de herramientas como Wireshark. Sin embargo, la perseverancia fue clave, y con el tiempo, logré identificar una API que proporcionaba las funcionalidades necesarias para extraer una vasta cantidad de características del tráfico de red. Esta fase de implementación, aunque desafiante, resultó profundamente apasionante, demostrando que la constancia es fundamental para alcanzar todas nuestras metas y objetivos.

Posteriormente, el proyecto evolucionó hacia el diseño de una aplicación web sencilla para visualizar los flujos de paquetes capturados en tiempo real. Esta etapa presentó un nuevo reto al requerir la selección e integración de un modelo de \textit{Machine Learning} para la detección de intrusiones. Esto implicó una inmersión en los fundamentos del aprendizaje supervisado, decantándome por el modelo \textit{Random Forest}. La asimilación de conceptos de minería de datos, preprocesamiento, análisis y limpieza de datos supuso un esfuerzo considerable, dada la densidad de la información. No obstante, este proceso culminó con el entrenamiento exitoso del sistema y su posterior integración en la aplicación web.

En lo referente a la documentación, este TFG ha representado una valiosa experiencia de aprendizaje sobre los requerimientos y la extensión que conlleva una memoria técnica completa, un contraste notable con los documentos de menor envergadura redactados en cursos anteriores. Asimismo, ha sido mi primera incursión en el entorno \LaTeX~, una herramienta que, a pesar de una curva de aprendizaje inicial, ha demostrado ser sorprendentemente intuitiva y sus utilidades para la redacción de memorias técnicas son inestimables.

Para concluir esta valoración personal, deseo expresar mi profunda satisfacción por la formación multidisciplinar adquirida en campos como la ciberseguridad, la inteligencia artificial, las redes y el desarrollo web. Destaco especialmente el dominio del lenguaje de programación Python, esencial en la ciencia de datos y en el mundo de la ciberseguridad, que ha sido fundamental para el éxito y la culminación de este proyecto. Me siento inmensamente orgulloso de este logro, que representa un hito significativo en mi trayectoria académica y profesional.

\section{Posibles mejoras futuras}

El presente Sistema de Detección de Intrusiones (IDS) y la herramienta de generación de conjuntos de datos constituyen una base sólida para futuras investigaciones y desarrollos. Las siguientes líneas proponen una serie de mejoras y extensiones que podrían potenciar significativamente las capacidades y la robustez del sistema, abordando tanto aspectos técnicos como funcionales.

\begin{enumerate}

    \item\textbf{Diversificación y Optimización del Componente de \textit{Machine Learning}}:
    \begin{itemize}
    
        \item\textbf{Integración de Modelos Avanzados}: Explorar la implementación de otros algoritmos de clasificación supervisada como \textit{Support Vector Machines} (SVM), \textit{Gradient Boosting} (XGBoost, LightGBM) o Redes Neuronales Recurrentes (RNN) y Convolucionales (CNN), así como arquitecturas basadas en \textit{Deep Learning} (como \textit{Transformers}), las cuales han demostrado gran potencial en el análisis de secuencias de datos de red.
        
        \item\textbf{Modelos \textit{Ensemble} Híbridos}: Investigar la creación de modelos en \textit{ensemble} que combinen la robustez de algoritmos como \textit{Random Forest} (especialmente eficaz para la clasificación binaria de tráfico benigno/malicioso) con la capacidad de modelos de \textit{Deep Learning} para diferenciar entre tipos específicos de ataque, aprovechando las fortalezas de cada técnica.
        
        \item\textbf{Funcionalidad de Re-entrenamiento Continuo}: Desarrollar un mecanismo para permitir que los modelos de \textit{Machine Learning} sean re-entrenados periódicamente con nuevos datos capturados y etiquetados, asegurando que el sistema se adapte a nuevas amenazas y evoluciones en los patrones de tráfico.
    
    \end{itemize}

    \item\textbf{Expansión y Refinamiento del Módulo de Adquisición y Preprocesamiento}:

    \begin{itemize}

        \item\textbf{Recolección de Características Adicionales}: Ampliar el conjunto de características extraídas de los paquetes y flujos de red. Esto podría incluir métricas más complejas a nivel de aplicación, características temporales de las conexiones o datos específicos del \textit{payload}, enriqueciendo la capacidad de discriminación del modelo.
        
        \item\textbf{Manejo de Tráfico Cifrado}: Investigar y desarrollar métodos para la inspección y el tratamiento de paquetes cifrados (ej., TLS/SSL). Esto podría implicar el uso de funcionalidades de descifrado (similares a las presentes en herramientas como Wireshark, si se dispone de las claves) o el análisis de metadatos del tráfico cifrado para identificar anomalías sin violar la privacidad.
        
        \item\textbf{Normalización y Compatibilidad de Datos}: Mejorar los métodos de preprocesamiento para asegurar la compatibilidad y unificación de conjuntos de datos provenientes de diversas fuentes, facilitando la creación de \textit{datasets} más grandes y robustos para el entrenamiento.

    \end{itemize}

    \item\textbf{Mejoras en la Interfaz de Usuario y Funcionalidades de Gestión}:

    \begin{itemize}
    
        \item\textbf{Gestión de usuarios y control de acceso}: Incorporar autenticación y autorización con sesiones de usuario (usuario y contraseña), recuperación de credenciales y, en su caso, roles (administrador/analista/solo lectura) para acotar permisos. Almacenamiento seguro de contraseñas (hash con bcrypt/Argon2), expiración de sesión y protección CSRF.
        
        \item\textbf{Interfaz remota y operación en arquitectura cliente-servidor}: Desplegar el IDS en una máquina virtual/servidor y acceder mediante una interfaz web remota; separar la máquina que ejecuta la captura y el cómputo pesado de la máquina cliente desde la que se opera. Exponer servicios vía API y asegurar la comunicación extremo a extremo (TLS).
        
        \item\textbf{Seguridad y Privacidad de la Aplicación Web}: Implementar mecanismos robustos de seguridad para la aplicación web, incluyendo la encapsulación y cifrado de la información transmitida entre el cliente y el servidor, así como la gestión de autenticación y autorización de usuarios.
        
        \item\textbf{Visualización Interactiva y Alertas Avanzadas}: Desarrollar visualizaciones más dinámicas e interactivas del tráfico de red y las detecciones. Esto incluiría \textit{dashboards} personalizables, la capacidad de filtrar alertas por tipo, gravedad o tiempo, y notificaciones en tiempo real (ej., vía correo electrónico o plataformas de mensajería).
        
        \item\textbf{Gestión de Reglas Personalizadas}: Ofrecer la posibilidad a los usuarios de definir y gestionar sus propias reglas de detección, permitiendo una mayor adaptabilidad del sistema a entornos específicos o necesidades particulares.
        
        \item\textbf{API \textit{RESTful}}: Implementar una API \textit{RESTful} para permitir la integración del IDS con otras herramientas de seguridad, sistemas de gestión de eventos e información de seguridad (SIEM) o plataformas de automatización de respuesta a incidentes.

    \end{itemize}
    
    \item\textbf{Escalabilidad, Despliegue y Robustez del Sistema}:

    \begin{itemize}
 
        \item\textbf{Optimización del Rendimiento}: Mejorar el rendimiento del capturador y del motor de detección para manejar grandes volúmenes de tráfico de red en entornos de producción, optimizando el uso de recursos computacionales.
        
        \item\textbf{Contenerización y Orquestación}: Empaquetar los diferentes módulos del sistema en contenedores (ej., Docker) para facilitar el despliegue, la portabilidad y la escalabilidad del IDS en entornos de producción o en la nube, utilizando herramientas de orquestación como Kubernetes.
        
        \item\textbf{Pruebas de Resistencia y Ciberseguridad Ofensiva}: Realizar pruebas de estrés exhaustivas y aplicar técnicas de ciberseguridad ofensiva (ej., \textit{penetration testing}, ataques adversarios a los modelos ML) para evaluar la resiliencia del sistema y su capacidad para detectar amenazas complejas y en evolución.
        
       \item\textbf{Ejecución en Linux (pendiente)}: Garantizar el funcionamiento nativo en Linux y facilitar su instalación (dependencias y permisos básicos).
        
    \end{itemize}

\end{enumerate}
\clearpage\thispagestyle{empty}\cleardoublepage

\chapter{APÉNDICES}
\section{Instalación y configuración del sistema}

\subsection*{Requisitos previos}
\begin{itemize}
  \item \textbf{Python} 3.10+ (probado con 3.10/3.11)
  \item \textbf{pip} para gestión de paquetes
  \item \textbf{Permisos de captura}: en Linux, ejecutar con \texttt{sudo} o asignar capacidades a \texttt{python}/\texttt{tcpdump}; en Windows, abrir la terminal como administrador
  \item \textbf{Hardware recomendado}: 8GB RAM mínimo, 16GB recomendado para entrenamiento
  \item \textbf{Dataset CIC-IDS2018}: descargar y ubicar en ruta accesible (para entrenamiento avanzado)
\end{itemize}

\subsection*{Preparación del entorno}

A continuación se muestran rutas y comandos típicos. Adáptalo según su estructura concreta donde lo quiera ejecutar.
\begin{lstlisting}[style=tfgbash,caption={Configuración inicial del proyecto},label=List.EnvSetup]
# Clonar o descargar y acceder al proyecto
cd ids_web 

# Crear entorno virtual
python -m venv reflex_env

# Activar entorno virtual (Windows)
cd .. (hay que movernos una rama hacia atrás para activar el entorno virtual) para luego una vez activado, volveremos al proyecto: cd ids_web para ejecutar las siguientes órdenes que se explicarán a continuación para lanzar el proyecto
.\reflex_env\Scripts\activate     

# Activar entorno (Linux/Mac)
source reflex_env/bin/activate

# Actualizar pip
pip install -U pip
\end{lstlisting}

\subsection*{Instalación de dependencias}
Se puede considerar una opción, pero en la siguiente subsección se recomienda el método más cómodo y fácil de utilizar ~\ref{Sec:apendices_instalacion}
\begin{lstlisting}[style=tfgbash,caption={Instalar dependencias del proyecto},label=List.InstallDeps]
# Si el proyecto incluye requirements.txt (como lo incluye en este caso, lo ejecutamos)
pip install -r requirements.txt

# Instalar dependencias GPU (opcional)
pip install torch torchvision torchaudio --index-url https://download.pytorch.org/whl/cu118

# Alternativa con poetry (si se usa):
poetry install

# O ejecutar el configurador automático
python training/setup_system.py
\end{lstlisting}

\subsection*{Instalación y entrenamiento automático (Recomendado)}\label{Sec:apendices_instalacion}
\textbf{El método más simple es usar el configurador automático que instala dependencias y entrena el modelo:}

\begin{lstlisting}[style=tfgbash,caption={Configuración completa automática},label=List.AutoSetup]
# Ejecutar configurador automático 
python training/setup_system.py
\end{lstlisting}

Cabe destacar que todos los comandos de entrenamiento y procesado de paquetes se ejecutan en el directorio ~\texttt{ids\_web/ids\_web} ya que en el se encuentran todos los archivos necesarios para que funcione excepto el comando ~\texttt{reflex run} que se ejecuta un nodo más arriba en la jerarquía: ~\texttt{ids\_web/}.

\textbf{¿Qué hace \texttt{training/setup\_system.py}?}
\begin{itemize}
  \item \textbf{Instala automáticamente} todas las dependencias necesarias (GPU y CPU)
  \item \textbf{Detecta} si tienes GPU NVIDIA y configura aceleración.
  \item \textbf{Presenta menú interactivo} con opciones de entrenamiento.
  \item \textbf{Genera automáticamente} todos los artefactos del modelo en \texttt{models/}.
\end{itemize}

\textbf{Opciones disponibles en el menú:}
\begin{enumerate}
  \item \textbf{Entrenamiento básico}: Dataset individual, rápido para pruebas.
  \item \textbf{Entrenamiento GPU}: Acelerado con GPU (si está disponible).
  \item \textbf{Entrenamiento con muestreo}: Subset de 100k muestras, muy rápido.
  \item \textbf{Dataset unificado balanceado}: Múltiples datasets CIC-IDS2018, recomendado para producción.
\end{enumerate}

\subsection*{Instalación manual (Solo si no funciona el método automático)}
\begin{lstlisting}[style=tfgbash,caption={Instalación manual de dependencias},label=List.ManualInstall]
# Instalar dependencias básicas
pip install reflex pandas scikit-learn joblib scapy numpy

# Dependencias adicionales para entrenamiento
pip install imbalanced-learn tqdm matplotlib seaborn

# GPU (opcional, solo si tienes NVIDIA GPU)
pip install torch torchvision torchaudio --index-url https://download.pytorch.org/whl/cu118
\end{lstlisting}

\subsection*{Entrenamiento manual (Opción avanzada)}
\textbf{Solo usar si necesitas control específico sobre el entrenamiento:}

\subsubsection*{Entrenamiento básico}
\begin{lstlisting}[style=tfgbash,caption={Entrenamiento con dataset individual},label=List.BasicTrain]
# Dataset individual (requiere un archivo CSV de CIC-IDS2018 y es más rápido)
python training/cicidstrainer_optimized.py

# Solo CPU (forzar sin GPU)
python training/cicidstrainer_optimized.py --cpu-only

# Con muestreo rápido
python training/cicidstrainer_optimized.py --sample=100000
\end{lstlisting}

\subsubsection*{Entrenamiento con dataset unificado}
\textbf{IMPORTANTE}: Para usar el dataset unificado balanceado, primero debe crearse:

\begin{lstlisting}[style=tfgbash,caption={Crear y entrenar con dataset unificado},label=List.UnifiedTrain]
# PASO 1: Crear dataset unificado balanceado (obligatorio)
python training/unify_datasets.py

# PASO 2: Entrenar con dataset unificado
python training/cicidstrainer_optimized.py --unified
\end{lstlisting}

\textbf{¿Qué hace cada script?}
\begin{itemize}
  \item \texttt{training/unify\_datasets.py}: Combina múltiples archivos CIC-IDS2018, balancea clases con SMOTE+Undersampling, genera \texttt{cic\_ids\_unified\_balanced.csv}
  \item \texttt{training/cicidstrainer\_optimized.py --unified}: Entrena con el archivo balanceado creado en el paso anterior
\end{itemize}

\subsection*{Verificación de artefactos}
Tras cualquier entrenamiento exitoso, se generan estos archivos en \texttt{models/}:

\begin{lstlisting}[style=tfgbash,caption={Verificar modelo entrenado},label=List.VerifyModel]
# Verificar archivos generados
dir *.pkl                              # Windows
# ls *.pkl                             # Linux/Mac

# Probar modelo con muestras conocidas
python analisis/test_ids_model.py
\end{lstlisting}

\textbf{Archivos generados en \texttt{models/}:}
\begin{itemize}
  \item \texttt{cic\_ids\_model.pkl}: Modelo Random Forest entrenado
  \item \texttt{cic\_ids\_scaler.pkl}: Normalizador StandardScaler
  \item \texttt{feature\_mapping.pkl}: Mapeo de características
  \item \texttt{cic\_features.pkl}: Orden de características
\end{itemize}

\section{Manual de usuario}
En esta sección, explicaremos una vez que hayamos descargado cada una de las dependencias y paquetes necesarios, como podremos poner nuestro IDS en marcha.

El proyecto IDS Web está organizado en una estructura modular que facilita el mantenimiento y escalabilidad:
\subsection{Arquitectura de directorios del proyecto}

En primer lugar, debemos de conocer que el proyecto IDS Web está organizado en una estructura modular que facilita el mantenimiento y escalabilidad:
\begin{lstlisting}[style=tfgbash,caption={Estructura de directorios del proyecto},label=List.ProjectStructure]
ids_web/
├── analysis/                    # Scripts de análisis y testing
│   ├── test_ids_model.py       # Pruebas del modelo entrenado
│   ├── debug_dataset.py        # Debug y verificación de datasets
│   └── debug_verify.py         # Verificación general del sistema
├── models/                      # Artefactos del modelo entrenado
│   ├── cic_ids_model.pkl       # Modelo Random Forest
│   ├── cic_ids_scaler.pkl      # Normalizador StandardScaler
│   ├── feature_mapping.pkl     # Mapeo de características
│   ├── cic_features.pkl        # Orden de características
│   └── cic_ids_unified_balanced.csv # Dataset balanceado
├── training/                    # Scripts de entrenamiento
│   ├── cicidstrainer_optimized.py  # Entrenador principal
│   ├── unify_datasets.py       # Unificador de datasets
│   └── setup_system.py         # Configurador automático
├── utils/                       # Utilidades del sistema
│   └── gpu_deps.py             # Dependencias GPU
├── core/                        # Lógica central del sistema
│   ├── state.py                # Estado de la aplicación
│   └── cicidspredictor.py      # Predictor de ataques
├── ids_web.py                  # Aplicación web principal
├── requirements.txt            # Dependencias del proyecto
└── rxconfig.py                 # Configuración de Reflex
\end{lstlisting}

\textbf{Descripción de componentes:}
\begin{itemize}
  \item \texttt{ids\_web.py}: Interfaz web principal con dashboard en tiempo real
  \item \texttt{core/}: Contiene la lógica central (estado y predictor)
  \item \texttt{training/}: Scripts para entrenamiento y configuración automática
  \item \texttt{models/}: Almacena todos los artefactos del modelo entrenado
  \item \texttt{analysis/}: Herramientas de testing y verificación
  \item \texttt{utils/}: Utilidades compartidas (dependencias GPU, etc.)
\end{itemize}

\subsection*{Lanzamiento del IDS web}
Una vez hayamos completado la instalación, el entrenamiento y tengamos un conocimiento previo de la jerarquía de directorios, procederemos al lanzamiento de la aplicación:

\begin{lstlisting}[style=tfgbash,caption={Iniciar aplicación web},label=List.WebLaunch]
# Activar entorno virtual (si no está activo)
.\reflex_env\Scripts\activate        # Windows

# Linux/Mac:
# Desconozco como se activará porque no me he ceñido en concreto en los demás sistemas operativos pero será bastante similar.

# Iniciar aplicación web
cd ids_web (Nos movemos al directorio del proyecto)
reflex run
\end{lstlisting}

\textbf{Acceso}: La aplicación estará disponible en \texttt{http://localhost:3000}

\subsection*{Funcionalidades de la interfaz web}
\begin{itemize}
  \item \textbf{Dashboard principal}: Visualización en tiempo real de detecciones de ataques
  \item \textbf{Control de captura}: Botones Iniciar/Parar/Limpiar en la cabecera
  \item \textbf{Selección de interfaz}: Dropdown con interfaces de red disponibles
  \item \textbf{Filtros BPF}: Campo opcional para filtros Berkeley Packet Filter
  \item \textbf{Actualización automática}: Cada 2 segundos durante la captura activa
  \item \textbf{Estadísticas}: Contadores de flujos benignos vs maliciosos detectados
\end{itemize}

\subsection*{Captura por línea de comandos}
Por otra parte, el proyecto incluye una CLI para captura y creación de datasets, la meta principal de este proyecto. En la versión actual, cuando se usa modo dataset, se indica el formato \texttt{CSV} o \texttt{TXT} mediante las flags y se pasa la salida (ruta de fichero).
\paragraph{Captura en vivo a CSV/TXT (requiere permisos)}

\begin{lstlisting}[style=tfgbash,caption={Comandos de captura CLI},label=List.CLICapture]
# CSV
capturador -i eth0 -c flows.csv

# TXT
capturador -i eth0 -t flows.txt
python capturador -h
\end{lstlisting}

\paragraph{Procesar un PCAP y convertirlo a flujos}
\begin{lstlisting}[style=tfgbash,caption={PCAP -> CSV},label=List.PcapCSV]
capturador -f example.pcap -c flows.csv
\end{lstlisting}

\paragraph{Con mayor detalle (logs)}
\begin{lstlisting}[style=tfgbash,caption={Añadir verbosidad},label=List.Verbose]
capturador -i eth0 -c flows.csv -v
\end{lstlisting}

En Windows, las interfaces pueden tener espacios (p.\,ej., \texttt{"Wi-Fi"}). Use comillas si es necesario (en mi caso, si lo ha sido):
\begin{lstlisting}[style=tfgbash,caption={Ejemplo con interfaz con espacios (Windows)},label=List.WinIface]
capturador -i "Wi-Fi" -c flows.csv
\end{lstlisting}

\paragraph{Nota sobre ayudas y nombres de interfaz}
\begin{lstlisting}[style=tfgbash,caption={Ayuda de la CLI},label=List.CLIHelp]
capturador -h
\end{lstlisting}

\subsection*{Análisis y testing}
\begin{lstlisting}[style=tfgbash,caption={Scripts de análisis disponibles},label=List.Analysis]
# Probar modelo con muestras de ataques conocidos
python analisis/test_ids_model.py

# Debug y verificación del dataset
python analisis/debug_dataset.py

# Verificar funcionamiento general
python analisis/debug_verify.py
\end{lstlisting}

\subsection*{Solución de problemas comunes}
\begin{itemize}
  \item \textbf{Error "Modelo no encontrado"}: Ejecutar \texttt{python training/setup\_system.py} para generar artefactos.
  \item \textbf{Error "Dataset no encontrado" con \texttt{--unified}}: Ejecutar primero \texttt{python training/unify\_datasets.py}.
  \item \textbf{Sin permisos de captura}: Ejecutar terminal como administrador (Windows) o usar \texttt{sudo} (Linux).
  \item \textbf{Interfaz de red no detectada}: Verificar nombre exacto con \texttt{ipconfig} (Windows) o \texttt{ifconfig} (Linux).
  \item \textbf{GPU no detectada}: Instalar CUDA toolkit y drivers NVIDIA apropiados.
  \item \textbf{Lentitud o cortes}: reduzca el filtro BPF, ajuste \texttt{EXPIRED\_UPDATE}/\texttt{FLOW\_DURATION}/\texttt{MAX\_COLLECT\_PACKETS} y/o utilice hardware más potente. Actualmente, con los parámetros que están por defecto en el archivo \textbf{constantes.py} funciona correctamente.
  \item \textbf{Error por columnas al inferir}: verifique que \texttt{models/cic\_features.pkl} y \texttt{models/feature\_mapping.pkl} estén presentes; los nombres y el orden de columnas deben ser exactamente los del entrenamiento.
\end{itemize}

\subsection*{Flujo de trabajo recomendado}
\begin{enumerate}
  \item \textbf{Primera vez}: \texttt{python training/setup\_system.py} (instala todo y entrena).
  \item \textbf{Uso diario}: \texttt{reflex run} (lanza aplicación web). Este comando se utilizará como uso diario una vez que ya hayamos cargado y entrenado el modelo con \texttt{python training/setup\_system.py}.
  \item \textbf{Re-entrenamiento}: \texttt{python training/cicidstrainer\_optimized.py --unified} (si quieres modelo actualizado).
  \item \textbf{Testing}: \texttt{python analisis/test\_ids\_model.py} (verificar precisión del modelo).
\end{enumerate}

\clearpage\thispagestyle{empty}\cleardoublepage

\chapter{DEFINICIONES Y ABREVIATURAS}

A continuación, se presenta un glosario de términos clave, acrónimos y abreviaturas utilizados a lo largo de esta memoria, con el fin de facilitar la comprensión de la terminología técnica del proyecto.

\begin{itemize}
\item \textbf{API} (\textit{Application Programming Interface}): Conjunto de reglas y especificaciones que un software debe seguir para interactuar con otro.
\item \textbf{BPF} (\textit{Berkeley Packet Filter}): Tecnología que permite filtrar paquetes de red para su análisis de forma eficiente.

\item \textbf{IA} (\textit{Inteligencia Articial}): Es la simulación de la inteligencia humana en máquinas, permitiéndoles percibir, razonar, aprender, tomar decisiones y resolver problemas.

\item \textbf{CLI} (\textit{Command-Line Interface}): Interfaz de usuario basada en texto para interactuar con un programa o sistema operativo.
\item \textbf{CSV} (\textit{Comma-Separated Values}): Formato de archivo de texto plano que utiliza comas para separar los valores, comúnmente usado para el intercambio de datos tabulares.

\item \textbf{SVG} (\textit{Scalable Vector Graphics}): Formato de imagen basado en XML para describir gráficos vectoriales bidimensionales. Permite crear imágenes que pueden escalarse a cualquier tamaño sin pérdida de calidad, y es ampliamente utilizado para gráficos, íconos, diagramas y animaciones.

\item \textbf{CPU} (\textit{Central Processing Unit}): Unidad central de procesamiento, el "cerebro" de un ordenador.
\item \textbf{DDoS} (\textit{Distributed Denial of Service}): Ataque de denegación de servicio distribuido que busca sobrecargar un servidor o red con una gran cantidad de tráfico.


\item \textbf{Deep Learning} (\textit{Aprendizaje Profundo}): Subcampo del \textit{Machine Learning} que utiliza redes neuronales artificiales de múltiples capas para aprender representaciones de datos.
\item \textbf{Dashboard}: Panel de control visual que presenta información clave de forma gráfica y resumida.

\item \textbf{Flujo}: Secuencia de paquetes de red que comparten un conjunto común de atributos, como las direcciones IP y puertos.
\item \textbf{Feature} (\textit{Característica}): Atributo o variable en un conjunto de datos que se utiliza para el entrenamiento de modelos.

\item \textbf{Git}: Sistema de control de versiones distribuido, ampliamente utilizado para el seguimiento de cambios en el código fuente.
\item \textbf{GitHub}: Plataforma de alojamiento de código para control de versiones utilizando \textit{Git}.
\item \textbf{GPU} (\textit{Graphics Processing Unit}): Unidad de procesamiento gráfico, optimizada para el procesamiento paralelo, lo que la hace ideal para el entrenamiento de modelos de \textit{Machine Learning}.
\item \textbf{IDS} (\textit{Intrusion Detection System}): Sistema de detección de intrusiones que monitoriza la red en busca de actividades maliciosas.

\item \textbf{KNN} (\textit{K-Nearest Neighbors}): Algoritmo de clasificación y regresión no paramétrico que asigna un objeto a la clase más común entre sus k vecinos más cercanos.

\item \textbf{Machine Learning (ML)} (\textit{Aprendizaje Automático}): Rama de la Inteligencia Artificial que permite a las computadoras aprender de los datos sin ser programadas explícitamente.

\item \textbf{Mockups}: Representaciones visuales estáticas del diseño de una interfaz de usuario.

\item \textbf{Naive Bayes}: Algoritmo de clasificación probabilístico basado en el teorema de Bayes.

\item \textbf{NumPy}: Librería de \textit{Python} que añade soporte para matrices y arreglos multidimensionales.

\item \textbf{Pandas}: Librería de \textit{Python} para la manipulación y el análisis de datos, especialmente a través de \textit{DataFrames}.

\item \textbf{Pipeline}: Cadena de procesamiento de datos, donde la salida de una etapa es la entrada de la siguiente.

\item \textbf{Python}: Lenguaje de programación robusto con un amplio ecosistema de librerías para \textit{Machine Learning}.

\item \textbf{Random Forest}: Algoritmo de clasificación y regresión que construye múltiples árboles de decisión y los fusiona para obtener una predicción más precisa.

\item \textbf{Scrum}: Marco de trabajo para la gestión ágil de proyectos de software.
\item \textbf{Sprint}: Periodos cortos y definidos de tiempo en los que se debe completar una cierta cantidad de trabajo en un proyecto Scrum.

\item \textbf{SQL Injection}: Tipo de ciberataque en el que se inserta código malicioso en una consulta SQL para manipular una base de datos.
\item\textbf{hash}: Función que convierte datos en una cadena de longitud fija, usada para verificación y seguridad.
\item \textbf{SVM} (\textit{Support Vector Machine}): Algoritmo de aprendizaje automático supervisado utilizado para la clasificación y regresión.

\item \textbf{UI} (\textit{User Interface}): Interfaz de usuario, los medios por los cuales un usuario interactúa con un sistema.

\item\textbf{Diagrama de Gantt}:Herramienta de gestión de proyectos que muestra el cronograma de las tareas.
\item \textbf{Wireframe}: Un esquema visual de la estructura básica de una página web o aplicación, sin detalles de estilo.

\item \textbf{Wireshark}: Herramienta popular para el análisis de tráfico de red, que permite la captura y el examen de paquetes.

\end{itemize}
\clearpage\thispagestyle{empty}\cleardoublepage

% ===== Configuración de la bibliografía

\pagenumbering{roman}  % Numeración romana para el índice
\bibliographystyle{unsrtnat}
\bibliography{bibliografia}
\addcontentsline{toc}{chapter}{Bibliografía}

\clearpage\thispagestyle{empty}
\begin{center}
{\color{flashwhite} Generado con la plantilla \LaTeX \textit{TFG Dpto. Informática - EPSJ - v1.1} \par Licencia CC0 1.0 Universal (CC0 1.0) - Francisco Charte Ojeda}
\end{center}

\end{document}