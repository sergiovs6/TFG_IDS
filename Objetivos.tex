\chapter{OBJETIVOS}

La creciente digitalización de la sociedad y la dependencia crítica de las infraestructuras de red han propiciado un incremento exponencial en el volumen y la sofisticación de las amenazas cibernéticas. Este panorama dinámico exige la evolución constante de los sistemas de seguridad, trascendiendo las soluciones reactivas para adoptar enfoques proactivos y adaptativos. En este contexto, la detección de intrusiones se posiciona como una piedra angular en la estrategia de defensa de cualquier organización. Los Sistemas de Detección de Intrusiones (IDS) tradicionales, basados principalmente en firmas, aunque eficaces contra amenazas conocidas, presentan limitaciones intrínsecas ante ataques emergentes o variantes polimórficas para las cuales no existen patrones predefinidos. Esta brecha de seguridad motiva la exploración de paradigmas más avanzados, como el aprendizaje automático (Machine Learning), que ofrecen la capacidad de identificar comportamientos anómalos y patrones de ataque previamente desconocidos.

La motivación principal de este Proyecto de Fin de Grado surge de la necesidad imperante de fortalecer la resiliencia de las redes frente a un entorno de amenazas en constante evolución. Se busca contribuir al campo de la ciberseguridad mediante el desarrollo de una solución práctica y académicamente relevante que integre las capacidades de la observación en tiempo real del tráfico de red con la inteligencia predictiva de los modelos de Machine Learning. Además, se reconoce la dificultad y el tiempo que implica la creación de conjuntos de datos de tráfico de red etiquetados para el entrenamiento y evaluación de modelos de IDS. Esta problemática subyacente proporciona una motivación adicional para desarrollar herramientas que faciliten la generación de dichos datasets.

\section{Objetivos Generales}
El objetivo general de este proyecto es diseñar, desarrollar e implementar un Sistema de Detección de Intrusiones (IDS) basado en aprendizaje automático, capaz de monitorear el tráfico de red en tiempo real y clasificarlo eficientemente como benigno o malicioso, contribuyendo así a la mejora de la postura de seguridad cibernética.

\section{Objetivos Específicos}
Para alcanzar el objetivo general, se han definido los siguientes objetivos específicos:

Diseñar y Desarrollar una Herramienta de Captura y Procesamiento de Tráfico de Red en Tiempo Real: Crear una aplicación robusta y eficiente, capaz de interceptar paquetes de red en vivo, reconstruir flujos de conexión y extraer un conjunto exhaustivo de características relevantes para el análisis de seguridad. Esta herramienta debe ser programable para generar salidas estructuradas (ej., CSV, TXT) que emulen el formato de datasets de referencia en ciberseguridad, como los de la serie CIC-IDS, facilitando así la creación de nuevas bases de datos para la investigación y el desarrollo de IDS.

Implementar y Entrenar un Modelo de Clasificación de Machine Learning: Seleccionar, configurar y entrenar un algoritmo de aprendizaje automático, específicamente Random Forest, para la tarea de clasificación binaria (tráfico normal vs. tráfico malicioso). Este modelo debe ser capaz de aprender los patrones distintivos de diversos tipos de ataques a partir de datasets de tráfico etiquetado, optimizando su rendimiento para minimizar tanto los falsos positivos como, crucialmente, los falsos negativos.

Integrar el Modelo de Detección con la Herramienta de Captura para una Operación Continua: Establecer un pipeline funcional que permita a la herramienta de captura alimentar los datos preprocesados al modelo de Machine Learning en tiempo real, permitiendo que el IDS clasifique el tráfico de forma dinámica y emita alertas ante la detección de comportamientos anómalos o intrusiones.

Desarrollar una Interfaz Web Interactiva para la Visualización y Monitorización: Crear un dashboard basado en una aplicación web que proporcione una representación clara e intuitiva de la actividad de red, las detecciones realizadas por el modelo de Machine Learning y las métricas de rendimiento clave. Esta interfaz mejorará la usabilidad del sistema y permitirá a los usuarios monitorizar eficazmente el estado de la red.

Evaluar Rigurosamente el Rendimiento del IDS Propuesto: Realizar pruebas exhaustivas del sistema implementado utilizando métricas de evaluación adecuadas para problemas de clasificación en ciberseguridad (ej., precisión, sensibilidad, F1-score, matriz de confusión, AUC-ROC). El objetivo es cuantificar la eficacia del IDS en la detección de diferentes tipos de ataques y demostrar su viabilidad como una solución de seguridad.
