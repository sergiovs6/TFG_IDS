\chapter{RESULTADOS}\label{ch:resultados}

\section{Evaluación del Modelo de Detección de Intrusiones}

En este capítulo se presentan los resultados obtenidos de la evaluación del modelo Random Forest entrenado sobre el dataset CIC-IDS2018. El análisis comprende métricas generales de rendimiento, evaluación por tipo de ataque específico y un estudio detallado de las limitaciones identificadas.

\subsection{Métricas Generales de Rendimiento}

El modelo fue evaluado sobre un conjunto de prueba de \textbf{309,523 muestras}, manteniendo la proporción original del dataset balanceado. La Tabla \ref{tab:main_results} presenta las métricas principales obtenidas.

\begin{table}[H]
\centering
\begin{tabular}{lr}
\toprule
\textbf{Métrica} & \textbf{Valor} \\
\midrule
Accuracy & 0.9293 (92.93\%) \\
Precision & 0.9981 (99.81\%) \\
Recall & 0.9185 (91.85\%) \\
F1-Score & 0.9566 (95.66\%) \\
AUC-ROC & 0.9851 (98.51\%) \\
\bottomrule
\end{tabular}
\caption{Métricas principales del modelo Random Forest en CIC-IDS2018}
\label{tab:main_results}
\end{table}

Los resultados demuestran un rendimiento excepcional del modelo, con una precisión del 99.81\% que indica una tasa extremadamente baja de falsos positivos. El valor de AUC-ROC de 0.9851 confirma la excelente capacidad discriminatoria del modelo entre tráfico benigno y malicioso.

\subsection{Reporte de Clasificación Detallado}

La Tabla \ref{tab:classification_report} presenta el reporte detallado de clasificación por clase:

\begin{table}[H]
\centering
\begin{tabular}{lrrrr}
\toprule
\textbf{Clase} & \textbf{Precision} & \textbf{Recall} & \textbf{F1-Score} & \textbf{Support} \\
\midrule
Benign & 0.6832 & 0.9902 & 0.8085 & 46,667 \\
Attack & 0.9981 & 0.9185 & 0.9566 & 262,856 \\
\midrule
\textbf{Accuracy} & & & \textbf{0.9293} & \textbf{309,523} \\
\textbf{Macro avg} & \textbf{0.8407} & \textbf{0.9544} & \textbf{0.8826} & \textbf{309,523} \\
\textbf{Weighted avg} & \textbf{0.9506} & \textbf{0.9293} & \textbf{0.9343} & \textbf{309,523} \\
\bottomrule
\end{tabular}
\caption{Reporte de clasificación detallado del modelo Random Forest}
\label{tab:classification_report}
\end{table}

\subsection{Matriz de Confusión}

La matriz de confusión detallada (Tabla \ref{tab:confusion_matrix}) proporciona una visión granular del comportamiento del modelo:

\begin{table}[H]
\centering
\begin{tabular}{c|cc|c}
\toprule
& \multicolumn{2}{c|}{\textbf{Predicción}} & \\
\textbf{Real} & Benign & Attack & \textbf{Total} \\
\midrule
Benign & 46,211 & 456 & 46,667 \\
Attack & 21,428 & 241,428 & 262,856 \\
\midrule
\textbf{Total} & 67,639 & 241,884 & 309,523 \\
\bottomrule
\end{tabular}
\caption{Matriz de confusión del modelo Random Forest}
\label{tab:confusion_matrix}
\end{table}

La Figura \ref{fig:confusion_matrix} proporciona una visualización más intuitiva de estos resultados mediante un mapa de calor que facilita la interpretación de los datos.

\begin{figure}[H]
\centering
\includegraphics[width=0.8\textwidth]{imagenes/diagramas/resultados/confusion_matrix.png}
\caption{Visualización de la matriz de confusión del modelo Random Forest.}
\label{fig:confusion_matrix}
\end{figure}
\paragraph{Explicación}
Los colores más intensos indican mayor concentración de predicciones. Se observa una alta concentración en la diagonal principal (predicciones correctas) con 46,211 casos benignos correctamente clasificados y 241,428 ataques detectados correctamente. Los falsos positivos (456) y falsos negativos (21,428) aparecen en menor intensidad fuera de la diagonal.

\textbf{Análisis de errores:}
\begin{itemize}
\item \textbf{Falsos positivos}: 456 casos (0.98\% del tráfico benigno)
\item \textbf{Falsos negativos}: 21,428 casos (8.15\% de los ataques)
\item \textbf{Tasa de error global}: 7.07\%
\end{itemize}

El modelo muestra una tendencia conservadora, priorizando la minimización de falsos positivos a costa de algunos falsos negativos, lo cual es apropiado para sistemas de detección de intrusiones en entornos de producción.

\subsection{Curvas de Rendimiento}

\paragraph{Curva ROC}

La Figura \ref{fig:roc_curve} muestra la curva ROC del modelo, que ilustra la relación entre la tasa de verdaderos positivos y la tasa de falsos positivos.

\begin{figure}[H]
\centering
\includegraphics[width=0.8\textwidth]{imagenes/diagramas/resultados/roc_curve.png}
\caption{Curva ROC del modelo Random Forest.}
\label{fig:roc_curve}
\end{figure}

\paragraph{Explicación}
El área bajo la curva (AUC = 0.9851) indica un rendimiento excelente. La curva se aproxima mucho a la esquina superior izquierda, lo que demuestra que el modelo logra altas tasas de detección (verdaderos positivos) manteniendo bajas tasas de falsos positivos. La línea diagonal representa el rendimiento de un clasificador aleatorio.

\paragraph{Curva Precision-Recall}

La Figura \ref{fig:precision_recall} presenta la curva Precision-Recall, especialmente relevante para datasets desbalanceados como el de detección de intrusiones.

\begin{figure}[H]
\centering
\includegraphics[width=0.8\textwidth]{imagenes/diagramas/resultados/precision_recall_curve.png}
\caption{Curva Precision-Recall del modelo Random Forest.}
\label{fig:precision_recall}
\end{figure}

\paragraph{Explicación}
Esta curva es particularmente importante en detección de intrusiones debido al desbalance de clases. El modelo mantiene alta precisión (>95\%) incluso con altos valores de recall, indicando que puede detectar la mayoría de ataques sin generar excesivas alarmas falsas. El área bajo esta curva complementa la información de la curva ROC.

\subsection{Distribución de Probabilidades}

La Figura \ref{fig:probability_distribution} muestra cómo el modelo asigna probabilidades a las diferentes clases, proporcionando insights sobre la confianza de las predicciones.

\begin{figure}[H]
\centering
\includegraphics[width=0.9\textwidth]{imagenes/diagramas/resultados/probability_distribution.png}
\caption{Distribución de probabilidades por clase real del modelo Random Forest.}
\label{fig:probability_distribution}
\end{figure}

\paragraph{Explicación}
La gráfica revela un comportamiento característico: el tráfico benigno (verde) se concentra fuertemente en probabilidades bajas (cercanas a 0), indicando alta confianza en su clasificación como no-ataque. Los ataques (rojo) muestran una distribución bimodal con una concentración masiva cerca de la probabilidad 1.0, reflejando la alta confianza del modelo para detectar la mayoría de ataques. El pico pronunciado del rojo en 1.0 explica la alta precisión del modelo (99.81\%).

Esta distribución explica varios aspectos clave del rendimiento:
\begin{itemize}
\item La \textbf{concentración verde en 0.0-0.2} indica que el modelo identifica claramente el tráfico benigno
\item El \textbf{pico rojo masivo en 1.0} demuestra alta confianza en la detección de ataques
\item La \textbf{separación clara} entre las distribuciones confirma la capacidad discriminatoria del modelo
\item Las \textbf{pequeñas superposiciones} en el rango medio (0.3-0.7) corresponden a casos de mayor incertidumbre
\end{itemize}

\subsection{Distribución del Dataset Completo}

El análisis por tipo de ataque se realizó sobre el dataset completo de \textbf{1,547,611 muestras} que incluye 9 tipos diferentes de ataques. La Tabla \ref{tab:dataset_distribution} muestra la distribución original:

\begin{table}[H]
\centering
\begin{tabular}{lrr}
\toprule
\textbf{Tipo de Ataque} & \textbf{Muestras} & \textbf{Porcentaje} \\
\midrule
Benign & 233,332 & 15.08\% \\
FTP-BruteForce & 193,360 & 12.49\% \\
SSH-Bruteforce & 187,589 & 12.12\% \\
Brute Force Web & 155,555 & 10.05\% \\
Brute Force XSS & 155,555 & 10.05\% \\
DoS attacks-Slowloris & 155,555 & 10.05\% \\
DoS attacks-GoldenEye & 155,555 & 10.05\% \\
Infiltration & 155,555 & 10.05\% \\
SQL Injection & 155,555 & 10.05\% \\
\midrule
\textbf{Total} & \textbf{1,547,611} & \textbf{100.00\%} \\
\bottomrule
\end{tabular}
\caption{Distribución original del dataset CIC-IDS2018 balanceado}
\label{tab:dataset_distribution}
\end{table}

\subsection{Análisis por Tipo de Ataque}

La evaluación detallada por tipo de ataque revela variaciones significativas en la capacidad de detección del modelo (Tabla \ref{tab:attack_detection}):

\begin{table}[H]
\centering
\begin{tabular}{lrrrr}
\toprule
\textbf{Tipo de Ataque} & \textbf{Total} & \textbf{Detectado} & \textbf{Tasa} & \textbf{Confianza} \\
\midrule
FTP-BruteForce & 193,360 & 193,360 & 100.0\% & 0.998 \\
SSH-Bruteforce & 187,589 & 187,586 & 100.0\% & 0.999 \\
Brute Force XSS & 155,555 & 155,555 & 100.0\% & 0.989 \\
DoS GoldenEye & 155,555 & 155,554 & 100.0\% & 0.999 \\
SQL Injection & 155,555 & 155,521 & 100.0\% & 0.977 \\
Brute Force Web & 155,555 & 155,538 & 100.0\% & 0.977 \\
DoS Slowloris & 155,555 & 155,059 & 99.7\% & 0.995 \\
Benign & 233,332 & 232,200 & 99.5\% & 0.914 \\
Infiltration & 155,555 & 51,094 & 32.8\% & 0.445 \\
\midrule
\textbf{Promedio ponderado} & \textbf{1,547,611} & \textbf{1,441,467} & \textbf{93.14\%} & \textbf{0.911} \\
\bottomrule
\end{tabular}
\caption{Tasa de detección por tipo de ataque específico}
\label{tab:attack_detection}
\end{table}

La Figura \ref{fig:attack_analysis} proporciona una visualización comprehensiva del rendimiento por tipo de ataque en múltiples dimensiones.

\begin{figure}[H]
\centering
\includegraphics[width=1.0\textwidth]{imagenes/diagramas/resultados/attack_type_analysis.png}
\caption{Análisis completo por tipo de ataque del modelo Random Forest.}
\label{fig:attack_analysis}
\end{figure}

\paragraph{Explicación}
El panel superior izquierdo muestra las tasas de detección, donde se observa un rendimiento excelente (>99\%) para la mayoría de ataques excepto infiltración (32.8\%). El panel superior derecho presenta la distribución de muestras, mostrando el balanceamiento del dataset. El panel inferior izquierdo ilustra la confianza promedio del modelo, donde infiltración muestra la menor confianza (0.445). El panel inferior derecho destaca los errores absolutos, donde infiltración presenta significativamente más errores que otros tipos.
 
La Figura \ref{fig:attack_heatmap} complementa este análisis con un mapa de calor que facilita la comparación entre tipos de ataque.

\begin{figure}[H]
\centering
\includegraphics[width=0.9\textwidth]{imagenes/diagramas/resultados/attack_performance_heatmap.png}
\caption{Mapa de calor del rendimiento por tipo de ataque.}
\label{fig:attack_heatmap}
\end{figure}

\paragraph{Explicación}
Los colores más cálidos (rojos/naranjas) indican mejor rendimiento. Se observa claramente que los ataques de fuerza bruta y DoS muestran rendimiento excelente (colores rojos intensos), mientras que infiltración presenta rendimiento deficiente (colores más fríos). Las tres métricas mostradas son: tasa de detección, confianza promedio y volumen de muestras normalizado.

\subsection{Análisis de Errores por Tipo de Ataque}

La Tabla \ref{tab:error_analysis} presenta un desglose detallado de los errores por tipo de ataque:

\begin{table}[H]
\centering
\begin{tabular}{lrrr}
\toprule
\textbf{Tipo de Ataque} & \textbf{Total Muestras} & \textbf{Errores} & \textbf{Tasa de Error} \\
\midrule
Benign & 233,332 & 1,132 & 0.5\% \\
SSH-Bruteforce & 187,589 & 3 & 0.0\% \\
Brute Force Web & 155,555 & 17 & 0.0\% \\
DoS attacks-Slowloris & 155,555 & 496 & 0.3\% \\
DoS attacks-GoldenEye & 155,555 & 1 & 0.0\% \\
SQL Injection & 155,555 & 34 & 0.0\% \\
Infiltration & 155,555 & 104,461 & 67.2\% \\
FTP-BruteForce & 193,360 & 0 & 0.0\% \\
Brute Force XSS & 155,555 & 0 & 0.0\% \\
\midrule
\textbf{Total} & \textbf{1,547,611} & \textbf{106,144} & \textbf{6.86\%} \\
\bottomrule
\end{tabular}
\caption{Análisis de errores por tipo de ataque}
\label{tab:error_analysis}
\end{table}

\textbf{Categorización del rendimiento:}

\textbf{Excelente detección ($\geq$99.5\%):}
\begin{itemize}
\item Ataques de fuerza bruta (FTP, SSH, Web, XSS): 100\% de detección
\item Ataques de denegación de servicio: 99.7-100\%
\item Inyección SQL: 100\%
\item Tráfico benigno: 99.5\%
\end{itemize}

\textbf{Detección problemática (<50\%):}
\begin{itemize}
\item Ataques de infiltración: 32.8\% de detección (67.2\% de errores)
\end{itemize}

\subsection{Importancia de Características}

El análisis de importancia de características identifica los atributos más relevantes para la clasificación. La Tabla \ref{tab:feature_importance} presenta las 20 características más importantes:

\begin{table}[H]
\centering
\begin{tabular}{rlr}
\toprule
\textbf{Ranking} & \textbf{Característica} & \textbf{Importancia} \\
\midrule
1 & Init Fwd Win Byts & 0.1094 \\
2 & Fwd Seg Size Min & 0.1020 \\
3 & Fwd Header Len & 0.0505 \\
4 & Flow IAT Min & 0.0459 \\
5 & Init Bwd Win Byts & 0.0333 \\
6 & Bwd Pkts/s & 0.0279 \\
7 & Fwd Pkt Len Std & 0.0256 \\
8 & Subflow Fwd Byts & 0.0255 \\
9 & TotLen Fwd Pkts & 0.0253 \\
10 & Fwd Pkt Len Mean & 0.0230 \\
11 & Flow Byts/s & 0.0224 \\
12 & Fwd Pkts/s & 0.0217 \\
13 & Flow IAT Max & 0.0201 \\
14 & Fwd IAT Min & 0.0194 \\
15 & Flow IAT Mean & 0.0193 \\
16 & Flow IAT Std & 0.0191 \\
17 & Flow Pkts/s & 0.0190 \\
18 & Tot Fwd Pkts & 0.0174 \\
19 & Tot Bwd Pkts & 0.0174 \\
20 & Fwd Act Data Pkts & 0.0169 \\
\bottomrule
\end{tabular}
\caption{Top 20 características más importantes según Random Forest}
\label{tab:feature_importance}
\end{table}

La Figura \ref{fig:feature_importance} visualiza estas importancias de manera más intuitiva.

\begin{figure}[H]
\centering
\includegraphics[width=0.9\textwidth]{imagenes/diagramas/resultados/feature_importance.png}
\caption{Importancia de características del modelo Random Forest.}
\label{fig:feature_importance}
\end{figure}

\paragraph{Explicación}
El gráfico muestra una distribución desigual donde las dos características principales (Init Fwd Win Byts y Fwd Seg Size Min) dominan con importancias de 10.94\% y 10.20\% respectivamente. Esto indica que el modelo se basa fuertemente en características relacionadas con el tamaño de ventana inicial y la segmentación de paquetes para distinguir entre tráfico benigno y malicioso. La disminución gradual sugiere que muchas características aportan información complementaria.

Las características relacionadas con el \textbf{tamaño de ventana inicial} y la \textbf{segmentación de paquetes} emergen como los discriminadores más potentes, representando conjuntamente el 21.14\% de la importancia total del modelo.

\subsection{Análisis Específico: Limitaciones en Ataques de Infiltración}

Dado el rendimiento significativamente inferior en la detección de ataques de infiltración (32.8\%), se realizó un análisis específico para identificar las causas subyacentes.

\paragraph{Distribución de Probabilidades}

El análisis de las probabilidades asignadas por el modelo reveló una distribución problemática:

\begin{itemize}
\item \textbf{Infiltración}: Promedio = 0.445, Desviación estándar = 0.327
\item \textbf{Tráfico benigno}: Promedio = 0.083, Desviación estándar = 0.090
\end{itemize}

La alta desviación estándar en ataques de infiltración (0.327) indica una \textbf{incertidumbre significativa} del modelo, contrastando con la confianza mostrada en tráfico benigno.

\paragraph{Análisis de Umbrales de Clasificación}

La evaluación bajo diferentes umbrales de clasificación (Tabla \ref{tab:threshold_analysis}) sugiere oportunidades de optimización:

\begin{table}[H]
\centering
\begin{tabular}{ccc}
\toprule
\textbf{Umbral} & \textbf{Detección Infiltración} & \textbf{Falsos Positivos Benigno} \\
\midrule
0.9 & 22.8\% & 0.1\% \\
0.7 & 26.4\% & 0.1\% \\
0.5 (actual) & 32.8\% & 0.3\% \\
0.3 & 53.6\% & 2.9\% \\
\bottomrule
\end{tabular}
\caption{Impacto del umbral de clasificación en la detección de infiltración}
\label{tab:threshold_analysis}
\end{table}

La Figura \ref{fig:infiltration_analysis} proporciona un análisis visual comprehensivo de las limitaciones en la detección de infiltración.

\begin{figure}[H]
\centering
\includegraphics[width=1.0\textwidth]{imagenes/diagramas/resultados/infiltration_analysis.png}
\caption{Análisis detallado de ataques de infiltración.}
\label{fig:infiltration_analysis}
\end{figure}

\paragraph{Explicación}
El panel superior izquierdo muestra las distribuciones de probabilidad superpuestas: infiltración (rojo) presenta una distribución más dispersa con menor concentración en valores altos, mientras que benigno (verde) se concentra fuertemente cerca de 0. El panel superior derecho ilustra cómo diferentes umbrales afectan la detección: umbrales más bajos mejoran la detección de infiltración. El panel inferior izquierdo identifica las características más distintivas, donde las diferencias temporales son predominantes. El boxplot inferior derecho confirma la mayor incertidumbre del modelo para infiltración.

Un \textbf{ajuste del umbral de 0.5 a 0.3} resultaría en:
\begin{itemize}
\item \textbf{Mejora del 63.4\%} en detección de infiltración (de 32.8\% a 53.6\%)
\item \textbf{Incremento tolerable} de falsos positivos (de 0.3\% a 2.9\%)
\end{itemize}

\paragraph{Características Distintivas de Infiltración}

El análisis comparativo entre ataques de infiltración y tráfico benigno identificó las 10 diferencias temporales más significativas (Tabla \ref{tab:infiltration_features}):

\begin{table}[H]
\centering
\begin{tabular}{lrrr}
\toprule
\textbf{Característica} & \textbf{Infiltración} & \textbf{Benigno} & \textbf{Diferencia} \\
\midrule
Fwd IAT Tot ($\mu$s) & 10,027,847 & 14,597,875 & -4,570,028 \\
Flow Duration ($\mu$s) & 10,461,854 & 14,964,644 & -4,502,789 \\
Flow IAT Max ($\mu$s) & 3,752,259 & 7,276,331 & -3,524,072 \\
Fwd IAT Max ($\mu$s) & 3,536,283 & 7,027,942 & -3,491,659 \\
Fwd IAT Mean ($\mu$s) & 1,308,751 & 4,475,634 & -3,166,882 \\
Flow IAT Mean ($\mu$s) & 1,021,423 & 4,121,081 & -3,099,658 \\
Fwd IAT Min ($\mu$s) & 622,602 & 3,697,850 & -3,075,248 \\
Flow IAT Min ($\mu$s) & 558,770 & 3,604,459 & -3,045,688 \\
Idle Min ($\mu$s) & 3,130,105 & 5,176,845 & -2,046,740 \\
Idle Mean ($\mu$s) & 3,346,631 & 5,362,075 & -2,015,444 \\
\bottomrule
\end{tabular}
\caption{Top 10 características temporales distintivas de ataques de infiltración}
\label{tab:infiltration_features}
\end{table}

\textbf{Hallazgo clave}: Los ataques de infiltración se caracterizan por \textbf{patrones temporales más compactos} que el tráfico benigno, con duraciones de flujo y intervalos entre paquetes sistemáticamente menores. Esta similitud con ciertos patrones de tráfico legítimo explica la dificultad del modelo para distinguirlos.

\subsection{Rendimiento Computacional}\label{res:rendimientocom}

El modelo demostró una eficiencia computacional adecuada para despliegue en producción:

\begin{itemize}
\item \textbf{Velocidad de procesamiento}: 152,761 muestras/segundo
\item \textbf{Tiempo promedio por muestra}: 0.007 ms
\item \textbf{Tiempo total para 309,523 muestras}: 2.028 segundos
\end{itemize}

Estos valores indican que el modelo es viable para análisis en tiempo real de tráfico de red de alta velocidad.

\section{Síntesis de Resultados}

El modelo Random Forest desarrollado demuestra un \textbf{rendimiento excepcional} en la detección de intrusiones, con una precisión global del 92.93\% y una precisión del 99.81\%. Las fortalezas principales incluyen:

\begin{enumerate}
\item \textbf{Detección perfecta} de ataques de fuerza bruta y denegación de servicio
\item \textbf{Tasa extremadamente baja} de falsos positivos (0.98\%)
\item \textbf{Eficiencia computacional} adecuada para tiempo real
\item \textbf{Identificación clara} de características discriminatorias
\end{enumerate}

La \textbf{limitación principal} radica en la detección de ataques de infiltración (32.8\%), atribuible a su similitud temporal con tráfico benigno. No obstante, esta limitación puede mitigarse mediante \textbf{ajuste de umbrales} o \textbf{enriquecimiento del conjunto de características} con información contextual adicional.

Los resultados posicionan al modelo como una \textbf{solución robusta y práctica} para sistemas de detección de intrusiones, con un balance favorable entre efectividad de detección y tasa de falsos positivos.

\subsection{Archivos de Soporte Generados}

Los siguientes archivos gráficos fueron generados para complementar el análisis:

\begin{itemize}
\item \texttt{confusion\_matrix.png} - Visualización de la matriz de confusión
\item \texttt{roc\_curve.png} - Curva ROC del modelo
\item \texttt{precision\_recall\_curve.png} - Curva Precision-Recall
\item \texttt{feature\_importance.png} - Importancia de características
\item \texttt{probability\_distribution.png} - Distribución de probabilidades por clase
\item \texttt{attack\_type\_analysis.png} - Análisis comparativo por tipo de ataque
\item \texttt{infiltration\_analysis.png} - Análisis específico de infiltración
\item \texttt{attack\_performance\_heatmap.png} - Mapa de calor del rendimiento
\end{itemize}