\chapter{DEFINICIONES Y ABREVIATURAS}

A continuación, se presenta un glosario de términos clave, acrónimos y abreviaturas utilizados a lo largo de esta memoria, con el fin de facilitar la comprensión de la terminología técnica del proyecto.

\begin{itemize}
\item \textbf{API} (\textit{Application Programming Interface}): Conjunto de reglas y especificaciones que un software debe seguir para interactuar con otro.
\item \textbf{BPF} (\textit{Berkeley Packet Filter}): Tecnología que permite filtrar paquetes de red para su análisis de forma eficiente.

\item \textbf{IA} (\textit{Inteligencia Articial}): Es la simulación de la inteligencia humana en máquinas, permitiéndoles percibir, razonar, aprender, tomar decisiones y resolver problemas.

\item \textbf{CLI} (\textit{Command-Line Interface}): Interfaz de usuario basada en texto para interactuar con un programa o sistema operativo.
\item \textbf{CSV} (\textit{Comma-Separated Values}): Formato de archivo de texto plano que utiliza comas para separar los valores, comúnmente usado para el intercambio de datos tabulares.

\item \textbf{SVG} (\textit{Scalable Vector Graphics}): Formato de imagen basado en XML para describir gráficos vectoriales bidimensionales. Permite crear imágenes que pueden escalarse a cualquier tamaño sin pérdida de calidad, y es ampliamente utilizado para gráficos, íconos, diagramas y animaciones.

\item \textbf{CPU} (\textit{Central Processing Unit}): Unidad central de procesamiento, el "cerebro" de un ordenador.
\item \textbf{DDoS} (\textit{Distributed Denial of Service}): Ataque de denegación de servicio distribuido que busca sobrecargar un servidor o red con una gran cantidad de tráfico.


\item \textbf{Deep Learning} (\textit{Aprendizaje Profundo}): Subcampo del \textit{Machine Learning} que utiliza redes neuronales artificiales de múltiples capas para aprender representaciones de datos.
\item \textbf{Dashboard}: Panel de control visual que presenta información clave de forma gráfica y resumida.

\item \textbf{Flujo}: Secuencia de paquetes de red que comparten un conjunto común de atributos, como las direcciones IP y puertos.
\item \textbf{Feature} (\textit{Característica}): Atributo o variable en un conjunto de datos que se utiliza para el entrenamiento de modelos.

\item \textbf{Git}: Sistema de control de versiones distribuido, ampliamente utilizado para el seguimiento de cambios en el código fuente.
\item \textbf{GitHub}: Plataforma de alojamiento de código para control de versiones utilizando \textit{Git}.
\item \textbf{GPU} (\textit{Graphics Processing Unit}): Unidad de procesamiento gráfico, optimizada para el procesamiento paralelo, lo que la hace ideal para el entrenamiento de modelos de \textit{Machine Learning}.
\item \textbf{IDS} (\textit{Intrusion Detection System}): Sistema de detección de intrusiones que monitoriza la red en busca de actividades maliciosas.

\item \textbf{KNN} (\textit{K-Nearest Neighbors}): Algoritmo de clasificación y regresión no paramétrico que asigna un objeto a la clase más común entre sus k vecinos más cercanos.

\item \textbf{Machine Learning (ML)} (\textit{Aprendizaje Automático}): Rama de la Inteligencia Artificial que permite a las computadoras aprender de los datos sin ser programadas explícitamente.

\item \textbf{Mockups}: Representaciones visuales estáticas del diseño de una interfaz de usuario.

\item \textbf{Naive Bayes}: Algoritmo de clasificación probabilístico basado en el teorema de Bayes.

\item \textbf{NumPy}: Librería de \textit{Python} que añade soporte para matrices y arreglos multidimensionales.

\item \textbf{Pandas}: Librería de \textit{Python} para la manipulación y el análisis de datos, especialmente a través de \textit{DataFrames}.

\item \textbf{Pipeline}: Cadena de procesamiento de datos, donde la salida de una etapa es la entrada de la siguiente.

\item \textbf{Python}: Lenguaje de programación robusto con un amplio ecosistema de librerías para \textit{Machine Learning}.

\item \textbf{Random Forest}: Algoritmo de clasificación y regresión que construye múltiples árboles de decisión y los fusiona para obtener una predicción más precisa.

\item \textbf{Scrum}: Marco de trabajo para la gestión ágil de proyectos de software.
\item \textbf{Sprint}: Periodos cortos y definidos de tiempo en los que se debe completar una cierta cantidad de trabajo en un proyecto Scrum.

\item \textbf{SQL Injection}: Tipo de ciberataque en el que se inserta código malicioso en una consulta SQL para manipular una base de datos.
\item\textbf{hash}: Función que convierte datos en una cadena de longitud fija, usada para verificación y seguridad.
\item \textbf{SVM} (\textit{Support Vector Machine}): Algoritmo de aprendizaje automático supervisado utilizado para la clasificación y regresión.

\item \textbf{UI} (\textit{User Interface}): Interfaz de usuario, los medios por los cuales un usuario interactúa con un sistema.

\item\textbf{Diagrama de Gantt}:Herramienta de gestión de proyectos que muestra el cronograma de las tareas.
\item \textbf{Wireframe}: Un esquema visual de la estructura básica de una página web o aplicación, sin detalles de estilo.

\item \textbf{Wireshark}: Herramienta popular para el análisis de tráfico de red, que permite la captura y el examen de paquetes.

\end{itemize}